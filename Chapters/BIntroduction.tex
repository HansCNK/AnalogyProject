\pagenumbering{arabic}
\chapter{Introduction}

    \section{Background}
		The present project is twofold. First, it gives a thorough representation of the concept of analogical reasoning in law by introducing and comparing contemporary theories regarding the subject. Second, it provides an independent analysis of analogical reasoning in the framework of immanent reasoning\index{immanent reasoning}. 
		
		The first part provides explanations and comparison of six contemporary theories of analogical reasoning in law. These theories are categorised into schema-based theories\index{schema-based theory} and inference-based theories\index{inference-based theory}. Schema-based theories\index{schema-based theory} of analogical reasoning capture the notion of analogy by a description of a rule or schema. Inference-based theories\index{inference-based theory} on the other hand explain analogical reasoning as a distinct way of reasoning. Based on this distinction, the project compares the theories by how they handle the notions of horizontal and vertical relation\index{relation!vertical}s and by how they analyse multiple\index{analogy!multiple} competing analogies. 
		
		The second part of the project provides an independent analysis of analogies by utilising the framework of immanent reasoning\index{immanent reasoning}. Immanent reasoning\index{immanent reasoning} is described, together with an explanation of other notions that are relevant for the given analysis. The project then introduces two kinds of analogical reasoning, general precedent-based reasoning and precedent-based reasoning with heteronomous imperatives\index{heteronomous imperatives}. These kinds are first analysed in the general formulation of constructive type theory (CTT) and is then given an alternative formulation in its dialogical interpretation. Following this, we introduce a discussion on the advantages of utilising this framework for analysing legal reasoning in general and analogical reasoning in particular.
		
		Based on a variant of the principle of proportionality\index{proportionality}, the present project provides a new analysis of analogical arguments in the framework of immanent reasoning\index{immanent reasoning}. By utilising the formalisation of \index{moral conditional}moral conditionals, we show how we can achieve an analysis of analogical reasoning. Because of the particular notion of dependent types\index{dependent types} in CTT, this approach also allows for formalising initial conditions\index{initial conditions} and thereby an explicit notion of \index{analogy!permitted}permitted analogy. The dialogical interpretation takes this one step further as this allows for representing this feature as an individual condition for the particular form of the introduced analogy.
		
		The inclusion of initial conditions\index{initial conditions} is a new feature not known to have been previously introduced in any other contemporary analyses of analogy and the particularity of this project is that it provides further meaning explanation\index{meaning!explanation}s of analogical reasoning that includes an initial permission in a simple and natural way, closely related to actual legal practice. 
        
        \paragraph{Contemporary theories of analogy}
        
        	Problems of analysing analogical reasoning have received much attention in Antiquity and Middle Ages, notably by Aristotle\index{Aristotle}'s analysis of \textit{proportionality\index{proportionality}}. And in recent years, these questions started again to receive attention. In the scientific context, this modern attention to questions about analogy can be traced back to \posscite{hesse1965models} \citetitle{hesse1965models}, where she provided a thorough analysis regarding the use of analogies in different scientific contexts. Somehow distinct from this tradition, another contemporary interest in such questions stemming from legal theory also emerged. 
        	
        	In the context of law, the use of analogies is widespread and theoretical problems regarding their analysis have the last 30 years received attention from researchers across multiple disciplines, such as philosophy, legal theory and computer science. Theorists have from different perspectives tried to analyse the notion of analogy and its use in legal argumentation. Some theorists attempt to unite theories about analogy coming from both the legal and the scientific contexts, while others try to apprehend it only as a legal phenomena. Theorists also differ in what they consider to be the object of their analyses. Some take the perspective of a particular legal framework, while others intend to have a more general scope. 
        
        	Broadly, we might distinguish between theories that explain analogical arguments as regular deductive arguments with some particular premises and theories that explain these arguments as a distinct form of reasoning. We will call the first kind \textit{schema-based} theories and the second kind \textit{inference-based} theories. In the schema-based theories\index{schema-based theory} the goal is to identify a \textit{schema} or \textit{rule} that enables us to consider the analogical argument as a valid deduction. In the inference-based theories\index{inference-based theory}, the goal is to identify analogical reasoning as a particular form of \textit{reasoning} or \textit{inference}. 
        	
        	\textcite{Brewer1996}, \textcite{alchourron1991argumentos} and \textcite{woods2015legal} describe three important schema-based theories\index{schema-based theory}. In \citetitle{Brewer1996}, \citeauthor{Brewer1996} provides a theory of analogical reasoning that became widely influential in the debate about the use of analogies in contemporary Common Law\index{Common Law}, particularly in the Anglo-American tradition. Even earlier, \citeauthor{alchourron1991argumentos} wrote a paper called \citetitle{alchourron1991argumentos} that provided a logical analysis of two kinds of analogical arguments. This paper became very influential in the Spanish-speaking (and to some extent German-speaking) academical debate, though since the paper never has been translated to English, its influence in the English-speaking context has unfortunately been marginal. The book \citetitle{woods2015legal} by \citeauthor{woods2015legal} is a very recent work that attempts to analyse the legal understanding of analogy in light of insights stemming from the theory of science. 
        	
        	\textcite{Bartha2010}, \textcite{Prakken1996} and \textcite{Rahman2017} provide three important inference-based theories\index{inference-based theory}. In \citetitle{Bartha2010}, \citeauthor{Bartha2010} provides a formal model for analogical reasoning in a wide range of areas, from mathematics to everyday reasoning, including legal arguments. It has recently shown to be a very influential general approach to questions of analogy. \citeauthor{Prakken1996} develop in multiple papers a particular dialogical theory that comes from a computer scientific perspective on legal argumentation generally. \citeauthor{Rahman2017} provide in the paper \citetitle{Rahman2017} a contemporary logical analysis of analogical reasoning stemming from the often neglected, but rich tradition of Islamic legal theory\index{Islamic Law}. 
        	
        	The different theories do indeed share many aspects, though there are also important differences, both between the schema-based theories\index{schema-based theory} and the inference-based theories\index{inference-based theory}, and across the individual theories. Some differences can be explained by the point that some theories reduce the notion of analogy to a schema, while others considers them as a particular kind of inference. A noticeable point is however how the different theories deal with multiple\index{analogy!multiple}, competing analogies. In this aspect, the theories varies greatly. Some reject that this should be a part of the formal framework. Some introduce a particular formal concept to create a systemised hierarchy, while some consider that the competing analogies should be decided by the use of an analogical argument in itself. 
        
        \paragraph{The framework of Immanent reasoning\index{immanent reasoning}}
    
    		Constructive type theory was developed by \textcite{martin1984intuitionistic} in order to have a language to reason constructively both about and with mathematics. The idea is to have a system where you do not distinguish between syntax and semantics in the same way as it is traditionally done. This enables us to keep meaning and form on the same level and therefore also interact with each other in a way that is explicit in the language itself. 
        
        	Dialogical logic should not be considered a logical system by itself, but rather a framework where we can interpret different logical systems. It may be considered as a general approach to meaning, and we can therefore use it to develop and compare different logical systems. The idea is to consider meaning as being constituted in the argumentative interaction between two agents. We may trace this idea back to later Wittgenstein\index{Wittgenstein} and his notion of \textit{meaning as use\index{meaning!as use}}. The interpretation of constructive type theory in the dialogical approach is what is called 'immanent reasoning\index{immanent reasoning}', where its most recent version has been developed by \textcite{rahman18}.
    
        	The connection between CTT and dialogical logic seems to be strong. Again, we can refer to Wittgenstein\index{Wittgenstein} and his claim that we should not position ourselves outside language when trying to determine meaning. In CTT we do not consider syntax and semantics to be distinguished in a similar way as in classical logic. For propositions, we have inference rules and not distinct syntactical rules that describe when a proposition can be formed. In the dialogical approach, the meaning is determined by how it is used in interaction. Because of this, we may consider dialogical logic to be a pragmatical approach to meaning and semantics. If we link CTT to the dialogical approach, we do only consider syntax to be a kind of semantics, but that both actually belong to the pragmatics. What we end up with is a system where we do not distinguish syntax, semantics and pragmatics as separate domains. Syntax, semantics and pragmatics are all essentially related to interaction and their meaning should be understood in terms of a normative obligation of interaction between the players, which in turn is a moral notion. In this sense, logic is considered to be a result of the moral obligation of interacting. It is not something essentially fundamental, but rather found in the investigation of ethics. 
    
        	The dialogical approach describes a dialogue between two players. These players may be called the 'Opponent' and the 'Proponent'. The two players argue on a thesis. A thesis is a statement that is subject for the dialogue. The Proponent begins by stating the thesis and the Opponent will try to challenge it. The Proponent will again try to defend this thesis from the Opponents attacks. The notions of challenging and defending is then used to establish the dialogical understanding of meaning.

    	\paragraph{Structure for formalising analogy in immanent reasoning\index{immanent reasoning}}   
    
    		In immanent reasoning\index{immanent reasoning} there is a distinction between formation rule\index{rule!formation}s and particle rule\index{rule!particle}s. In order to represent analogies in this framework, we will use this distinction and consider the formation rule\index{rule!formation}s to represent initial conditions\index{initial conditions} that might be imposed on the analogy and particle rule\index{rule!particle}s to represent a principle coming from Aristotle\index{Aristotle}'s notion of proportionality\index{proportionality}. This proposal of the implementation of analogies in immanent reasoning\index{immanent reasoning} will be an essential part of this project. 
    		
    		In this part we will first analyse analogical reasoning by providing two different kinds of conditions that need to be satisfied. The first kind of condition describes a permission for the utilisation of analogical argumentation in this precise case. In many situations the use of analogies in the legal argumentation process is restricted. This can be for several reasons, like constitutional limitations or because other legal rules already answer this particular legal question. This restriction is not found in other analyses of analogical argumentation and we consider this to be a newly introduced feature, introduced in this project. The second kind of condition is one that we find in most contemporary works regarding the logical foundation of analogical reasoning. This is what will be called the efficiency\index{requirement!efficiency} requirement or the proportionality\index{proportionality}-principle. It will be implemented as ways for the challenger to attack the thesis of the defender by providing a counterexample\index{counterexample}.
    		
    		The methodology used for analysing analogical argumentation in immanent reasoning\index{immanent reasoning} is to describe general aspects relevant for arguments by analogy. This comes partially from the comparison of theories in the first part, but we will also argue for including the new aspect called \textit{permission of the analogy}. This aspect is present in the literature from a legal point of view, but absent in the contemporary logical analyses of analogical reasoning. It is the CTT approach to consider both meaning and form in the same language that enables us to also include this in the analysis in a simple way. 
    
    \section{Structure of book}
    
        The book is structured in three main parts. The first part gives a general overview over the notion of analogy and explains different contemporary theories proposed for analysing analogical reasoning. The second part gives an overview over immanent reasoning\index{immanent reasoning}, which is the logical framework that is used as a basis for the included analysis. Based on the two preceding parts, the third part presents this proposal for formalising analogical reasoning. 
        
        \paragraph{Presentation of present theories}
        
            The first part of the project describes and compares different theories of analogical reasoning in respect to how they handle different kinds of analogies. It starts by a very brief historical introduction related to the concept of analogy and then goes on by describing some relevant terminological distinctions based on contemporary legal theory that will provide useful for the rest of the work. The different theories of analogy in the second chapter will be categorised in schema-based theories\index{schema-based theory} and inference-based theories\index{inference-based theory}. The described schema-based theories\index{schema-based theory} are the ones by \textcite{Brewer1996}, \textcite{alchourron1991argumentos} and \textcite{woods2015legal}. And the described inference-based theories\index{inference-based theory} are the ones by \textcite{Bartha2010}, \textcite{Prakken1996} and \textcite{Rahman2017}. The part then finishes with a comparison in respect to how they represent what will be called horizontal and vertical relation\index{relation!vertical}s and how they handle multiple\index{analogy!multiple} analogies.
            
        \paragraph{Theoretical background}
        	
        The second part provides a thorough description of the theoretical framework of immanent reasoning\index{immanent reasoning}, a very recent interpretation of the constructive type-theoretical framework by \textcite{martin1984intuitionistic}. CTT enables us to describe the interaction between form and  meaning in a way that standard logical frameworks are not able to express. The idea was to develop a language where we can reason constructively in the same time with and about mathematics. CTT is then a powerful language that enables the formulation of hypothetical judgments that are not only dependent on objects, but also on categories by its notion of dependent types\index{dependent types}. Immanent reasoning\index{immanent reasoning} is presented here in its last version, as given in \textcite{rahman18}. This framework gives a sophisticated interpretation of both formal and informal reasoning by the means of a dialogical conception of truth. The idea of the dialogical approach is to consider meaning not relevant to some abstract model, but as argumentative moves in a play. By combining this dialogical approach with CTT we get a framework that is able to express the powerful notion of hypothetical judgments from CTT in a comprehensible way, that corresponds well together with the actual utilisation of argumentative moves. 
        
        \paragraph{A dialogical interpretation of analogy}
        
        	The last part of the book describes the implementation of analogical reasoning as presented in the first part, by the framework of immanent reasoning\index{immanent reasoning} given in the second part. It is this part that constitutes the original development and addition to the contemporary scientific debate particular to this project. This part first provides a description of different particular notions and formalisations in CTT that are essential for the analysis of analogical argumentation. It then goes on by giving a general description of reasoning by analogy, both with characteristics and with heteronomous imperatives\index{heteronomous imperatives}. This analysis provides us with a complex formula for analogical reasoning that will need a particular notational practice to be explained. The project continues by describing a way to translate this into a dialogical explanation. It then provides rules for eight different forms of analogical arguments. The permission of the analogical argument will also be described as more refined in the dialogical conception compared to the general CTT approach, as we would allow the \index{analogy!permitted}permitted analogies to be attached to the relevant particular form of analogical argument. This last part terminates by a discussion of the philosophically relevant aspects of the previously introduced analysis and for the projects choice of a dialogical interpretation of constructive type theory.

	\section{Intention, motivation and goal}

		\paragraph{The role of the logician in a practical context}
		
			When attempting to analyse a concept from a domain as a logician we are given a seemingly conflicting role. On one side we should attempt to base the analysis on the actual practice, so that the analysis reflects the use of this concept. On the other side we are also dealing with a domain that seems to have some normative character. The analysis should not simply be an empirical investigation of how the concept is used, but it should also provide some guidelines for distinguishing good from bad practice of this concept. In this sense such project seems to be in conflict. Should our project only describe the actual practice and risk not being useful because of its lack of opinion regarding the practice? Or, should our project only describe the normative foundations for this concept and thereby risk to end up too far away from the concept we originally wanted to analyse? 
			
			These questions do not seem to be particular to logic, but rather as attached to something more general regarding the practice of doing philosophy. Philosophical concepts should ideally not be too far away from what we normally understand by these concepts, while at the same time they should make us able to explain a correct use. \textit{Truth} is an obvious example of this problem. A proper philosophical definition of truth should in the same time be able to accommodate our intuitions about what truth is and give us guidelines for a correct usage of such concept. This might be reduced to a question about the relationship between normativity and reality, though it is a problem that one would need to overcome in order to provide the kind of analysis that is intended in this projected. Legal theory does not seem to be any exception. 
			
			The role of logic in legal theory seems indeed to be affected by these mentioned problems. Logic is on one hand expected to show the actual practice, while on the other hand also expected to have some effect on the practice. It is not sure that there is one specific way to solve this problem. The best one can do is, as a good legal practitioner, to show discretion. By showing discretion, one can hopefully arrive at an analysis that is neither too far away nor too close to actual practice so that our analysis will be based on reality, without losing the normativity that characterises philosophical and logical concepts. 
		
		\paragraph{Why immanent reasoning\index{immanent reasoning}?}
		
			There are multiple reasons for choosing precisely immanent reasoning\index{immanent reasoning} to represent analogies and a thorough explanation of these reasons is given later in this book. 
			
			The most important reason is the framework's particular way of explaining meaning. In most classical frameworks there is a clear distinction between the syntax and the semantics, or the form and the meaning. When an expression is assigned a meaning, it is assumed that this expressions is well-formed. CTT has a different approach to this. Here, the form of an expression is not simply assumed to be made in a correct way, but the explanation of this form is also included in the object-language by means of formation rule\index{rule!formation}s. Together with the constructive aspect, this enables the framework to provide a clear distinction between the Aristotelean concepts of \textit{meaning}, \textit{actuality\index{actuality}} and \textit{potentiality\index{potentiality}}. CTT is then able to express a very sophisticated notion of conditional that includes dependent types\index{dependent types}, which will show to be essential for the present formalisations of juridical concepts. It is this notion of dependent types\index{dependent types} that enables the implementation of \textit{initial conditions\index{initial conditions}} for analogical reasoning. 
			
			The introduced analysis is based on a formalisation of a special kind of conditional, called \textit{conditional right\index{conditional right}} or \textit{\index{moral conditional}moral conditional}. Leibniz\index{Leibniz, Gottfried W} analysed these conditionals by imposing some particular requirements that would distinguish them from other conditionals. CTT has proven to be a powerful tool for precisely capturing many important aspects of Leibniz\index{Leibniz, Gottfried W}'s analysis. The present project utilises this CTT formalisation of \index{moral conditional}moral conditionals as a foundation for its analysis of analogical reasoning. One result of this project is to show that by embedding the expression of one \index{moral conditional}moral conditional inside another, we can achieve a formalisation of the procedure for analogical reasoning. An argument by analogy might therefore be said to be a special and complex form of \index{moral conditional}moral conditional. 
			
			The third reason for utilising immanent reasoning\index{immanent reasoning} is related to the meaning explanation\index{meaning!explanation}s provided by CTT, and particularly by its dialogical interpretation. By means of the hypothetical judgment, CTT enables us to capture not only corresponding truth conditions as for the classical material conditional, but also the precise \textit{dependency} that the consequent has on its antecedent. This is what makes CTT so expressive regarding the formalisation of juridical and moral claims. We can then show how a decision is dependent on its reason and how the deontic qualification\index{deontic!qualification} is dependent on the performance of the action in very precise ways. Its dialogical interpretation provides a natural and comprehensible framework for meaning explanation\index{meaning!explanation}s that is closely linked to actual legal practice and that unites logical inferences and argumentation theory in one single framework.  

		\paragraph{Academic context}
		
			The project positions itself in a contemporary debate regarding the logical analysis of analogies and analogical reasoning. The goal of the project is twofold. It first intends to present and compare contemporary theories for reasoning by analogy. Second, it intends to provide an independent analysis of such reasoning based on the framework of immanent reasoning\index{immanent reasoning}. This is based on the comparison of the different theories as it attempts to include important aspects introduced in the contemporary debate, while at the same time provide a more refined analysis accounted for by the particular dependency found in analogical reasoning.
			
			The dialogical approach to logic and argumentation can be traced back to antiquity, by the works of Aristotle\index{Aristotle} and Plato\index{Plato}. Logic was then considered to be an activity that was performed as a dialogue regarding some proposition. In the modern approach introduced by Frege, the view on the role of logic changed to become a question about abstract manipulation of formulas. The contemporary dialogical approach was introduced by \textcite{lorenzen1961dialogisches} who brought back the antique idea to again consider logic as dialectical by utilising a game-theoretical approach to meaning. This modern dialogical approach has been further refined and developed into a framework where a great variety of logical systems have been interpreted and compared, creating multiple branches of dialogical logics. One of these branches is what is now called immanent reasoning\index{immanent reasoning}. It is the result of describing an intimate connection between dialogues, constructivism and intuitionism, particularly related to the research done in the University of Lille. The goal of this project is to enter into this tradition and show how the framework can be applied also in the context of analogical reasoning in law. 
			
			The project introduces a new analysis of analogies in immanent reasoning\index{immanent reasoning}. The particularities of this framework enables us to provide a formalisation of analogical reasoning in a precise way in line with corresponding contemporary analyses in other frameworks. Furthermore, this project utilises the concept of dependent types\index{dependent types} to introduce a condition of initial permission for the use of analogies, not introduced in any previously given logical analysis. This condition is indicated by different theorists, particularly from legal scholars, though from the logical perspective this seems to be the first time that this condition is included in the explicit logical representation. It enables the analysis to not only account for the use of analogy, but also to explain the introduction of the analogical argument in the first place. 































