\chapter{Conclusion}

	\section{Project}
	
		The present project is twofold. First, it gives a thorough representation of the concept of analogical reasoning in law by introducing and comparing contemporary theories. Second, it provides an independent analysis of analogical reasoning in the framework of immanent reasoning\index{immanent reasoning}. 
		
		The first part explained and compared six contemporary theories of analogical reasoning in law. These theories were categorised into \textit{schema-based} theories and \textit{inference-based} theories. Schema-based theories of analogical reasoning capture the notion of analogy by a description of a rule or schema. Inference-based theories\index{inference-based theory} on the other hand explain analogical reasoning as a distinct way of reasoning. We identified the theories by \textcite{Brewer1996}, \textcite{alchourron1991argumentos} and \textcite{woods2015legal} to be schema-based and the theories of \textcite{Bartha2010}, \textcite{Prakken1996} and \textcite{Rahman2017} to be inference-based. We then compared the theories by how they handle the notions of \textit{horizontal} and \textit{vertical} relations and by how they analyse \textit{multiple\index{analogy!multiple}, competing} analogies. 
		
		The second part of the project was to provide an independent analysis of reasoning by analogy by utilising the framework of immanent reasoning\index{immanent reasoning}. Immanent reasoning\index{immanent reasoning} was described, together with an informal explanation of the other relevant notions of \textit{case}, \textit{relations} and \textit{initial conditions\index{initial conditions}}. The project then introduced two kinds of analogical reasoning, \textit{general precedent-based reasoning} and \textit{precedent-based reasoning with heteronomous imperatives}. These kinds were first analysed in the general formulation of constructive type theory and then given an alternative formulation in the dialogical interpretation. Following this, we introduced a discussion on the advantages of utilising immanent reasoning\index{immanent reasoning} as a framework for analysing legal reasoning in general and analogical reasoning in particular.
	
	\section{Results}
	
		The first goal is to describe analogical reasoning by introducing and comparing different contemporary theories of analogical reasoning in law. The second goal is to provide an independent analysis of analogies. 
		
		Even though the different contemporary theories had greatly different starting points, they all provided thorough and deep analyses of the concept of analogy. The schema-based theories provided explanations of both the horizontal and the vertical relation\index{relation!vertical}s as either explicit or implicit \textit{formal structures}. The inference-based theories\index{inference-based theory} on the other hand reduced the question of the horizontal relation\index{relation!horizontal}s to be a question of \textit{identity} or \textit{similarity}. The vertical relation\index{relation!vertical}s were then described by a particular form of logical \textit{dependency}. Across the categorisations of schema-based and inference-based theories\index{inference-based theory}, the theories differed in how they handled multiple\index{analogy!multiple}, competing analogies. Two leave this notion \textit{unexplained}, two introduce a particular \textit{higher-order} operator on the different analogies and two consider multiple\index{analogy!multiple} competing analogies as something that should motivate a \textit{change} of the original analogy. 
		
		Based on a variant of the principle of \textit{proportionality\index{proportionality}}, the present project provided an analysis of analogy in the framework of immanent reasoning\index{immanent reasoning}. By utilising the formalisation of \index{moral conditional}moral conditionals where one formulation is embedded in another formulation, we showed how we could represent analogical reasoning. Because of the particular notion of dependent types\index{dependent types} in CTT, this approach also allowed for formalising \textit{initial conditions\index{initial conditions}} by an explicit notion of \textit{permitted analogies\index{analogy!permitted}}. This is a new feature, not previously known to have been introduced in any contemporary analysis of analogy. The dialogical interpretation takes this one step further, as this does allow for representing this feature as an individual condition for the particular \textit{form} of the introduced analogy. This was done by distinguishing in total eight different forms of analogies. The dialogical interpretation also enabled the unification of general precedent-based reasoning and precedent-based reasoning with heteronomous imperatives in a simple way. We have shown that the framework of immanent reasoning\index{immanent reasoning} is a powerful tool to handle analogical reasoning, which also seem capable of analysing inferences in law more generally. 

	\section{Further research}
	
		In the contemporary legal discourse, a notion that is often considered to be closely connected to reasoning by analogy is \textit{balance of interests}. This is an aspect that is often given considerable practical attention when solving a legal issue, a point highlighted by \textcite{armgardt2022formal}. Here, we have not attempted to include interests in the analysis. Though because of its close relationship with interpretation of precedents and analogical reasoning, it would indeed seem to be an aspect worth considering in an extended analysis of legal reasoning.
		
		One of the particularities of the present analysis is its ability to express initial conditions\index{initial conditions} in the formalisation. However, the precise \textit{content} of these conditions stays to a large extent unexplained. A natural continuation of this project would then be to analyse the exact content of the initial conditions\index{initial conditions}. This could then show the effects the no-answer question and the requirement of no-constitutional restraints could potentially have on the analysis. 
		
		Furthermore, the present project briefly described the notion of \textit{precedent}. It identified the conditional structure found in legal cases, though a precise analysis of precedents and cases would seem valuable not only for the understanding of analogies in law, but generally for all kinds of legal reasoning. Precedents and legal cases provide a significant aspect of most contemporary legal systems and a thorough analysis of such notions in CTT could provide a deeper understanding of the logical interactions that take place between reasoning, logic and law. 
	
	
	
	
	
	
	
	
	
	
	