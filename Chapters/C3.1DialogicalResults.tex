
\chapter{CTT analysis}\label{cap-CTTAnalysis}

	We might distinguish two kinds of analogical argumentation or reasoning with precedents. We have reasoning about heteronomous imperatives\index{heteronomous imperatives}, together with the more general reasoning about characteristics. The difference between the two is what kind of result we achieve from the analogical argument. In the case of reasoning about heteronomous imperatives\index{heteronomous imperatives}, the result of the argument should be an imperative, understood as a decision whether the performance of a certain action is law-breaking or law-abiding. In reasoning about characteristics, the result is whether a certain situation has a given property. Typically, this is the situation for reasoning about definitions or borderline rules. In the intuitionistic framework, imperatives seem to be a special kind of predicate. Reasoning about heteronomous imperatives\index{heteronomous imperatives} is therefore considered to be a special kind of reasoning about characteristics. 

	The requirement of efficiency\index{requirement!efficiency} consists of two conditions, the condition of co-extensiveness\index{requirement!co-extensiveness} and the condition of co-exclusiveness\index{requirement!co-exclusiveness}. It is meant to provide restrictions on the choice of the occasioning characteristic\index{characteristic!occasioning}. The occasioning characteristic\index{characteristic!occasioning} should be so that in all cases where we have this occasioning characteristic\index{characteristic!occasioning}, we also have the entailed characteristic\index{characteristic!entailed}, and in all cases where we do not have this occasioning characteristic\index{characteristic!occasioning}, we do not have the entailed characteristic\index{characteristic!entailed} either. Instead of being implemented explicitly, the condition of efficiency\index{requirement!efficiency} is included in the procedure as a whole. This analysis will be not be based on the requirement of efficiency\index{requirement!efficiency} as such, but rather a variant of the \textit{Proportionality\index{proportionality}-principle}, namely:
		%
		\begin{itquote}
			Treat like cases alike and unlike cases differently.	
		\end{itquote}

		
	This is a principle that can be traced back to Aristotle\index{Aristotle}'s \textit{Nicomachean Ethics} (V.3,1131a10-b15) and provides a fundamental notion regarding equality, based on the notion of proportionality\index{proportionality}. It is often considered to be the foundation of analogical reasoning and reasoning with precedents, or sometimes even as a fundamental principle of law. It also provides justification for the principle of \textit{stare decisis}\index{stare decisis text@\textit{stare decisis}}. We can see that this principle actually consists of two parts, what we will call the \textit{Alike-principle\index{principle!alike}} and \textit{Differently-principle\index{principle!differently}}. The first can be described as:
		%
		\[
			\textbf{Alike-principle\index{principle!alike}: } \textit{Treat like cases alike.}
		\] 
		%
	That a case is like another is understood as it being similar to the other. All cases are similar in one way or another and we cannot include all ways cases can be like each other. In order for this principle to be meaningful, we must therefore speak about \textit{relevant similarities} or \textit{similar relevancies}. A similarity is then based upon a relevant characteristic, shared by both cases and this characteristic should be the reason for its precise treatment.
	
	We can easily see how this principle relates to analogical reasoning as an analogy is provided by two things that are similar in some aspect. If we have decided that a case is similar to another one, we should by the principle of equality, treat the first case in the same way as we treated the other. This provides the justification for arguments based upon similarity, namely positive analogies. However, we do also want to include arguments based upon differences, namely negative analogies. For this we would need to rely on the second part of the Proportionality\index{proportionality}-principle, which will be called the \textit{Differently-principle\index{principle!differently}}, 
		%
		\[
			\textbf{Differently-principle\index{principle!differently}: } \textit{Treat unlike cases differently.}
		\]
		%
	At first sight, this is a seemingly more controversial principle than the previous. However, we seem to have good reason also to accept this principle if we accept the first. Seemingly, no cases are the same; all cases are different to each other in some way or another. As with the first principle, we would therefore be inclined to speak about \textit{relevant differences}, or \textit{different relevancies}. That a case $A$ is different from another case $B$ is understood as $A$ not sharing the relevant aspect with $B$ that caused $B$ being treated in the way it was. This means that there was some aspect in $B$ that was the reason for this particular treatment and this aspect was not found in $A$. When we do not have the reason for this particular treatment, the treatment is seemingly groundless and therefore not applicable. That the treatment is not applicable means that we will have to find some other treatment, namely to treat it differently. 
	
	Together, these two principles provide justification for this analysis of analogical reasoning. The two principles seem to reflect the requirement of efficiency\index{requirement!efficiency}, though the Proportionality\index{proportionality}-principle is formulated closer to Aristotle\index{Aristotle}'s original analysis. By distinguishing the Proportionality\index{proportionality}-principle into the Alike-principle\index{principle!alike} and the Differently-principle\index{principle!differently}, we also provide the grounds for distinguishing between positive and negative analogies. This principle will be used to reflect the last step of the analysis performed here, namely the application to the target case.

\section{General precedent-based reasoning}\label{sec:general-precedent-based-reasoning}

	\subsection{Performing analogical reasoning}

	The first step in the analysis will be to describe an informal seven-step procedure for how to perform general analogical reasoning with characteristics. The last step of the procedure might be said to be the most complex and controversial step as it involves the application to the target case. This step is a twofold step where the first part involves what we might call a standard or positive analogy\index{analogy!positive} (based on the Alike-principle\index{principle!alike}), while the second part involves what we might call a negative analogy\index{analogy!negative} (based on the Differently-principle\index{principle!differently}). 
		
		\subsubsection{Procedure for performing analogical reasoning:}
		
		\begin{enumerate}
			\item Include a target case where the presence or absence of some (consequent) characteristic has to be decided.
			\item Find a relevant (occasioning) characteristic that will be chosen for reaching a decision in the target case.
			\item Make sure that the terms are well-defined and that the use of the analogical argument is legally acceptable in this particular situation. 
			\item Decide whether this occasioning characteristic\index{characteristic!occasioning} is present in the target case or not.
			\item Find some source case and decide whether this occasioning characteristic\index{characteristic!occasioning} is present in the source case or not.
			\item Decide whether the entailed characteristic\index{characteristic!entailed} is present in the source case or not.
			\item Decide whether the occasioning characteristic\index{characteristic!occasioning} has the same status in both the target case and the source case (that it is present or absent). This is a twofold step:
				%
				\begin{enumerate}
					\item If the occasioning characteristic\index{characteristic!occasioning} has the same status in both the source case and the target case (that it is present or absent):
						%
						\begin{enumerate}
							\item the status of the entailed characteristic\index{characteristic!entailed} in the source case can be transferred directly to the target case;
						\end{enumerate}
						%
					\item If the occasioning characteristic\index{characteristic!occasioning} has a different status in the source case and the target case (that it was present or absent):
					 %
						\begin{enumerate}
							\item the status that the entailed characteristic\index{characteristic!entailed} has in the source case should not be the situation in the target case (its negation can be transferred to the target case). 
						\end{enumerate}
				\end{enumerate}	
			\end{enumerate}

		\subsubsection{Explaining the procedure}

		The first step is the foundation for the argument in the first place as there has to be some particular case that motivates the introduction of the argument. This is the target case, which essentially can be described as a problem that has to be solved. 
		
		The second step is the most difficult and controversial step when speaking about analogical reasoning. This step corresponds to what \textcite{Brewer1996} refers to as an abductive step, as it is the introduction of the occasioning characteristic\index{characteristic!occasioning}. The choice of relevant occasioning characteristic\index{characteristic!occasioning} seems to depend on some creative or intuitive aspect that cannot be fully described by a procedure. We might use constraints similar to ones developed for abductive inferences, like simplicity, generality, coherence and possibly particular constraints related to legal reasoning, but even with such constraints it seems difficult or impossible to describe an efficient procedure for choosing such characteristic. However, if we have found a potential characteristic, we might reject it if it does not give a coherent result. This might motivate us to go back to change or revise the originally chosen characteristic. This is a result of the efficiency\index{requirement!efficiency} requirement that is implemented in the analysis by means of the Proportionality\index{proportionality}-principle.
		
		The third step is the starting point for the analysis. The occasioning characteristic\index{characteristic!occasioning} chosen in the second step has to be well-defined, which in CTT means to be well-typed. The legal result of the occasioning characteristic\index{characteristic!occasioning} and its negation must be type-declared. We need to declare a set for the accepted source cases. We must also declare that the legal system permits the use of analogical reasoning in this particular situation. This can involve the proposed requirements mentioned earlier, but they can also be different. 
		
		The fourth step is the decision of whether the occasioning characteristic\index{characteristic!occasioning} or its negation is present in the target case. This is the first \textit{investigation} that is represented in the analysis. 
		
		The fifth step refers to the source case. The same investigation has to be performed in the chosen source case, whether the chosen occasioning characteristic\index{characteristic!occasioning} or its negation is present in the source case. 
		
		The sixth step is a second investigation in the source case and creates the foundation for the decision in the target case. This investigation relates to whether the presence of the occasioning characteristic\index{characteristic!occasioning}, or the absence of the occasioning characteristic\index{characteristic!occasioning}, provides the presence or absence of the entailed characteristic\index{characteristic!entailed} in the source case. 
		
		The seventh step is the most complex step as it is the application of a consequence in the target case. It is a twofold step where the first part explains what should happen if the occasioning characteristic\index{characteristic!occasioning} either is present in both the target case and the source case or is absent in both the target case and the source case. In this situation, we might directly transfer the presence/absence of the entailed characteristic\index{characteristic!entailed} from the source case to the target case. This is what might be called a positive analogy\index{analogy!positive}. It is a result of the Alike-principle\index{principle!alike}, to treat like cases alike. The second alternative describes a negative analogy\index{analogy!negative} where the occasioning characteristic\index{characteristic!occasioning} is present in either the target case or in the source case and it is absent in the other. In this situation we might infer that the negation of the presence or absence of the entailed characteristic\index{characteristic!entailed} in the source case, holds in the target case. This is a result of the Differently-principle\index{principle!differently}, to treat unlike cases differently. It is in the interpretation of this step that we notice a difference between intuitionistic and classical logic. Since in classical logic we have the elimination rule of double negation, we might infer that when the entailed characteristic\index{characteristic!entailed} is absent in the source case, it should be \textit{present} in the target case. This is not the case in intuitionistic logic. Intuitionistically, we can only infer that the entailed characteristic\index{characteristic!entailed} should \textit{not be absent}, and this is not the same as to say that it should be present. 
		
		
		\subsubsection{Positive and negative analogies}
		
		Positive analogies occur when the status of the chosen occasioning characteristic\index{characteristic!occasioning} is shared between the source and the target. We can then infer that the entailed characteristic\index{characteristic!entailed} has the same status in the target case as it had in the source case. If the entailed characteristic\index{characteristic!entailed} is present in the source case, we can infer that it should be present in the target case. While if the entailed characteristic\index{characteristic!entailed} is absent in the source case, we can infer that it should be absent also in the target case. 
		
		Negative analogies occur when the status of the chosen occasioning characteristic\index{characteristic!occasioning} is different in the source and the target case. For negative analogies, when the entailed characteristic\index{characteristic!entailed} is present in the source case, we can infer that this characteristic should be absent (its negation should be present) in the target case. Similarly, if the negated entailed characteristic\index{characteristic!entailed} is present in the source case, we can infer that this negated characteristic should be absent (its negated should be present) in the target case. We assume that the absence of a characteristic is the same as the presence of the negated characteristic. That a characteristic is present is denoted as $B(x)$ while that a characteristic that is absent is denoted as $\neg B(x)$. In negative analogies, when we have a presence, $B(x)$, of an entailed characteristic\index{characteristic!entailed} in the source case, we can infer that this characteristic should be absent, $\neg B(x)$, in the target case. A particular situation occurs when we have an absence, $\neg B(x)$, of an entailed characteristic\index{characteristic!entailed} in the source case. We can then infer that this (absent) characteristic should be absent, $\neg \neg B(x)$, in the target case. This is not the same as saying that the entailed characteristic\index{characteristic!entailed} should be present in the target case, as we then end up with a double negated characteristic in the target case. Since we are in an intuitionistic framework, we cannot infer the non-negated characteristic from this. This means that in negative analogies, we end up with a double negated characteristic in the target case when we have a negated characteristic in the source case.
		
		 The meaning of this double negated characteristic might seem slightly unclear at first sight. However, by a closer look this distinction seem to be rather natural in the legal context. Since the framework of immanent reasoning\index{immanent reasoning} is constructive, the notion of truth is connected to its provability. That something is true means that it can be proven. That a negation is true means that the non-negated is not provable (that the attempt of proving is aborted). That a double negation is true means that the negated is not provable. In a legal context, this means that we have a refutation of the negation, but not an explicit proof for a (non-negated) decision. We are then provided with a distinction between reasons in favour of some claim and reasons against rejecting a claim. Negative analogies will generally provide reasons of the last kind. 
		 
		 \subsubsection{Restricted and unrestricted analogies}
		 
		 The informal description describes how to reason when including one source case. This means that the result depends on the situation in a single source case, independently of what are the situations in all other source cases. Usually in legal reasoning, we would like to base the arguments on what generally holds, not only the situation of a particular source case. This can be included by restricting the analysis to include not a single source case, but all source cases.
		 
		 Based on this, we can make a distinction between what we might call \textit{restricted} and \textit{unrestricted} analogical reasoning. The informal analysis of the procedure describes what happens in the unrestricted variant of analogical reasoning. This distinction can then be expressed in the following way:
		 %
		 \begin{description}
		 	\item[Unrestricted analogy:] An unrestricted analogy is an inference based on a single source case that share (or differ based on) a characteristic with the target case;
		 	\item[Restricted analogy:] A restricted analogy is an inference based on all source cases that share (or differ based on) a characteristic with the target case.
		 \end{description}
		 
		 The difference between the restricted and the unrestricted analogical reasoning is whether we require the analogy to hold for all cases or only for a single case, irrespective of all other cases. An unrestricted analogy is based on simply a similarity or difference between the source and the target. Since it does not limit the point of reference for the similarity or the difference, it is also vulnerable to the previously described argument of a potential infinite amount of similarities (or differences). This means that it can be used as an argument for any proposition, and that it therefore hardly can be used as a reference for the use of analogical reasoning in a legal context. 
		 
		 Since the unrestricted analogy only requires one source case, it is considered to be weaker than the restricted form. In ordinary life outside of the context of legal reasoning, it is the form that we often refer to when we speak about analogy. Because of this generality, an unrestricted analogical argument does not seem to provide very strong justification for its result. In short, the unrestricted analogy does not seem to include anything that corresponds to the requirement of efficiency\index{requirement!efficiency}. 
		 
		 A restricted analogy will provide a stronger justification and is also the form that is used in the legal context. Instead of being an analogy over a single source case, it quantifies over all source cases so that all source cases that are similar (or different) to the target case should be coherent in regard to the entailed characteristic\index{characteristic!entailed} of the analogy. However, it is important to note that also in a restricted analogy, the analogy depends on particular source cases and should not necessarily be considered dependent on the generalisation as such. This shows that there seem to be some tension inside the restricted analogy, whether we speak about a particular source case or a generalisation. The framework of immanent reasoning\index{immanent reasoning} seems capable of capturing this in a rather subtle way as it allows the analogical argument to based upon a particular case, though captures the general aspect by the ability to \textit{choose}. One could have chosen differently, but end up with the same result and after the choice is made, the analogy is dependent on the particular case that was chosen. In this way, immanent reasoning\index{immanent reasoning} is able to capture both the particular and the general aspect of restricted analogical reasoning. 
		
		\subsubsection{Imposing conditions on analogical reasoning}\label{ConditionsAnalogicalReasoningCTT}

		A widespread condition for analogical reasoning is the condition of efficiency\index{requirement!efficiency}. There is no explicit implementation of this condition in the process, but the condition of efficiency\index{requirement!efficiency} can be introduced by restricting step 5 to 7 to hold not only for one source case, but for all source cases, so that we speak about a \textit{restricted} analogy. If no decision can be reached based on all source cases, one has to go back to step 2, choose another characteristic and continue the process from there. Other conditions regarding the use of analogical reasoning can be implemented in step 3, as they will be analysed as formation conditions. The logical analysis provided in this work describes and analyses step 3 to step 7. 
		
		In the third step, the conditions regarding the formation of an analogy are implemented. This step could be said to consist of several substeps. The first is the type declaration of the proposition or characteristic. We represent the propositions and characteristics as the standard type $prop$. For some proposition $A$, we would need to suppose:
			%
			\[
				A : prop.
			\]
		%
		The source cases have to be declared as a set. It means that we have some defined and accepted source cases that might be used in the analogy.\footnote{In the CTT representation, we leave out the explicit representation of the target case as a set. This is to avoid further complexity and keep the analysis as simple as possible. In the dialogical analysis, complexity is however less of a problem and we will introduce the explicit set for the target case in this representation. See \autoref{Sec:DialogicalImplementation} for details about the implementation of the target case set.} This can be represented in the following way:
		%
			\[   
				Source : set.
			\]
			%
		In addition, we have to declare that the intended entailed characteristic\index{characteristic!entailed} is a proposition. This proposition is also dependent on both the proposition $A$ and on the set $Source$, introduced in the following way:
		%
			\[
				B(x,s) : prop (x : A, s : Source).
			\]
			%
		The last part is the inclusion of the permission in the legal system of utilising analogical reasoning in this particular situation. The permission of an analogy is rooted in its result in the target case, not in the proposition itself. This means that we will introduce the permission of the analogy on the presence or absence of $B$. We therefore have four sets of \index{analogy!permitted}permitted analogies, represented in the following way:
			%
			\[
			\begin{array}{l}
				PA_1(z_1) : set (z_1 : (x_1 : A) B(x_1) \lor (x_2 : A) \neg B(x_2)); \\
				PA_2(z_2) : set (z_2 : (y_1 : \neg A) \neg B(y_1) \lor (y_2 : \neg A) \neg \neg B(y_2)); \\
				PA_3(z_3) : set (z_3 : (x_3 : A) \neg B(x_3) \lor (x_4 : A) \neg \neg B(x_4)); \\
				PA_4(z_4) : set (z_4 : (y_3 : \neg A) B(y_3) \lor (y_4 : \neg A) \neg B(y_4)). 
			\end{array}
			\]
		The explicit permission of an analogy is introduced by the previously described judgments. These judgments can be produced by an instance of the elimination rule of the disjoint union. The instance for the first judgment is the following:
			%
			\[
			\infer[PA]{D(c, (x)d(x), (y)e(y)) : PA(c),}{\begin{array}{ccc} & (x : B) & (y : \neg B) \\
			c : B \lor \neg B & d(x) : PA(i(x)) & e(y) : PA(j(y)) \end{array}}
			\]
		which reads that when we have an $a$ that is a \index{analogy!permitted}permitted analogy, it is justified by the proof object $D(c, (x)d(x), (y)e(y))$ that is produced when $c$ is the proof object of the disjunction and $(x)d(x)$ is verified in the case of $B$, while $(y)e(y)$ is verified in the case of $\neg B$. 
		
		It is important to note that this does not explain the content of this requirement, for example that it should be a lacuna\index{lacuna text@\textit{lacuna}} in the law and not undermine constitutional values, but is rather a representation of the result assessing this content. 
		
		For general analogies, we then end up with the following context:
		%
			\[
			\begin{array}{c}
				A : prop,                                                                                \\
				Source : set,                                                                            \\
				B(x,s) : prop (x : A, s : Source),														\\
				PA_1(z_1) : set (z_1 : (x_1 : A) B(x_1) \lor (x_2 : A) \neg B(x_2)),                     \\
				PA_2(z_2) : set (z_2 : (y_1 : \neg A) \neg B(y_1) \lor (y_2 : \neg A) \neg \neg B(y_2)), \\
				PA_3(z_3) : set (z_3 : (x_3 : A) \neg B(x_3) \lor (x_4 : A) \neg \neg B(x_4)),           \\
				PA_4(z_4) : set (z_4 : (y_3 : \neg A) B(y_3) \lor (y_4 : \neg A) \neg B(y_4)).
			\end{array}
			\]
			%
		In the following sections, the explicit formulation of this context will be left out for the sake of simplicity.
		
	\subsection{Representing source cases}
				 
		We can now describe how to reach a decision based on the available source cases at hand. Here, we suppose that the absence of a characteristic is the presence of its negation. That $A$ is being absent will therefore be described as $\neg A$ being present. For a source case $s$, we then have several situations: 
		%
			\begin{enumerate}
				\item $A$ is present in $s$,
					\begin{enumerate}
						\item $B$ is present in $s$,
						\item $\neg B$ is present in $s$;
					\end{enumerate}
				\item  $\neg A$ is present in $s$,
					\begin{enumerate}
						\item $B$ is present in $s$,
						\item $\neg B$ is present in $s$.
					\end{enumerate}
			\end{enumerate}		
		
		Depending on the source case $s$, we can use $s$ as an argument for a certain standpoint in the target case. The standard form of analogical reasoning is based on $A$ being present in both the source case and the target case and since $B$ was present/absent in the source case, it should also be present/absent in the target case. If we assume that $A$ is present in the target case, this can be represented in the following way, so that when there are two lines from a statement, it represents an implication of a conjunction: \medskip
			
			\noindent\begin{minipage}{0.9\textwidth}
			\dirtree{%
			.1 For a source case $s$.
				.2 $A$ is present in $s$.
					.3 $B$ is present in $s$.
						.4 $B$ should be present in the target case.
					.3 $\neg B$ is present in $s$.
						.4 $\neg B$ should be present in the target case..
			}	
			\end{minipage}	\medskip	
			
		In addition to the standard form of analogical arguments, we also have what we called negative analogies. Instead of being based on the similarity between the target case and the source case, it depends on their difference. If $A$ is present in the target case and not in the source case, and $B$ is present in the source case, $B$ should be absent in the target case. And if $B$ is absent in the source case, $B$ should not be absent in the target case. We can represent this in the following way, where we still suppose that $A$ is present in the target case: \medskip
			
			\noindent\begin{minipage}{0.9\textwidth}
			\dirtree{%
			.1 For a source case $s$.
				.2 $\neg A$ is present in $s$.
					.3 $B$ is present in $s$.
						.4 $\neg B$ should be present in the target case.
					.3 $\neg B$ is present in $s$.
						.4 $\neg \neg B$ should be present in the target case..
			}		
			\end{minipage}\medskip

		
		By combining these two representations, we end up with a procedure for handling analogical reasoning when $A$ is present in the target case. This yields the following form:\medskip
			
			\noindent\begin{minipage}{0.9\textwidth}
			\dirtree{%
			.1 For a source case $s$.
				.2 $A$ is present in $s$.
					.3 $B$ is present in $s$.
						.4 $B$ should be present in the target case.
					.3 $\neg B$ is present in $s$.
						.4 $\neg B$ should be present in the target case.
				.2 $\neg A$ is present in $s$.
					.3 $B$ is present in $s$.
						.4 $\neg B$ should be present in the target case. 
					.3 $\neg B$ is present in $s$.
						.4 $\neg \neg B$ should be present in the target case..
			}		
			\end{minipage}\medskip	
		
		Correspondingly, we can describe the process of analogical reasoning when $\neg A$ is present in the target case. If $\neg A$ is also present in the source case, we speak about a positive analogy\index{analogy!positive} since $\neg A$ is shared between the source and the target. The decision whether $B$ is present or absent can be directly transferred from the source case to the target case. We suppose here that $\neg A$ is present in the target case. This transfer can be represented in the following way: \medskip
			
			\noindent\begin{minipage}{0.9\textwidth}
			\dirtree{%
			.1 For a source case $s'$.
				.2 $\neg A$ is present in $s'$.
					.3 $B$ is present in $s'$.
						.4 $B$ should be present in the target case.
					.3 $\neg B$ is present in $s'$.
						.4 $\neg B$ should be present in the target case..
			}
			\end{minipage}	\medskip			
		
		When $\neg A$ is present in the target case, we might also speak about negative analogies. The negative analogy\index{analogy!negative} occurs when $A$ is present in the source case. The analogy is then based on some characteristic that is not shared between the source and the target. This can be represented in the following way, where it is supposed that $\neg A$ is present in the target case: \medskip
			
			\noindent\begin{minipage}{0.9\textwidth}
			\dirtree{%
			.1 For a source case $s'$.
				.2 $A$ is present in $s'$.
					.3 $B$ is present in $s'$.
						.4 $\neg B$ should be present in the target case.
					.3 $\neg B$ is present in $s'$.
						.4 $\neg \neg B$ should be present in the target case..
			}		
			\end{minipage}\medskip
		
		We then end up with a procedure for handling analogical reasoning when $\neg A$ is present in the target case. This yields the following form: \medskip
			
			\noindent\begin{minipage}{0.9\textwidth}
			\dirtree{%
			.1 For a source case $s'$.
				.2 $A$ is present in $s'$.
					.3 $B$ is present in $s'$.
						.4 $\neg B$ should be present in the target case.
					.3 $\neg B$ is present in $s'$.
						.4 $\neg \neg B$ should be present in the target case.
				.2 $\neg A$ is present in $s'$.
					.3 $B$ is present in $s'$.
						.4 $B$ should be present in the target case.
					.3 $\neg B$ is present in $s'$.
						.4 $\neg B$ should be present in the target case..
			}		
			\end{minipage}\medskip				
		
	\subsection{Representing the analogical procedure in CTT}
		
		By combining the procedures for handling analogies when $A$ is present and when $\neg A$ is present in the target case, we end up with a description of the whole process for analogical reasoning. This can be represented by the following: \medskip
		
			\noindent\begin{minipage}{0.9\textwidth} %fix???
			\dirtree{%
			.1 For a target case where $A \lor \neg A$ is present.
				.2 $A$ is present in the target case.
					.3 For a source case $s$.
						.4 $A$ is present in $s$.
							.5 $B$ is present in $s$.
								.6 $B$ should be present in the target case.
							.5 $\neg B$ is present in $s$.
								.6 $\neg B$ should be present in the target case.
						.4 $\neg A$ is present in $s$.
							.5 $B$ is present in $s$.
								.6 $\neg B$ should be present in the target case. 
							.5 $\neg B$ was present in $s$.
								.6 $\neg \neg B$ should be present in the target case.
				.2 $\neg A$ is present in the target case.
					.3 For a source case $s'$.
						.4 $A$ is present in $s'$.
							.5 $B$ is present in $s'$.
								.6 $\neg B$ should be present in the target case.
							.5 $\neg B$ is present in $s'$.
								.6 $\neg \neg B$ should be present in the target case.
						.4 $\neg A$ is present in $s'$.
							.5 $B$ is present in $s'$.
								.6 $B$ should be present in the target case.
							.5 $\neg B$ is present in $s'$.
								.6 $\neg B$ should be present in the target case..	
			}	
			\end{minipage}
		

		This provides the foundation for its representation in CTT. The target case can be represented by a similar form as the conditional analysis described in \textcite{Rahman2019}, though including an explicit dependency on the source case. In a similar way as previously described, we will use $\{H1\},\{H2\}, \hdots$ as abbreviations for formulas like $A \lor \neg A$ in identity statements. We then use a conditional formulation to represent the inquiry of whether it is the right or the left side of this disjunction that makes it true in the target case. If it is the left side $A$, we continue with the source case $s$. If it is the right side $\neg A$, we continue with the source case $s'$. This yields the following, incomplete formula:
			%
				\begin{multline*}
					b(x) : [(\forall y : A) \textbf{left}^\lor(y) =_{\{H1\}} x \supset (\forall s : Source) ...] \land \\
					[(\forall y' : \neg A) \textbf{right}^\lor(y') =_{\{H1\}} x \supset (\forall s' : Source) ...] (x : A \lor \neg A).
				\end{multline*}
			%
			In the tree structure, this formula receives the following notation: \newline \medskip

				\noindent\begin{minipage}{0.9\textwidth}
				\dirtree{%
					.1 $(x : A \lor \neg A)$ \\ $b(x) : $.
						.2 $(\forall y : A)$.
							.3 $\textbf{left}^\lor(y) =_{\{H1\}} x$.
								.4 $(\forall s : Source)$. 
									.5 $...$.
						.2 $(\forall y' : \neg A)$. 
							.3 $\textbf{right}^\lor(y') =_{\{H1\}} x $.
								.4 $(\forall s' : Source)$.
									.5 $...$..
				}
				\end{minipage}\medskip

		This formula is incomplete because it does not include the description of the source cases. They should be included in '$...$'. By understanding the target case in this way, we include a notion of \textit{suspense}. Based on the procedure for performing analogical reasoning as described earlier, this representation will include a suspense on the decision of whether $A$ or $\neg A$ is present in the target case. Essentially, this means that the presence or absence of $A$ is not given by the target case, but is performed as an investigation into the target case when referring to an analogy. We might therefore say that the investigation or inquiry whether $A$ or $\neg A$ is present in the target case comes after the target case itself. The source cases are implemented inside the formula after it has been decided whether $A$ or $\neg A$ is present in the target case. The source cases are only introduced when trying to build an analogical argument after deciding the presence or absence of some proposition in the target case. The analysis therefore does not introduce the source cases before they are needed. 
		
		
		There are two '$...$' in the incomplete formula. The source cases are introduced in two parts, first when $A$ is present in the target case and second when $\neg A$ is present in the target case. If $A$ is present in the target case, the process can be formalised in the following way: \medskip
				
				\noindent\begin{minipage}{0.9\textwidth}
				\dirtree{%
					.1 $(\forall s : Source)$. 
						.2 $(\forall z : A(s) \lor \neg A(s))$.
							.3 $(\forall u_1 : A(s))$.
								.4 $\textbf{left}^\lor (u_1) =_{\{H2\}} z $.
									.5 $(\forall v_1 : B(s) \lor \neg B(s))$.
										.6 $(\forall w_1 : B(s))$.
											.7 $\textbf{left}^\lor (w_1) =_{\{H3\}} v_1 $.
												.8 $B$ should be present in the target case.
										.6 $(\forall w_2 : \neg B(s))$.
											.7 $\textbf{right}^\lor (w_1) =_{\{H3\}} v_1 $.
												.8 $\neg B$ should be present in the target case.
							.3 $(\forall u_2 : \neg A(s))$.
								.4 $\textbf{right}^\lor (u_2) =_{\{H2\}} z$.
									.5 $(\forall v_2 : B(s) \lor \neg B(s))$.
										.6 $(\forall w_3 : B(s))$.
											.7 $\textbf{left}^\lor (w_3) =_{\{H4\}} v_2 $.
												.8 $\neg B$ should be present in the target case.
										.6 $(\forall w_4 : \neg B(s))$.
											.7 $\textbf{right}^\lor (w_4) =_{\{H4\}} v_2 $.
												.8 $\neg \neg B$ should be present in the target case..
				}
				\end{minipage}\medskip	
		
		Similarly, if $\neg A$ is present in the target case, the process can be formalised in the following way: \medskip
				
				\noindent\begin{minipage}{0.9\textwidth}
				\dirtree{%
					.1 $(\forall s' : Source)$. 
						.2 $(\forall z' : A(s') \lor \neg A(s'))$.
							.3 $(\forall u'_1 : A(s'))$.
								.4 $\textbf{left}^\lor (u'_1) =_{\{H2'\}} z' $.
									.5 $(\forall v'_1 : B(s') \lor \neg B(s'))$.
										.6 $(\forall w'_1 : B(s'))$.
											.7 $\textbf{left}^\lor (w'_1) =_{\{H3\}} v'_1 $.
												.8 $\neg B$ should be present in the target case.
										.6 $(\forall w'_2 : \neg B(s'))$.
											.7 $\textbf{right}^\lor (w'_1) =_{\{H3'\}} v'_1 $.
												.8 $\neg \neg B$ should be present in the target case.
							.3 $(\forall u'_2 : \neg A(s'))$.
								.4 $\textbf{right}^\lor (u'_2) =_{\{H2'\}} z'$.
									.5 $(\forall v'_2 : B(s') \lor \neg B(s'))$.
										.6 $(\forall w'_3 : B(s'))$.
											.7 $\textbf{left}^\lor (w'_3) =_{\{H4\}} v'_2 $.
												.8 $B$ should be present in the target case.
										.6 $(\forall w'_4 : \neg B(s'))$.
											.7 $\textbf{right}^\lor (w'_4) =_{\{H4\}} v'_2 $.
												.8 $\neg B$ should be present in the target case..
				}
				\end{minipage}\medskip	
		
		By combining both of these formulations, we end up with a CTT analysis of the process of analogical reasoning. The general formulation for analogical reasoning in CTT gives the following: 
				
				\scalebox{0.74}{
				\noindent\begin{minipage}{\textwidth}
				\vspace{\baselineskip}
				\dirtree{%
					.1 $(x : A \lor \neg A)$ \\ $b(x) : $.
						.2 $(\forall y : A)$.
							.3 $\textbf{left}^\lor(y) =_{\{H1\}} x$.
								.4 $(\forall s : Source)$. 
									.5 $(\forall z : A(s) \lor \neg A(s))$.
										.6 $(\forall u_1 : A(s))$.
											.7 $\textbf{left}^\lor (u_1) =_{\{H2\}} z $.
												.8 $(\forall v_1 : B(u_1) \lor \neg B(u_1))$.
													.9 $(\forall w_1 : B(u_1))$.
														.10 $\textbf{left}^\lor (w_1) =_{\{H3\}} v_1 $.
															.11 $B(y)$.
													.9 $(\forall w_2 : \neg B(u_1))$.
														.10 $\textbf{right}^\lor (w_2) =_{\{H3\}} v_1 $.
															.11 $\neg B(y)$.
										.6 $(\forall u_2 : \neg A(s))$.
											.7 $\textbf{right}^\lor (u_2) =_{\{H2\}} z$.
												.8 $(\forall v_2 : B(u_2) \lor \neg B(u_2))$.
													.9 $(\forall w_3 : B(u_2))$.
														.10 $\textbf{left}^\lor (w_3) =_{\{H4\}} v_2 $.
															.11 $\neg B(y)$.
													.9 $(\forall w_4 : \neg B(u_2))$.
														.10 $\textbf{right}^\lor (w_4) =_{\{H4\}} v_2 $.
															.11 $\neg \neg B(y)$.
						.2 $(\forall y' : \neg A)$. 
							.3 $\textbf{right}^\lor(y') =_{\{H1\}} x $.
								.4 $(\forall s' : Source)$. 
									.5 $(\forall z' : A(s') \lor \neg A(s'))$.
										.6 $(\forall u'_1 : A(s'))$.
											.7 $\textbf{left}^\lor (u'_1) =_{\{H2'\}} z' $.
												.8 $(\forall v'_1 : B(u'_1) \lor \neg B(u'_1))$.
													.9 $(\forall w'_1 : B(u'_1))$.
														.10 $\textbf{left}^\lor (w'_1) =_{\{H3\}} v'_1 $.
															.11 $\neg B(y')$.
													.9 $(\forall w'_2 : \neg B(u'_1))$.
														.10 $\textbf{right}^\lor (w'_2) =_{\{H3'\}} v'_1 $.
															.11 $\neg \neg B(y')$.
										.6 $(\forall u'_2 : \neg A(s'))$.
											.7 $\textbf{right}^\lor (u'_2) =_{\{H2'\}} z'$.
												.8 $(\forall v'_2 : B(u'_2) \lor \neg B(u'_2))$.
													.9 $(\forall w'_3 : B(u'_2))$.
														.10 $\textbf{left}^\lor (w'_3) =_{\{H4\}} v'_2 $.
															.11 $B(y')$.
													.9 $(\forall w'_4 : \neg B(u'_2))$.
														.10 $\textbf{right}^\lor (w'_4) =_{\{H4'\}} v'_2 $.
															.11 $\neg B(y')$..
				}
				\end{minipage}
				}
							
				
		This procedure is the CTT representation of step 4 to 7 in the description of the process of analogical reasoning. By including the context that was described earlier, we would also include step 3. When applying this formula to some particular case, every disjunction will play the role as a question, inquiry or investigation. After step 3, the first investigation that occurs is formalised as $x : A \lor \neg A$ and represents the decision whether the chosen characteristic is present or absent in the target case. After it has been decided whether $A$ is present or absent in the target case, the source cases are introduced and for every source case the investigation whether the characteristic is present or absent in this source case is formalised $(\forall z : A(s) \lor \neg A(s))$ and $(\forall z' : A(s') \lor \neg A(s'))$. After deciding whether $A$ is present or absent in the source case, the next investigation is whether the entailed characteristic\index{characteristic!entailed} is present or absent in the source case, which is represented as $(\forall v_1 : B (u_1) \lor \neg B (u_1))$, $(\forall v_2 : B (u_2) \lor \neg B (u_2))$, $(\forall v_1' : B (u_1') \lor \neg B (u_1'))$ and $(\forall v_2' : B (u_2') \lor \neg B (u_2'))$. If the occasioning characteristic\index{characteristic!occasioning} is present in both the target case and the source case, the presence (or absence) of the entailed characteristic\index{characteristic!entailed} in the source case can be directly transferred to the target case. This is the situation for $w_1$, $w_2$, $w'_3$ and $w'_4$. This transfer is formalised by binding the entailed characteristic\index{characteristic!entailed} to the choice that was performed in the target case, $y$ or $y'$. This corresponds to what we have called a positive analogy\index{analogy!positive}. 
		
		The other alternative is that the characteristic is present in either the target case or the source case and not present in the other. This is the foundation for what we have called negative analogies. Negative analogies are however slightly more complicate than positive analogies. The negative analogies occur in $w_3$, $w_4$, $w'_1$ and $w'_2$. In $w_3$, we have $\neg A$ and $B$ in the source case, so since we also have $A$ in the target case, we can infer $\neg B$ in the target case. In $w_4$, we have $\neg A$ and $\neg B$ in the source case, so since we also have $A$ in the target case, we can infer $\neg \neg B$ in the target case. In $w'_1$, we have $A$ and $B$ in the source case, so since we also have $\neg A$ in the target case, we can infer $\neg B$ in the target case. In $w'_2$, we have $A$ and $\neg B$ in the source case, so since we also have $A$ in the target case, we can infer $\neg \neg B$ in the target case. 
		

		We notice that since we operate within an intuitionistic framework, the notion $\neg \neg B$ is not equivalent or reducible to $B$. We do however claim that this distinction is important as it highlights the kind of evidence provided by such negative analogies. A source case, where the point of reference is not shared with the target case cannot be used to provide evidence for applying a notion not present in the source case to the target case. It seems all in all to correspond well together with legal practice regarding evidence as it highlights the difference of being evidence for a certain claim and being evidence against its negation.




\section{Precedent-based reasoning with imperatives}\label{sec:precedent-based-reasoning-with-heteronomous-imperatives}

	\subsection{Performing analogical reasoning}
		
		\subsubsection{Procedure for performing analogical reasoning}
		
		A more specific variant of analogical reasoning is reasoning about heteronomous imperatives\index{heteronomous imperatives}. What is understood by imperatives is discussed in \autoref{sec:dialogical-implementation-of-heteronomous-imperatives}. When performing reasoning with precedents about imperatives we seem to be entitled to some further inferences compared to the situation with characteristics. In this representation, we will base it on the description on the three deontic categories\index{deontic!category} described earlier. If we had included all five categories by Ibn \d{H}azm\index{Ibn \d{H}azm}, we would indeed arrive at a different process as there would be a distinction between an action being law-abiding and an action being legally neutral.\footnote{For an analysis in the same system based on these five categories, see \textcite{Kvernenes2020}.} Note also that we here speak about \textit{actions} rather than characteristics, as it is a condition for an imperative that the proposition describes an action, not a characteristic or property.\footnote{We could implement a new type of proposition specific for action-descriptions, though for the purposes in this analysis an explanation by the standard type $prop$ seems sufficient.} The descriptions of \textit{positive}/\textit{negative} and \textit{unrestricted}/\textit{restricted} analogies in \autoref{sec:general-precedent-based-reasoning} also hold here. 
		
		We will use the notions 'law-abiding' and 'law-breaking' instead of 'reward\index{reward}' and 'sanction\index{sanction}', as they seem to be more general in respect to different points of view on legal theory. Analogical reasoning based on the law-abidingness or law-breakingness of an action can be described as the following seven-step process:
		%
		\begin{enumerate}
			\item Include a target case where some decision regarding the deontic status\index{deontic!status} of an action has to be reached.
			\item Find a relevant(occasioning) action-proposition that will be chosen for reaching a decision in the target case.
			\item Make sure that the terms are well-defined and that the use of analogical reasoning is legally acceptable in this particular situation. 
			\item Decide whether this action, as it is described by the action-proposition, was performed in the target case or not.
			\item Find some source case and decide whether this action was performed in the source case or not.
			\item Decide whether the performance, or non-performance, was judged law-abiding or law-breaking in the source case.
			\item Decide whether the action has the same status in both the target case and the source case (that it was performed/not performed). This is a twofold step:
				%
				\begin{enumerate}
					\item If the action has the same status in the source case and the target case (that it was performed/not performed):
						%
						\begin{enumerate}
							\item the decision of the performance or non-performance of the action from the source case can be transferred directly to the performance or non-performance of the action in the target case;
						\end{enumerate}
						%
					\item If the action has a different status in the source case and the target case (that it was performed/not performed):
					 %
						\begin{enumerate}
							\item the performance/non-performance was law-abiding in the source case, and the opposite (that is the situation in the target case) should be law-abiding or law-breaking in the target case and;
							\item the performance/non-performance was law-breaking in the source case, and the opposite (that is the situation in the target case) should be law-abiding in the target case.
						\end{enumerate}
						
				\end{enumerate}	
				
		\end{enumerate}

		\subsubsection{Explaining the procedure}

		Step one and two correspond to the procedure for general analogies. 
		
		In the third step, the action-proposition chosen in the second step has to be well-defined, which in CTT means to be well-typed. The law-abidingness and law-breakingness of the chosen action-proposition and its negation must be type declared. We need to declare a set for the accepted source cases. We must also assure that the legal system permits the use of analogical reasoning in this particular situation. 
		
		The fourth step is the decision of whether the chosen action-proposition or its negation can be used to describe the action that was performed in the target case. For simplicity, we will speak about action-propositions as being present or absent. 
		
		The fifth step refers to the source case. The same investigation has to be performed in the chosen source case, whether the chosen action-proposition or its negation can be used to describe the action in the source case. 
		
		The sixth step is a second investigation in the source case and creates the foundation for the decision in the target case. This investigation is whether the performance of this action (or the absence of the performance) was judged law-abiding or law-breaking in the source case. 
		
		The seventh step is a twofold step where the first part explains what should happen if the action-proposition either is present in both the target case and the source case or is absent in both the target case and the source case. In this situation, we might directly transfer the consequence from the source case to the target case. The second alternative is where the action-proposition is present in either the target case or the source case and it is absent in the other. In this situation, there are two alternatives. In the source case, the presence or absence of the action-proposition could be law-abiding or it could be law-breaking. If it was law-abiding, the opposite, what is the situation in the target case, can be either law-abiding or law-breaking in the target case. This can be explained by the deontic categories\index{deontic!category} previously mentioned. If it was law-breaking, we can claim that the opposite should law-abiding in the target case. 
		

		\subsubsection{Imposing conditions on analogical reasoning}
		
		In the third step, the conditions regarding formation of an analogy is implemented. This step could be said to consist of several substeps. The first is the type-declaration of the action-proposition. For some action $A$, we would need to suppose:
			%
			\[
				A : prop.
			\]
		
		The source cases need to be declared a set. It means that we have some defined and accepted source cases that might be used in the analogy. This can be represented in the following way:
			%
			\[   
				Source : set.
			\]
		
		The permission of an analogy is rooted in its result, not in the action itself. This means that we will introduce the permission on the analogy on the law-abidingness or law-breakingness of the action. We will see that this closely relates to the deontic categories\index{deontic!category} that are introduced. More precisely, we might say that the permission of using an analogical argument depends on the deontic categories\index{deontic!category} that we include in our framework, not on the particular legal consequence.\footnote{If we want to include deontic categories\index{deontic!category} with a possibility of legal reward\index{reward}, this could be done by introducing additional judgments here. However, such change will also have other effects on the procedure.} We therefore have a set of \index{analogy!permitted}permitted analogies, which might be represented in the following way:
			%
			\[
			\begin{array}{l}
				PA_1(z_1) : set (z_1 : LA_A \lor LB_{\neg A}); \\
				PA_2(z_2) : set (z_2 : LB_A \lor LA_{\neg A}); \\
				PA_3(z_3) : set (z_3 : LA_A \lor LA_{\neg A}), \\
			\end{array}
			\]
			%
		where $LB$ and $LA$ are abbreviations for the following:
			%
			\[
			\begin{array}{l | l}
				LB_A        & (v_1 : A) LB(v_1);      \\
				LB_{\neg A} & (w_1 : \neg A) LB(w_2); \\
				LA_A        & (v_1 : A) LA(v_1);      \\
				LA_{\neg A} & (w_2 : \neg A) LA(w_2).
			\end{array}
			\]
			
		The permission of an analogy can be produced by an instance of the elimination rule of the disjoint union. The instance that provides the permission of analogies for the first permission is
		%
			\[
			\infer[PA]{D(c, (x)d(x), (y)e(y)) : PA(c),}{\begin{array}{ccc} & (x : LA) & (y : LB) \\
			c : LA \lor LB & d(x) : PA(i(x)) & e(y) : PA(j(y)) \end{array}}
			\]
		which reads that when we have an $a$ that is a \index{analogy!permitted}permitted analogy, it is justified by the proof object $D(c, (x)d(x), (y)e(y))$ that is produced when $c$ is the proof object of the disjunction and $(x)d(x)$ is verified in the case of law-abidingness, while $(y)e(y)$ is verified in the case of law-breakingness.\footnote{We here implicitly assume that law-abidingness and law-breakingness are defined as propositions over the chosen action-proposition $A$.}  
		
		To represent analogical reasoning based on the deontic imperatives\index{deontic!imperative} by choosing $A$ as action-proposition, we end up with the following context:
		%
			\[
			\begin{array}{c}
				A : prop,                                      \\
				Source : set,                                  \\
				PA_1(z_1) : set (z_1 : LA_A \lor LB_{\neg A}), \\
				PA_2(z_2) : set (z_2 : LB_A \lor LA_{\neg A}), \\
				PA_3(z_3) : set (z_3 : LA_A \lor LA_{\neg A}).
			\end{array}
			\]	

	\subsection{Representing source cases}
		
		For a source case $s$, we have several potential situations:
			%
			\begin{samepage}
			\begin{enumerate}
				\item $A$ is present in $s$,
					\begin{enumerate}
						\item $A$ was law-abiding in $s$;
						\item $A$ was law-breaking in $s$;
					\end{enumerate}
				\item  $\neg A$ is present in $s$,
					\begin{enumerate}
						\item $\neg A$ was law-abiding in $s$;
						\item $\neg A$ was law-breaking in $s$.
					\end{enumerate}
			\end{enumerate}	
			\end{samepage}	

		Depending on what is the situation for the source case $s$, we can use it as an argument for a certain standpoint in the target case. The standard form of analogical reasoning is based on that $A$ is present in both the source case and the target case and since $A$ was law-abiding/law-breaking in the source case, it should also be law-abiding/law-breaking in the target case. Assuming that $A$ is present in the target case, this can be represented in the following way: \medskip
			
			\noindent\begin{minipage}{0.9\textwidth}
			\dirtree{%
			.1 For a source case $s$.
				.2 $A$ is present in $s$.
					.3 $A$ was law-abiding in $s$.
						.4 $A$ should be law-abiding in the target case.
					.3 $A$ was law-breaking in $s$.
						.4 $A$ should be law-breaking in the target case..
			}	
			\end{minipage} \medskip	
			
		We also have negative analogies. If the action $A$ that is present in the target case is not shared by the source case so that $\neg A$ is law-abiding in the source case, $A$ should be law-abiding or law-breaking in the target case. This means that we cannot decide on this basis whether $A$ should be law-abiding or law-breaking. If $A$ that is present in the target case and $\neg A$ is law-breaking in the source case, $A$ should be law-abiding in the target case. Similarly, we can represent this in the following way, where we still suppose that $A$ is present in the target case:\medskip
			
			\noindent\begin{minipage}{0.9\textwidth}
			\dirtree{%
			.1 For a source case $s$.
				.2 $\neg A$ is present in $s$.
					.3 $\neg A$ was law-abiding in $s$.
						.4 $A$ should be law-abiding or law-breaking in the target case.
					.3 $\neg A$ was law-breaking in $s$.
						.4 $A$ should be law-abiding in the target case..
			}		
			\end{minipage}\medskip

		This can be explained by referring to the deontic categories\index{deontic!category}. If $\neg A$ has been law-abiding, the deontic status\index{deontic!status} of $A$ can be either forbidden\index{action!forbidden} or permissible\index{action!permissible}. The performance of $A$ can be either law-breaking if it is a forbidden\index{action!forbidden} action or law-abiding if it is a permissible\index{action!permissible} action. If on the other hand $\neg A$ is law-breaking, the deontic status\index{deontic!status} of $A$ can only be that of an obligatory\index{action!obligatory} action. There are no other deontic categories\index{deontic!category} where $\neg A$ is law-breaking. Since in obligatory\index{action!obligatory} actions, $A$ is law-abiding we can include such inference in the representation. 
		
		By combining these two representations, we end up with a procedure for handling analogical reasoning when $A$ is present in the target case. This yields the following form:\medskip
			
			\noindent\begin{minipage}{0.9\textwidth}
			\dirtree{%
			.1 For a source case $s$.
				.2 $A$ is present in $s$.
					.3 $A$ was law-abiding in $s$.
						.4 $A$ should be law-abiding in the target case.
					.3 $A$ was law-breaking in $s$.
						.4 $A$ should be law-breaking in the target case.
				.2 $\neg A$ is present in $s$.
					.3 $\neg A$ was law-abiding in $s$.
						.4 $A$ should be law-abiding or law-breaking in the target case.
					.3 $\neg A$ was law-breaking in $s$.
						.4 $A$ should be law-abiding in the target case..
			}		
			\end{minipage}\medskip	
		
		Correspondingly, we can describe the process of analogical reasoning when $\neg A$ is present in the target case. If $\neg A$ is also present in the source case, we speak about a positive analogy\index{analogy!positive} since $\neg A$ is shared between the target and the source case. The decision whether $\neg A$ was law-abiding or law-breaking can be directly transferred to the target case. We can represent this in the following way, where we suppose that $\neg A$ is present in the target case:\medskip
			
			\noindent\begin{minipage}{0.9\textwidth}
			\dirtree{%
			.1 For a source case $s'$.
				.2 $\neg A$ is present in $s'$.
					.3 $\neg A$ was law-abiding in $s'$.
						.4 $\neg A$ should be law-abiding in the target case.
					.3 $\neg A$ was law-breaking in $s'$.
						.4 $\neg A$ should be law-breaking in the target case..
			}
			\end{minipage}	\medskip			
		
		When $\neg A$ is present in the target case, we might also speak about negative analogies. A negative analogy\index{analogy!negative} occurs when $A$ is present in the source case. This can be represented in the following way:\medskip
			
			\noindent\begin{minipage}{0.9\textwidth}
			\dirtree{%
			.1 For a source case $s'$.
				.2 $A$ is present in $s'$.
					.3 $A$ was law-abiding in $s'$.
						.4 $\neg A$ should be law-abiding or law-breaking in the target case.
					.3 $A$ was law-breaking in $s'$.
						.4 $\neg A$ should be law-abiding in the target case..
			}		
			\end{minipage}\medskip			
		
		If $A$ is law-abiding in the source case, $A$ can either be obligatory\index{action!obligatory} or permissible\index{action!permissible}. This means that $\neg A$ can be either law-breaking if it is an obligatory\index{action!obligatory} action or law-abiding if it is a permissible\index{action!permissible} action. If $A$ is law-breaking, the deontic status\index{deontic!status} of $A$ can only be that of a forbidden\index{action!forbidden} action. There are no other deontic categories\index{deontic!category} where $A$ is law-breaking. Since in forbidden\index{action!forbidden} actions, $\neg A$ is law-abiding we can include this in the representation.
		
		We then end up with a procedure for handling analogical reasoning when $\neg A$ is present in the target case. This yields the following form:\medskip
			
			\noindent\begin{minipage}{0.9\textwidth}
			\dirtree{%
			.1 For a source case $s'$.
				.2 $A$ is present in $s'$.
					.3 $A$ was law-abiding in $s'$.
						.4 $\neg A$ should be law-abiding or law-breaking in the target case.
					.3 $A$ was law-breaking in $s'$.
						.4 $\neg A$ should be law-abiding in the target case.
				.2 $\neg A$ is present in $s'$.
					.3 $\neg A$ was law-abiding in $s'$.
						.4 $\neg A$ should be law-abiding in the target case.
					.3 $\neg A$ was law-breaking in $s'$.
						.4 $\neg A$ should be law-breaking in the target case..
			}		
			\end{minipage}\medskip				
		
	\subsection{Representing the analogical procedure in CTT}
		
		By combining the procedures for handling analogies when $A$ is present and when $\neg A$ is present in the target case, we end up with a description of the whole process for analogical reasoning with imperatives. This can be represented by the following description:\medskip
		
			\noindent\begin{minipage}{0.9\textwidth}
			\dirtree{%
			.1 For a target case where $A \lor \neg A$ is present.
				.2 $A$ is present in the target case.
					.3 For a source case $s$.
						.4 $A$ is present in $s$.
							.5 $A$ was law-abiding in $s$.
								.6 $A$ should be law-abiding in the target case.
							.5 $A$ was law-breaking in $s$.
								.6 $A$ should be law-breaking in the target case.
						.4 $\neg A$ is present in $s$.
							.5 $\neg A$ was law-abiding in $s$.
								.6 $A$ should be law-abiding or law-breaking in the target case.
							.5 $\neg A$ was law-breaking in $s$.
								.6 $A$ should be law-abiding in the target case.
				.2 $\neg A$ is present in the target case.
					.3 For a source case $s'$.
						.4 $A$ is present in $s'$.
							.5 $A$ was law-abiding in $s'$.
								.6 $\neg A$ should be law-abiding or law-breaking in the target case.
							.5 $A$ was law-breaking in $s'$.
								.6 $\neg A$ should be law-abiding in the target case.
						.4 $\neg A$ is present in $s'$.
							.5 $\neg A$ was law-abiding in $s'$.
								.6 $\neg A$ should be law-abiding in the target case.
							.5 $\neg A$ was law-breaking in $s'$.
								.6 $\neg A$ should be law-breaking in the target case..	
			}	
			\end{minipage}	\medskip			
		

		The structure of the formalisation follows the patterns as described for the general analogies. If $A$ is present in the target case, the process can be formalised in the following way: \medskip
		
				
				\noindent\begin{minipage}{0.95\textwidth}
				\dirtree{%
					.1 $(\forall s : Source)$. 
						.2 $(\forall z : A(s) \lor \neg A(s))$.
							.3 $(\forall u_1 : A(s))$.
								.4 $\textbf{left}^\lor (u_1) =_{\{H2\}} z $.
									.5 $(\forall v_1 : LA_A(s) \lor LB_A(s))$.
										.6 $(\forall w_1 : LA_A(s))$.
											.7 $\textbf{left}^\lor (w_1) =_{\{H3\}} v_1 $.
												.8 $A$ should be law-abiding in the target case.
										.6 $(\forall w_2 : LB_A(s))$.
											.7 $\textbf{right}^\lor (w_1) =_{\{H3\}} v_1 $.
												.8 $A$ should be law-breaking in the target case.
							.3 $(\forall u_2 : \neg A(s))$.
								.4 $\textbf{right}^\lor (u_2) =_{\{H2\}} z$.
									.5 $(\forall v_2 : LA_{\neg A}(s) \lor LB_{\neg A}(s))$.
										.6 $(\forall w_3 : LA_{\neg A}(s))$.
											.7 $\textbf{left}^\lor (w_3) =_{\{H4\}} v_2 $.
												.8 $A$ should be either law-abiding or law-breaking in the target case.
										.6 $(\forall w_4 : LB_{\neg A}(s))$.
											.7 $\textbf{right}^\lor (w_4) =_{\{H4\}} v_2 $.
												.8 $A$ should be law-abiding in the target case..
				}
				\end{minipage}\medskip	
		
		Similarly, if $\neg A$ is present in the target case, the process can be formalised in the following way: \medskip
				
				\noindent\begin{minipage}{0.95\textwidth}
				\dirtree{%
					.1 $(\forall s' : Source)$. 
						.2 $(\forall z' : A(s') \lor \neg A(s'))$.
							.3 $(\forall u'_1 : A(s'))$.
								.4 $\textbf{left}^\lor (u'_1) =_{\{H2'\}} z' $.
									.5 $(\forall v'_1 : LA_A(s') \lor LB_A(s'))$.
										.6 $(\forall w'_1 : LA_A(s'))$.
											.7 $\textbf{left}^\lor (w'_1) =_{\{H3\}} v'_1 $.
												.8 $\neg A$ should be either law-abiding or law-breaking in the target case.
										.6 $(\forall w'_2 : LB_A(s'))$.
											.7 $\textbf{right}^\lor (w'_1) =_{\{H3'\}} v'_1 $.
												.8 $\neg A$ should be law-abiding in the target case.
							.3 $(\forall u'_2 : \neg A(s'))$.
								.4 $\textbf{right}^\lor (u'_2) =_{\{H2'\}} z'$.
									.5 $(\forall v'_2 : LA_{\neg A}(s') \lor LB_{\neg A}(s'))$.
										.6 $(\forall w'_3 : LA_{\neg A}(s'))$.
											.7 $\textbf{left}^\lor (w'_3) =_{\{H4\}} v'_2 $.
												.8 $\neg A$ should be law-abiding in the target case.
										.6 $(\forall w'_4 : LB_{\neg A}(s'))$.
											.7 $\textbf{right}^\lor (w'_4) =_{\{H4'\}} v'_2 $.
												.8 $\neg A$ should be law-breaking in the target case..
				}
				\end{minipage} \medskip		
		
		The complete analysis of analogical reasoning with imperatives in CTT yields the following formula:

				\scalebox{0.74}{
				\noindent\begin{minipage}{\textwidth}
				\vspace{\baselineskip}
				\dirtree{%
					.1 $(x : A \lor \neg A)$ \\ $b(x) : $.
						.2 $(\forall y : A)$.
							.3 $\textbf{left}^\lor(y) =_{\{H1\}} x$.
								.4 $(\forall s : Source)$. 
									.5 $(\forall z : A(s) \lor \neg A(s))$.
										.6 $(\forall u_1 : A(s))$.
											.7 $\textbf{left}^\lor (u_1) =_{\{H2\}} z $.
												.8 $(\forall v_1 : LA_A(u_1) \lor LB_A(u_1))$.
													.9 $(\forall w_1 : LA_A(u_1))$.
														.10 $\textbf{left}^\lor (w_1) =_{\{H3\}} v_1 $.
															.11 $LA_A(y)$.
													.9 $(\forall w_2 : LB_A(u_1))$.
														.10 $\textbf{right}^\lor (w_2) =_{\{H3\}} v_1 $.
															.11 $LB_A(y)$.
										.6 $(\forall u_2 : \neg A(s))$.
											.7 $\textbf{right}^\lor (u_2) =_{\{H2\}} z$.
												.8 $(\forall v_2 : LA_{\neg A}(u_2) \lor LB_{\neg A}(u_2))$.
													.9 $(\forall w_3 : LA_{\neg A}(u_2))$.
														.10 $\textbf{left}^\lor (w_3) =_{\{H4\}} v_2 $.
															.11 $LA_A(y) \lor LB_A(y)$.
													.9 $(\forall w_4 : LB_{\neg A}(u_2))$.
														.10 $\textbf{right}^\lor (w_4) =_{\{H4\}} v_2 $.
															.11 $LA_A(y)$.
						.2 $(\forall y' : \neg A)$. 
							.3 $\textbf{right}^\lor(y') =_{\{H1\}} x $.
								.4 $(\forall s' : Source)$. 
									.5 $(\forall z' : A(s') \lor \neg A(s'))$.
										.6 $(\forall u'_1 : A(s'))$.
											.7 $\textbf{left}^\lor (u'_1) =_{\{H2'\}} z' $.
												.8 $(\forall v'_1 : LA_A(u'_1) \lor LB_A(u'_1))$.
													.9 $(\forall w'_1 : LA_A(u'_1))$.
														.10 $\textbf{left}^\lor (w'_1) =_{\{H3\}} v'_1 $.
															.11 $LA_{\neg A}(y') \lor LB_{\neg A}(y')$.
													.9 $(\forall w'_2 : LB_A(u'_1))$.
														.10 $\textbf{right}^\lor (w'_2) =_{\{H3'\}} v'_1 $.
															.11 $LA_{\neg A}(y')$.
										.6 $(\forall u'_2 : \neg A(s'))$.
											.7 $\textbf{right}^\lor (u'_2) =_{\{H2'\}} z'$.
												.8 $(\forall v'_2 : LA_{\neg A}(u'_2) \lor LB_{\neg A}(u'_2))$.
													.9 $(\forall w'_3 : LA_{\neg A}(u'_2))$.
														.10 $\textbf{left}^\lor (w'_3) =_{\{H4\}} v'_2 $.
															.11 $LA_{\neg A}(y')$.
													.9 $(\forall w'_4 : LB_{\neg A}(u'_2))$.
														.10 $\textbf{right}^\lor (w'_4) =_{\{H4\}} v'_2 $.
															.11 $LB_{\neg A}(y')$..
				}
				\end{minipage}
				}
				
				
			
		
			
		The formula that we end up with is the CTT representation of step 4 to 7 in the description of the process of analogical reasoning with imperatives. By including the context that was described earlier, we would also formalise step 3. After step 3, the first investigation that occurs is formalised as $x : A \lor \neg A$ and represents the investigation whether the chosen action-proposition is present or absent in the target case. After it has been decided whether $A$ is present or absent in the target case, the source cases are introduced and for every source case the investigation whether the action-proposition is present or absent in this source case can be formalised as $(\forall z : A(s) \lor \neg A(s))$ and $(\forall z' : A(s') \lor \neg A(s'))$. When it is decided whether $A$ is present or absent in the source case, the next question is whether this action was law-abiding or law-breaking in the source case, formalised as $(\forall v_1 : LA_A (u_1) \lor LB_A (u_1))$, $(\forall v_2 : LA_{\neg A} (u_2) \lor LB_{\neg A} (u_2))$, $(\forall v_1' : LA_A (u_1') \lor LB_A (u_1'))$ and $(\forall v_2' : LA_{\neg A} (u_2') \lor LB_{\neg A} (u_2'))$. If the action was performed (or not performed) in both the target case and the source case, the decision of law-abidingness or law-breakingness in the source case can be directly transferred to the target case. This is the situation for $w_1$, $w_2$, $w'_3$ and $w'_4$. Such transfer is formalised by binding the law-abidingness or law-breakingness to the choice that was performed in the target case, $y$ or $y'$, providing a positive analogy\index{analogy!positive}. 
		
		The other alternative is that the action was performed in either the target case or the source case and not performed in the other. This is the foundation for negative analogies. Since we describe analogy for deontic notions, namely law-abidingness and law-breakingness, this will be based upon the three previously described deontic categories\index{deontic!category}. The negative analogies occur in $w_3$, $w_4$, $w'_1$ and $w'_2$. In $w_3$, $\neg A$ is law-abiding in the source case, $LA_{\neg A}(u_2)$, and $A$ should therefore be either law-abiding or law-breaking in the target case, $LA_A(y) \lor LB_A(y)$. In $w_4$, $\neg A$ is law-breaking in the source case, $LB_{\neg A}(u_2)$, and $A$ should therefore be law-abiding in the target case, $LA_A(y)$. In $w'_1$, $A$ is law-abiding in the source case, $LA_A(u'_1)$, and $\neg A$ should therefore be either law-abiding or law-breaking in the target case, $LA_{\neg A}(y') \lor LB_{\neg A}(y')$. In $w'_2$, $A$ is law-breaking in the source case, $LB_A(u'_1)$, and $\neg A$ should therefore be law-abiding in the target case, $LA_{\neg A}(y)$. 
		
\section{Steamboat example}\label{SteamboatExampleCTT}
	
		\subsection{Assumptions and description of example}
	
			We will illustrate the procedure by a commonly used example in the literature, namely the steamboat example. It is taken from \textcite[pp. 1003-1005]{Brewer1996} and is the case of \textsf{\textit{Adams v. New Jersey Steamboat Co.}, 151 N.Y. 163 (1896)}\index{Adams v. New Jersey Steamboat Co., 151 N.Y. 163 (1896)}. 
		
			The example is a case about whether or not a certain steamboat owner is liable for a theft from a customer. \textit{Adams} had rented a cabin in a steamboat owned by \textit{New Jersey Steamboat Co.} and some of his valuables were stolen. The question was whether the steamboat company was strictly liable towards \textit{Adams}. There were (at least) two references to older cases, one where an innkeeper had been held strictly liable for the thefts of valuables of a customer, and one where a railway company had not been held strictly liable for theft of valuables from an open-berth sleeping car. The question was therefore whether the steamboat was like a railroad or like an inn from a legal viewpoint. The judge decides that the steamboat was similar to an inn, and that the steamboat company therefore was strictly liable for the theft of valuables from its customer. The judge claimed that the relevant similarities were that \textit{the client paid for a room for some specified reasons and that the company has tempting opportunity for fraud and plunder of the client}. The previous case with an inn was similar in both of these aspects, while the railroad case was not similar in either. \autocite[,1003-1005, 1013-1015]{Brewer1996}
			
			There are some preliminary remarks regarding the explanation of the example in the procedure. First, we suppose that there are only these two precedents, the inn-case and the railroad-case, to be considered in the present example. Second, we suppose that \textit{New Jersey Steamboat Co.} refused to compensate \textit{Adams} for his stolen valuables, which otherwise would not seem to be a legal case in the first place. The question is then whether \textit{New Jersey Steamboat Co.}'s refusal to compensate was according to the law, whether it was law-abiding or law-breaking. 
			
			In the steamboat example, we have two different precedents that are considered to be potentially relevant for the case at hand, the innkeeper case and the railway case. We will consider both of these cases in the analysis. The relevant similarities that made the case be similar to the innkeeper case and not the railway case was that \textit{the client paid for a room for some specified reasons and that the company has tempting opportunity for fraud and plunder of the client}. \textcite[p. 1005]{Brewer1996} analyses this as two distinct requirements, by two different predicates. We might analyse it in a similar way, as a conjunction of two predicates, $(F' \land G') : prop$, though since they will always occur together in the argument, we choose to represent them by a single predicate, so that $F : prop$ stands for the proposition \textit{refusing strict liability for the theft of valuables when the client paid for a room for some specified reasons and that the company has tempting opportunity for fraud and plunder of the client}. We will first show how the example can be explained in terms of the informal procedure and then show how it can be transformed into a CTT analysis. 
			
		\subsection{Informal analysis}
			%
			\begin{enumerate}
				\item The first step is the creation of a target case, which is the disagreement because \textit{New Jersey Steamboat Co.} refused to compensate \textit{Adams} for the theft of his valuables. 
				\item The second step is the choice of action-proposition, which in this case is \textit{refusing strict liability for theft of valuables when the client paid for a room for some specified reasons and that the company has tempting opportunity for fraud and plunder of the client}. This was decided by the judge to be the relevant similarity\index{relevant similarity} and therefore also what caused that decision in the case.
				\item The third step is about the well-definition of terms and the permissibility of analogical argumentation. We should know what it means for a case to be covered by the chosen action-proposition, namely that we can tell whether in a precedent there have been \textit{refusing strict liability for theft of valuables when the client paid for a room for some specified reasons and that the company has tempting opportunity for fraud and plunder of the client} or not. We should also know what source cases or precedents we are bound by, here the innkeeper case and the railroad case. The last part is that we have to make sure that the framework allows for the use of analogical argumentation regardless of the deontic status\index{deontic!status} of the action-proposition. In Common Law\index{Common Law}, the legal framework of this case, analogical argumentation lies in its core and the principle of stare decisis\index{stare decisis text@\textit{stare decisis}} binds the judge to the legal result of the source cases. The judge is also in a context of doubt, meaning that there is no clear answer to this problem. Because of this, the use of analogical argumentation is assumed to be permitted. 
				\item The fourth step is the decision whether the action-proposition covers the target case or not. The question is then whether in the case at hand, \textit{New Jersey Steamboat Co.} was \textit{refusing strict liability for theft of valuables when the client paid for a room for some specified reasons and that the company has tempting opportunity for fraud and plunder of the client}. This action was performed in the target case.
				\item At this step we should find some source case and decide whether or not the action was performed in this source case. In this situation we have two alternatives here, the innkeeper case and the railway case. This motivates two different paths ahead. We will start by the railway case.
			\end{enumerate}
			
			
			\paragraph{The railway case:}
			
			\begin{enumerate}[leftmargin=4\parindent]
				\setcounter{enumi}{4}
				
				\item In the source case, the railway case, the action \textit{refusing strict liability for theft of valuables when the client paid for a room for some specified reasons and that the company has tempting opportunity for fraud and plunder of the client} was not performed. 
				\item In the source case, the railway company was not held strictly liable for the theft of valuables for its customer. It was not acting against the law by refusing to compensate its customer, which means that refusing compensation for a theft when \textit{the client paid for a room for some specified reasons and that the company has tempting opportunity for fraud and plunder of the client} was not performed, was judged to be law-abiding in the source case.
				\item Since in the steamboat case, the action \textit{refusing strict liability for theft of valuables when the client paid for a room for some specified reasons and that the company has tempting opportunity for fraud and plunder of the client} was performed while in the railway case it was not, the source case and the target case do not have the same status. The action was performed in the target case, while not performed in the source case. This means that we are in the situation of (b), a different status for the action-proposition in the source case and the target case. The non-performance in the source case was also judged to be law-abiding. This means that we are in the situation of i., that the case was law-abiding in the source case. We can therefore infer that the performance in the target case should be law-abiding or law-breaking. This means that this action could be either forbidden\index{action!forbidden} or legally permissible\index{action!permissible}. 
				
			\end{enumerate}
			
			
			\paragraph{The innkeeper case:}
			
			\begin{enumerate}[leftmargin=4\parindent]
				\setcounter{enumi}{4}
				
				\item In the source case, the innkeeper case, the action \textit{refusing strict liability for theft of valuables when the client paid for a room for some specified reasons and that the company has tempting opportunity for fraud and plunder of the client} was performed. 
				\item In the source case, the innkeeper was strictly  liable for the theft of valuables for its customer. It was acting against the law by refusing to compensate its customer. This means that refusing compensation for a theft when \textit{the client paid for a room for some specified reasons and that the company has tempting opportunity for fraud and plunder of the client}, was judged to be law-breaking in the source case.
				\item Since both in the case at hand and in the innkeeper case, the action \textit{refusing strict liability for theft of valuables when the client paid for a room for some specified reasons and that the company has tempting opportunity for fraud and plunder of the client} was performed, the source case and the target case do have the same status. The action was performed both in the target case and in the source case. This means that we are in the situation of (a), the same status in the source case and the target case. The performance in the source case was also judged to be law-breaking, and we can transfer the decision of the source case to the target case. We can therefore infer that the performance in the target case should be law-breaking and that the action is forbidden\index{action!forbidden}. 
			\end{enumerate}		
			
			Based on the railway case, we can infer that the performance in the target case should be law-breaking or law-abiding, while based on the innkeeper case we can infer that the performance in the target case should be law-breaking. The railway case makes the grounds for what is called a negative analogy\index{analogy!negative} because it is based on a difference between the target and the source. The innkeeper case provides a positive analogy\index{analogy!positive}. The results of these two source cases are compatible, that an action is law-abiding or law-breaking is perfectly compatible with it being law-breaking. Based on the two source cases, we can say that the action is forbidden\index{action!forbidden}. Since the innkeeper case enables us to infer that the performance in the target case was law-breaking, we end up with the same result as the judge in the example, which was what we wanted.
			
		\subsection{Formal analysis}
			
		For the formal analysis of the argumentation process, the analysis starts at step 3. The first step is given by the conflict between \textit{Adams} and \textit{New Jersey Steamboat Co.}, for \textit{refusing strict liability for the theft of valuables}. The second step is the selection of relevant action-proposition, which is given explicitly as \textit{refusing strict liability for the theft of valuables when the client paid for a room for some specified reasons and that the company has tempting opportunity for fraud and plunder of the client}. We will therefore assume that this rule is occurs as an emergence in the legal context. What our analysis does is to show the inner structure that any such rule need to satisfy.  
			
		Step three involves the well-definition of the terms. The first part is therefore to make sure that the action-proposition is well-defined. Let $F$ represent \textit{refusing strict liability for the theft of valuables when the client paid for a room for some specified reasons and that the company has tempting opportunity for fraud and plunder of the client}, so that:
			%
			\[
				F : prop.
			\]
			%
		The next part is the definition of source cases, which in this situation are the innkeeper case and the railway case. They are not introduced explicitly at this point, but we assure the well-definition of the set,
			%
			\[
				Source : set.
			\]
			%
		The last thing we need to assure is the permission of the use of analogy. As mentioned earlier, the principle of stare decisis\index{stare decisis text@\textit{stare decisis}} in common law\index{Common Law} systems seems to permit the use of analogy to a great extent. The judge is also in a context of doubt. We then need to assure that the use of analogical reasoning is permitted for this particular action-proposition, no matter what deontic status\index{deontic!status} the relevant action-proposition might have. This is done in the following way for the first deontic category, of obligatory\index{action!obligatory} actions:
			%
			\[
				PA_1(z_1) : set (z_1 : LA_F \lor LB_{\neg F}),
			\]
		and similarly for the other deontic categories\index{deontic!category}. We then end up with the following context for the steamboat example:
			%
			\[
			\begin{array}{c}
				F : prop,                                                                                                       \\
				Source : set,                                                                                                   \\
				PA_1(z_1) : set (z_1 : LA_F \lor LB_{\neg F}),                                                                  \\
				PA_2(z_2) : set (z_2 : LB_F \lor LA_{\neg F}),                                                                  \\
				PA_3(z_3) : set (z_3 : LA_F \lor LA_{\neg F}).                                          							\\
			\end{array}
			\]
				
		Step four is deciding whether the action-proposition is present or absent in the target case. This will be represented by the following incomplete formula: \newline\medskip
			
			\noindent\begin{minipage}{\textwidth}
			\dirtree{%
				.1 $(x : F \lor \neg F)$ \\ $b(x) : $.
					.2 $(\forall y : F)$.
						.3 $\textbf{left}^\lor(y) =_{\{H1\}} x$.
							.4 $...$.
					.2 $(\forall y' : \neg F)$. 
						.3 $\textbf{right}^\lor(y') =_{\{H1\}} x $.
							.4 $...$..
			}
			\end{minipage} \medskip
			
		\noindent The complete formulation of the example is then the following: 

				\scalebox{0.74}{
				\noindent\begin{minipage}{\textwidth}
				\vspace{\baselineskip}
				\dirtree{%
					.1 $(x : F \lor \neg F)$ \\ $b(x) : $.
						.2 $(\forall y : F)$.
							.3 $\textbf{left}^\lor(y) =_{\{H1\}} x$.
								.4 $(\forall s : Source)$. 
									.5 $(\forall z : F(s) \lor \neg F(s))$.
										.6 $(\forall u_1 : F(s))$.
											.7 $\textbf{left}^\lor (u_1) =_{\{H2\}} z $.
												.8 $(\forall v_1 : LA_F(u_1) \lor LB_F(u_1))$.
													.9 $(\forall w_1 : LA_F(u_1))$.
														.10 $\textbf{left}^\lor (w_1) =_{\{H3\}} v_1 $.
															.11 $LA_F(y)$.
													.9 $(\forall w_2 : LB_F(u_1))$.
														.10 $\textbf{right}^\lor (w_2) =_{\{H3\}} v_1 $.
															.11 $LB_F(y)$.
										.6 $(\forall u_2 : \neg F(s))$.
											.7 $\textbf{right}^\lor (u_2) =_{\{H2\}} z$.
												.8 $(\forall v_2 : LA_{\neg F}(u_2) \lor LB_{\neg F}(u_2))$.
													.9 $(\forall w_3 : LA_{\neg F}(u_2))$.
														.10 $\textbf{left}^\lor (w_3) =_{\{H4\}} v_2 $.
															.11 $LA_F(y) \lor LB_F(y)$.
													.9 $(\forall w_4 : LB_{\neg F}(u_2))$.
														.10 $\textbf{right}^\lor (w_4) =_{\{H4\}} v_2 $.
															.11 $LA_F(y)$.
						.2 $(\forall y' : \neg F)$. 
							.3 $\textbf{right}^\lor(y') =_{\{H1\}} x $.
								.4 $(\forall s' : Source)$. 
									.5 $(\forall z' : F(s') \lor \neg F(s'))$.
										.6 $(\forall u'_1 : F(s'))$.
											.7 $\textbf{left}^\lor (u'_1) =_{\{H2'\}} z' $.
												.8 $(\forall v'_1 : LA_F(u'_1) \lor LB_F(u'_1))$.
													.9 $(\forall w'_1 : LA_F(u'_1))$.
														.10 $\textbf{left}^\lor (w'_1) =_{\{H3\}} v'_1 $.
															.11 $LA_{\neg F}(y') \lor LB_{\neg F}(y')$.
													.9 $(\forall w'_2 : LB_F(u'_1))$.
														.10 $\textbf{right}^\lor (w'_2) =_{\{H3'\}} v'_1 $.
															.11 $LA_{\neg F}(y')$.
										.6 $(\forall u'_2 : \neg F(s'))$.
											.7 $\textbf{right}^\lor (u'_2) =_{\{H2'\}} z'$.
												.8 $(\forall v'_2 : LA_{\neg F}(u'_2) \lor LB_{\neg F}(u'_2))$.
													.9 $(\forall w'_3 : LA_{\neg F}(u'_2))$.
														.10 $\textbf{left}^\lor (w'_3) =_{\{H4\}} v'_2 $.
															.11 $LA_{\neg F}(y')$.
													.9 $(\forall w'_4 : LB_{\neg F}(u'_2))$.
														.10 $\textbf{right}^\lor (w'_4) =_{\{H4\}} v'_2 $.
															.11 $LB_{\neg F}(y')$..
				}
				\end{minipage}
				}
				
			
		Regarding the fourth step, $F$ is present, so the first alternative should be chosen:
			
			\noindent\begin{minipage}{\textwidth}
			\dirtree{%
				.1  .
					.2 $(\forall y : F)$.
						.3 $\textbf{left}^\lor(y) =_{\{H1\}} x$.
							.4 $...$..
			}
			\end{minipage}\medskip			
		
		The fifth step depends on whether we use the railway case or the innkeeper case as a basis for the analysis. As with the informal representation, the formal analysis also takes different paths based on the two cases. The decision in the source case is represented by a quantification on the set of source cases,
		%
			\[
				(\forall s : Source).
			\]
		
		This is followed up by an inquiry on whether the chosen action-proposition is present or absent in the source case. This inquiry is represented as:
		%
			\[
				(\forall z : F(s) \lor \neg F(s)),
			\]	
		which gives us the following procedure in the tree notation:
		
			\noindent\begin{minipage}{\textwidth}
			\dirtree{%
				.1	.
					.2 $(\forall s : Source)$. 
						.3 $(\forall z : F(s) \lor \neg F(s))$.
							.4 $(\forall u_1 : F(s))$.
								.5 $...$ .
							.4 $(\forall u_2 : \neg F(s))$.
								.5 $...$..
			}
			\end{minipage}\medskip	
			
		\noindent It is in answering this inquiry that the two precedents differ. 
			
		\paragraph{The railway case}
			
			In the railway-case, the action-proposition is absent. This means that we will follow the path of $u_2$. The answer of this inquiry has the following form:
			%
				\[
					(\forall u_2 : \neg F(s)) \textbf{right}^\lor (u_2) =_{\{H2\}} z,
				\]
			which is represented in the following way in the tree notation:
				\noindent\begin{minipage}{\textwidth}
				\dirtree{%
				.1 .
					.2 $(\forall s : Source)$. 
						.3 $(\forall z : F(s) \lor \neg F(s))$.
							.4 $(\forall u_2 : \neg F(s))$.
								.5 $\textbf{right}^\lor (u_2) =_{\{H2\}} z$.
									.6 $...$..
				}
				\end{minipage}\medskip
			
			\noindent The inquiry is answered by the proof object $u_2$ for $\neg F$.
			
			The sixth step is deciding whether the action-proposition was judged law-abiding or law-breaking in the source case. The decision of the railway case was that the railway company was not held strictly liable for the theft of their customers. This means that the railway company was acting law-abiding when refusing this strict liability. The decision was not based on the absence of the action-proposition in particular, but since the action-proposition was absent in the source case, it implies that this absence was considered law-abiding. This decision is represented as an inquiry of the form:
			%
				\[
					(\forall v_1 : LA_{\neg F}(s) \lor LB_{\neg F}(s)).
				\]
				%
			As with the other inquiries, there are two potential answers, that it was law-abiding, $LA_{\neg F}(s')$, or that it was law-breaking, $LB_{\neg F}(s')$. In the railway case, the first alternative was the situation and this is represented in the following way:
			%
				\[
					(\forall w_3 : LA_{\neg F}(u_2)) \textbf{left}^\lor (w_3) =_{\{H4\}} v_2,
				\]
			which gives us the path of $w_3$ in the tree notation: \newline
				\noindent\begin{minipage}{\textwidth}
				\dirtree{%
				.1  .
					.2 $(\forall v_2 : LA_{\neg F}(u_2) \lor LB_{\neg F}(u_2))$.
						.3 $(\forall w_3 : LA_{\neg F}(u_2))$.
							.4 $\textbf{left}^\lor (w_3) =_{\{H4\}} v_2 $.
								.5 $...$..
				}
				\end{minipage}\medskip
				
			The seventh and last step is the application to the target case. Since we have established that the action-proposition was present in the target case, while absent in the source case, the action has a different status in the target case and in the source case. We are therefore in the case of (b). The non-performance in the source case was judged law-abiding. We can infer from the analysis that the performance in the target case should be judged either law-abiding or law-breaking in the target case, namely the following:
			%
				\[
					LA_F(y) \lor LB_F(y),
				\]	
			which reads that the performance of $F$ is either law-abiding or law-breaking in the target case. This corresponds to the result we ended up with in the informal description. 
			
		\paragraph{The innkeeper case}
			
			In the innkeeper case the action-proposition is present. This means that we will follow the path of $u_1$. The answer of this inquiry has the following form:
			% 
				\[
					(\forall u_1 : F(s)) \textbf{left}^\lor (u_1) =_{\{H2\}} z,
				\]
			which is represented in the following way in the tree notation:
			\noindent\begin{minipage}{\textwidth}
			\dirtree{%
			.1 .
				.2 $(\forall s : Source)$. 
					.3 $(\forall z : F(s) \lor \neg F(s))$.
						.4 $(\forall u_1 : F(s))$.
							.5 $\textbf{left}^\lor (u_1) =_{\{H2\}} z$.
								.6 $...$..
			}
			\end{minipage}\medskip
			
			The inquiry is answered by the proof object $u_1$ for $F$.
			
			The sixth step is deciding whether the action-proposition was judged law-abiding or law-breaking in the source case. The decision of the innkeeper case was that the innkeeper was held strictly liable for the theft of their customers. This means that the innkeeper was acting law-breaking when refusing this strict liability. This decision is represented as an inquiry of the form:
			%
				\[
					(\forall v_1 : LA_F(s) \lor LB_F(s)).
				\]
				%
			As with the other inquiries, there are two potential answers, that is was law-abiding, $LA_F(s)$, or that it was law-breaking, $LB_F(s)$. In the innkeeper case, the second alternative was decided, and this is represented in the following way:
			%
				\[
					(\forall w_2 : LA_F(u_1)) \textbf{right}^\lor (w_2) =_{\{H4\}} v_1,
				\]
			which gives us the path of $w_2$ in the tree notation: \newline
				\noindent\begin{minipage}{\textwidth}
				\dirtree{%
				.1  .
					.2 $(\forall v_1 : LA_F(u_1) \lor LB_F(u_1))$.
						.3 $(\forall w_2 : LA_F(u_1))$.
							.4 $\textbf{right}^\lor (w_2) =_{\{H4\}} v_1 $.
								.5 $...$..
				}
				\end{minipage}\medskip
				
			The seventh and last step is the application to the target case. Since we have established that the action-proposition was present both in the target case and in the source case, the action has the same status in the target case and in the source case. We are therefore in the situation of (a). The performance in the source case was judged law-breaking. We can infer from the analysis that the performance in the target case should be judged law-breaking, namely the following:
			%
				\[
					LB_F (y),
				\]	
			which reads that the performance of $F$ is law-breaking in the target case. This corresponds to the result we ended up with in the informal description. Because of the similarity to the innkeeper case, its decision can be transferred to steamboat case, which is precisely the result that we expected. Note that this result if perfectly compatible with the result for the railway case, which said that the performance in the target case should be either law-abiding or law-breaking.
		
\addtocontents{toc}{\protect\pagebreak[4]}		
\chapter{Dialogical implementation}\label{Sec:DialogicalImplementation}
	
	\section{Precedent-based reasoning}
	
		\subsection{Terminology}
	
		Until now the analysis has been described in type-theoretical terms. However, we will see that immanent reasoning\index{immanent reasoning} provides a natural and comprehensible alternative explanation to the procedure previously described. The dialogical analysis of analogical reasoning will show us that we might consider analogy to consist of eight different dialogical rules, one for each form of analogy, instead of one complex framework. The dialogical framework enables us to do this because of the repetition ranks of the players. A repetition rank of 1 will provide us with an \textit{unrestricted} analogy, as it does not enable the play to compare different source cases. For a \textit{restricted} analogy, the players would need repetition ranks of 2 or higher. We will show that the repetition ranks will decide how many source cases that might be brought into the play. The repetition rank is a particularity of the dialogical framework that is not present in the general formulation of CTT. Since it is the repetition rank that enables us to separate analogical reasoning into different rules, we are only enabled to do so in this dialogical interpretation, not in the general CTT representation.
		
		The dialogical representation will also enable us to express the permission of a particular analogy form\index{analogy!form} only when this kind of analogy is actually utilised. This means that we are not bound to permit all forms of analogical arguments, but we can permit only the kinds that are introduced in the particular play. However, there is no logical reason that this can only be done in the dialogical implementation of CTT, though it facilitates the integration in a natural and comprehensible way. To integrate such dependent permission of analogical arguments without the dialogical framework would require a formalisation that is significantly more complex than what has previously been introduced. 
		
		Since we will introduce eight different forms of analogy, we will categorise them in a particular way. As previously mentioned, an analogy can be said to depend on three inquiries, namely: 
		%
			\begin{enumerate}
				\item the presence or absence of the occasioning characteristic\index{characteristic!occasioning} in the target case;
				\item the presence or absence of the occasioning characteristic\index{characteristic!occasioning} in the source case;
				\item the presence or absence of the entailed characteristic\index{characteristic!entailed} in the source case.
			\end{enumerate}
		
		We might then categorise the three different steps, according to whether the characteristic is present or absent. An analogy where the occasioning characteristic\index{characteristic!occasioning} is present both in the target case and in the source case, and where the entailed characteristic\index{characteristic!entailed} is present in the source case will be called a \textit{Present-present-present-analogy}, or 'PPP-analogy' for short. An analogy where the occasioning characteristic\index{characteristic!occasioning} is absent both in the target case and in the source case, and where the entailed characteristic\index{characteristic!entailed} is absent in the source case will be called a \textit{Absent-absent-absent analogy} or 'AAA-analogy' for short. An analogy where the occasioning characteristic\index{characteristic!occasioning} is present in the target case and absent in the source case, and where the entailed characteristic\index{characteristic!entailed} is present in the source case will be called a \textit{Present-absent-present analogy} or 'PAP-analogy' for short. Similar descriptions hold for the other alternatives. This section will start by describing the general analogies that infer a property in the target case. Analogies with deontic qualification\index{deontic!qualification}s will be introduced by simply adding two additional definitions.
		
		To provide the dialogical implementation of the mentioned procedure, we will introduce two kinds of rules, challenge rules\index{rule!challenge} and explanation rule\index{rule!explanation}s, together with corresponding formation rule\index{rule!formation}s. As mentioned, we can separate the analogical procedure into eight different forms of analogical argument and these rules are intended to provide a way to break the formula into different parts that can more easily be utilised in the dialogical framework. The expression 'Analogy$[A,B]$' can in some sense be understood as standing for the procedure of analogy described in \autoref{sec:general-precedent-based-reasoning}, where the presence or absence of $A$ in the target case and the source case, together with the presence or absence of $B$ in the target case, provide justification for the presence or absence of $B$ in the target case. This complex expression can however be separated into different parts, described by the eight different forms of analogy. To do this operation, we have two Analogy Challenge Rule\index{rule!challenge}s that can be used when a player has stated an analogy between $A$ and $B$.				
	
		We previously argued for considering analogy to be a question of \textit{similar relevancy\index{relevancy}} rather than similarity \textit{and} relevancy\index{relevancy}. This dialogical approach takes this seriously and we can see that in the formulation of a particular analogy form\index{analogy!form}, the dependency in the source case is connected to the dependency in the target case by an intuitionistic implication. It does not introduce any distinct notion of similarity and relevancy\index{relevancy}, but rather it considers the combined notion to be an aspect of a general dependency of dependencies. In such way, this approach seem to provide meaning explanation\index{meaning!explanation}s of an analogies closely related to the Aristotelean notion of proportionality\index{proportionality}. 
	
	
		\subsection{Challenge rules\index{rule!challenge}}
		
		A player can propose that an analogy holds between $A$ and $B$. This is done by the move '! Analogy$[A,B]$'. The general formation requirements for such statement is given in \autoref{AnalogyFormation}.
		
					\begin{Scheme}[h] 
					\small
        			\centering
               		\begin{tabularx}{\textwidth}{c|c|X}
               			      \textbf{Move}        &            \textbf{Challenge}            &        \textbf{Defence}         \\ \toprule
               										& \textbf{Y} ? F$_{Analogy[A,B]1}$ & \textbf{X} $A : prop$ \\
               			                           &                    Or                    &                    \\
               			                           & \textbf{Y} ? F$_{Analogy[A,B]2}$ & \textbf{X} $Tar : set$ \\
               			     \textbf{X} Analogy$[A,B] : prop$ &                    Or                    &                   \\
               			                           & \textbf{Y} ? F$_{Analogy[A,B]3}$ & \textbf{X} $Sou : set$ \\
               			                           &                    Or                    &                   \\
               			                           & \textbf{Y} ? F$_{Analogy[A,B]4}$ & \textbf{X} $B(x,y) : prop[x : A$ \newline \phantom{\textbf{X}} $\lor \neg A, y : Tar \lor Sou]$ \\ \bottomrule
               		\end{tabularx}	
               		\caption{Analogy Formation Rule}
                	\label{AnalogyFormation}
					\end{Scheme}
					
		There are here four formation requirements for this statement. If we had followed the previously described general CTT explanation, we could also introduce the permission of the analogies here. However, for the reasons mentioned earlier, these requirements are rather introduced for each individual analogy form\index{analogy!form}. This is done in order to enable the formalisation to permit only those kind of analogies that are in play.
		
		The statement of an analogy between $A$ and $B$ can be challenged in two ways. The first way is by asking the other player to show what way this analogy can be used to advocate a certain result in the target case. This is done by the demand '? AnForm$[A,B]$', which stands for \textit{Analogy Form}. It is here up to the defender to choose what kind of analogy that will be presented. The Analogy Challenge Rule\index{rule!challenge} 1 is given in \autoref{AnalogyChallengeRule1}.
					
					\begin{Scheme}[h]      
					\small 		
        			\centering
               		\begin{tabular}{c|c|c}
               			\textbf{Move}        & \textbf{Challenge} & \textbf{Defence} \\ \toprule
               			& & \textbf{X} ! PPP-Analogy$[A,B]$ \\ 
               			& & Or \\
               			& & \textbf{X} ! PPA-Analogy$[A,B]$ \\
               			& & Or \\
               			& & \textbf{X} ! PAP-Analogy$[A,B]$ \\
               			& & Or \\
               			& & \textbf{X} ! PAA-Analogy$[A,B]$ \\
               			\textbf{X} ! Analogy$[A,B]$ & \textbf{Y} ? AnForm$[A,B]$ & Or \\
               			& & \textbf{X} ! APP-Analogy$[A,B]$ \\
               			& & Or \\
               			& & \textbf{X} ! APA-Analogy$[A,B]$ \\
               			& & Or \\
               			& & \textbf{X} ! AAP-Analogy$[A,B]$ \\
               			& & Or \\
               			& & \textbf{X} ! AAA-Analogy$[A,B]$ \\ \bottomrule					
               		\end{tabular}	
               		\caption{Analogy Challenge Rule\index{rule!challenge} 1}
                	\label{AnalogyChallengeRule1}
					\end{Scheme} 

			The second way to challenge an analogy statement is for the challenger to suggest a particular form of analogy. This is done after the first rule, as it is a way for the challenger to attack the proposed analogy by a counterexample\index{counterexample}. Notice that we use the term \textit{counterexample\index{counterexample}} in a very broad way. The point is that when a player proposes an analogy between $A$ and $B$, that player should also concede any analogy form\index{analogy!form}, not only the form chosen by himself. This is a result of what is called the condition of efficiency\index{requirement!efficiency} or the Proportionality\index{proportionality}-principle. The defender will then be committed to the result of the analogy form\index{analogy!form} chosen by the challenger. The challenge in this rule opens up a subplay where the challenger tries to either force the defender into an inconsistency or to make the defender unable to respond to the challenge. The Analogy Challenge Rule\index{rule!challenge} 2 is given in \autoref{AnalogyChallengeRule2}.
					
					\begin{Scheme}[h]  
					\footnotesize   	
        			\centering
               		\begin{tabular}{@{} c|c|l @{}}
               			      \textbf{Move}        &            \textbf{Challenge}            &        \textbf{Defence}         \\ \toprule
               										& \textbf{Y} ! PPP-Analogy$[A,B]$ & \textbf{X} ! $B(x,t)[x : A, t : Tar]$ \\
               			                           &                    Or                    &                                 \\
               			                           & \textbf{Y} ! PPA-Analogy$[A,B]$ & \textbf{X} ! $\neg B(x,t)[x : A, t : Tar]$ \\
               			                           &                    Or                    &                                 \\
               			                           & \textbf{Y} ! PAP-Analogy$[A,B]$ & \textbf{X} ! $\neg B(x,t)[x : \neg A, t : Tar]$ \\
               			                           &                    Or                    &                                 \\
               			                           & \textbf{Y} ! PAA-Analogy$[A,B]$ & \textbf{X} ! $\neg \neg B(x,t)[x : A, t : Tar]$ \\
               			\textbf{X} ! Analogy$[A,B]$&                    Or                    &                                 \\
               			                           & \textbf{Y} ! APP-Analogy$[A,B]$ & \textbf{X} ! $\neg B(x,t)[x : \neg A, t : Tar]$ \\
               			                           &                    Or                    &                                 \\
               			                           & \textbf{Y} ! APA-Analogy$[A,B]$ & \textbf{X} ! $\neg \neg B(x,t)[x : \neg A, t : Tar]$ \\
               			                           &                    Or                    &                                 \\
               			                           & \textbf{Y} ! AAP-Analogy$[A,B]$ & \textbf{X} ! $B(x,t)[x : \neg A, t : Tar]$ \\
               			                           &                    Or                    &                                 \\
               			                           & \textbf{Y} ! AAA-Analogy$[A,B]$ & \textbf{X} ! $\neg B(x,t)[x : \neg A, t : Tar]$ \\ \bottomrule
               		\end{tabular}	
               		\caption{Analogy Challenge Rule\index{rule!challenge} 2}
                	\label{AnalogyChallengeRule2}
					\end{Scheme}

		Regarding the Analogy Challenge Rule\index{rule!challenge} 2, it is restricted in the way that it can only be played after the Analogy Challenge Rule\index{rule!challenge} 1. This is to avoid the challenger to win the play, simply because the challenger never was forced to concede the defenders proposed analogy. Notice that this is a structural restriction, special to these analogy rules. This structural rule\index{rule!structural} is described in \autoref{AnalogyChallengeRuleRestriction}.
		
					\begin{Restriction}[H] 
					\centering
					\begin{minipage}{0.9\textwidth}      		
               			\textit{Whenever a player proposes an analogy, \textnormal{\textbf{X} ! Analogy$[A,B]$}, the other player, \textnormal{\textbf{Y}}, can only challenge it with the Analogy Challenge Rule\index{rule!challenge} 2 after challenging \textnormal{\textbf{X}}'s defence in the Analogy Challenge Rule\index{rule!challenge} 1 with the corresponding explanation rule\index{rule!explanation}.}
               			\end{minipage}
               		\caption{Analogy Challenge Rule\index{rule!challenge} Restriction}
                	\label{AnalogyChallengeRuleRestriction}
					\end{Restriction}			
			
		The third kind of rules is the explanation rule\index{rule!explanation}s. They make the challenger spell out the expression that is behind the form of analogy. These rules provide the meaning for each of the eight forms of analogy, and we therefore have eight different rules. We will describe each rule and provide the explanation rule\index{rule!explanation} related to each form of analogy. Since we distinguish the procedure for analogical reasoning into eight different forms, we are also able to attach the permission of the analogy to the formation requirement to the particular form of analogy that is utilised in the play. Notice that this approach presupposes that the players agree on the source cases. The relevant source cases are included in the initial conditions\index{initial conditions}. The framework is not intended to provide a description of disagreements on the status of these source cases. 
		
		In terms of meaning explanation\index{meaning!explanation}, the explanation rule\index{rule!explanation} challenge can be understood as the step where the initial concessions about the target and source cases are established. When a player plays an explanation rule\index{rule!explanation}, the player concedes a statement about some analogy relative to certain target and source cases. Whenever a player states a certain analogy form\index{analogy!form} (for example PPP-analogy), this should be understood as a move that introduces some specified source cases together with the claim that this analogy form\index{analogy!form} holds over these source cases. The analogy form\index{analogy!form} statements should then be understood as: "given target case $t$ and source case $s_1, s_2, ...$ this particular analogy form\index{analogy!form} holds". This is also the reason for what at first sight seems to give the burden of proof to the challenger in these rules. Such challenge should not be understood as taking the responsibility for the proof of the analogy form\index{analogy!form}, but as a challenge of some claim about this analogy form\index{analogy!form} holding for certain cases. The analysis then presupposes agreement on these source cases in the sense that they are added directly to the initial concessions.\footnote{A natural way to further extend this analysis would be to show how the players might disagree about the status of the source cases (that they might be rejected on the basis of for example inconsistency). This will however be outside the scope here and we will therefore presuppose agreement on the target and source cases.} The agreement regarding the target case is also presupposed in this analysis. However, we might consider the target case as a presupposition for the introduction of the analogy in the first place and therefore as an initial concession that is introduced before or at the same time as the proposed analogy. This discussion is anyway not immediately relevant here as the analysis presupposes agreement also on the target case. 
		
		A similar point as the previously mentioned also holds for the permission of the analogy. When a player suggests that a certain analogy form\index{analogy!form} holds, the player should also ensure that the use of this particular analogy form\index{analogy!form} in this particular situation is permitted. Whenever a player states a certain analogy form\index{analogy!form}, one should consider that the permission of this analogy form\index{analogy!form} should be added to the initial concessions as a precondition for performing the analogical argument in the first place. 
		
			
		
		\subsection{Explanation rule\index{rule!explanation}s}
		
			The introduction of analogy explanation rule\index{rule!explanation}s requires some assumptions on their formulations. In the original CTT analysis, the analogical procedure is introduced with suspensions. This was done in order to implement the notion of an investigation or inquiry. We then introduced a formalisation that took account of the uncertainty regarding the fulfilment for each step. In the dialogical approach, this seems less pressing. Here, the notion of inquiry is directly implemented in the approach itself. The notion of 'challenge' seems to account well for the semantical notion of suspension that we wanted to introduce in the pure CTT analysis. This point, together with the intention of avoiding very complex dialogical rules, the analogy explanation rule\index{rule!explanation}s will be simplified compared to their CTT counterparts. Since the inclusion of suspensions in the formalisations seems less urgent than in the CTT analysis, we will leave them out in the dialogical implementation to avoid overly complex rules and a significant increase of moves in the plays. This is not to say that previously introduced suspensions cannot be included also in the dialogical approach, but simply that we will give priority to a simplified variant. 
			
			Particular to the dialogical approach is also the explicit introduction of the target case. There is no logical reason for not including it also in the non-dialogical CTT analysis, but a practical one to avoid an overly complex system. Dialogically, this practical matter is less pressing since we are able to distinguish the analysis into different explanation rule\index{rule!explanation}s. In each explanation rule\index{rule!explanation}, we have therefore included explicit notions of both a source case and a target case. 
			
			Each analogy form\index{analogy!form} consists of a formation rule\index{rule!formation} and an explanation rule\index{rule!explanation}. The formation rule\index{rule!formation}s describe the corresponding permission of the analogy, attached to that particular analogy form\index{analogy!form}. The explanation rule\index{rule!explanation}s challenge the chosen analogy form\index{analogy!form} by stating the inference of the result in the target case from the source case. 
			
			The formation rule\index{rule!formation}s attach the permission of the analogy form\index{analogy!form} and represents the initial condition of the utilisation of analogy in this particular situation. In CTT, any hypothetical or categorical judgment must be preceded by a type declaration where its type is specified. By using PPP-analogies as an example, this means that we must assume:
			%
				\[
				\text{PPP-Analogy}[A,B] : prop. 
				\]
			%
			The challenger can then attack this by demanding the defender to show that this kind of analogy is indeed permitted, 
				%
				\[
				\text{? }F_{PPP-Analogy[A,B]}.
				\]
			%
			 This provides the starting point for the formation rule\index{rule!formation}, as the permission of the analogy can be introduced as a defence to this challenge. In \autoref{ConditionsAnalogicalReasoningCTT} we provide the corresponding formulations of the permission of the analogy. For this example the relevant permitted-analogy form\index{analogy!form}ulation is:
			 %
			 	\[
			 	PA_1(z) : set [z : (x : A)B(x) \lor (y : A) \neg B(y)]. 
			 	\]
			 	%
			 This reads that analogies showing that when $A$ is present, $B$ is either present or absent are permitted. This is then a requirement that does not bind the target case to any particular result, as it opens up for either $B$ or $\neg B$ to be concluded. Each formation rule\index{rule!formation} is shared with one other analogy form\index{analogy!form} that provides an incompatible result. We might call this a \textit{direct counterexample\index{counterexample}}. In this example, the formation rule\index{rule!formation} is shared with the PPA-analogy. 
			 
			 After the permission of the analogy has been established, the play can continue by providing the explanation rule\index{rule!explanation} regarding the chosen analogy form\index{analogy!form}. The explanation rule\index{rule!explanation}s are based on the formulation of particular decisions in a case. Generally, we can then describe the structure of a case as a hypothetical judgment where the decision $B$ is dependent on some $A$,
			%
				\[
					b : B(x : A).
				\]
			%
			Quantified by the $\Pi$-form, a decision in a case has the form of a universal,
			%
				\[
					(\forall x : A)B(x).
				\]
			
			In order to distinguish between the target case and the source case, we also make the set that the decision belongs to explicit. The target case belongs to the set of target cases, $Tar : set$, and the source case belongs to the set of source cases, $Sou : set$. This inclusion is then done by introducing a subset separation on the set that the universal quantifies over. For the source case we get:
			%
				\[
					(\forall y : \{ x : A | Sou(x) \} ) B(y),
				\]
			and for the target case we get: 
			%
				\[
					(\forall z : \{ x : A | Tar(x) \} ) B(z).
				\]
			%
			This provides the foundation for the analysis that enables us to describe the dependency that the target case has on the source case by an implication,
			%
				\[
					(\forall y: \{ x: A| Sou(x)\} ) B(y) \supset (\forall z : \{ x: A| Tar(x)\} ) B(z).
				\]
			
			The explanation rule\index{rule!explanation}s describe the particular content of the analogy form\index{analogy!form}. First a player places a certain analogy form\index{analogy!form} into play,
			 %
			 	\[
			 		\text{! PPP-Analogy}[A,B].
			 	\]
			%
			The other player can then challenge this move by stating the particular content of this analogy form\index{analogy!form}. For the PPP-analogy this is:
			%
				\[
					\text{! } (\forall y: \{ x: A| Sou(x)\} ) B(y) \supset (\forall z : \{ x: A| Tar(x)\} ) B(z).
				\]
				which reads that we have an implication from the source case to the target case. The antecedent describes the source case and the consequent describes the target case. Both the target case and the source case are described by a universal that explains how the juridical decision $B$ is dependent on the presence of $A$ and the respective sets. The defence of this challenge is the result that is decided in the target case,
				%
				\[
					\text{! } B(z).
				\]
				%
			In a play, this means that also the defender will be obliged to state that the resulting decision also holds in the target case. Here we have used PPP-analogy as an example, though similar points hold for all forms of analogy. 
			
			In this analysis, we depend on defining target case as a set, which opens up a possibility for multiple target cases. However, in many situations this is not a possibility that is needed or wished for as we might already have an established target case that is the case for discussion. In the examples here, we will therefore include the particular target case as an initial condition in the play. In order to distinguish the target case from the source cases in the initial conditions\index{initial conditions} and in the same time ensure that they all are dependent on the same $A$ (or eventually $\neg A$), they will both be dependent on a disjunction,
			%
				\[
					a : A \lor \neg A.
				\]
				%
			The source and target cases can then be formulated as either dependent on the left injection of $a$ or on the right injection of $a$ as for example 
			%
				\[
					t : Tar(L^\lor (a))
				\]
			and
				\[
					s : Sou(R^\lor (a)).
				\]
				%
			This connects very closely to the approach in the non-dialogical CTT analysis. Since the different players agree on the source cases, we will also formulate the presence or absence of the entailed characteristic\index{characteristic!entailed} in the source cases in the initial conditions\index{initial conditions},
			%
				\[
					b : B(s).
				\]
				%
			By including such formulations as initial conditions\index{initial conditions}, the players are restricted in choosing which target case and eventually which source cases that are brought into play. 
		
			\paragraph{PPP-analogies}
			
				The present-present-present analogy is the positive and standard form for analogy, where the occasioning characteristic\index{characteristic!occasioning} is present in both the target case and the source case and where the entailed characteristic\index{characteristic!entailed} is present in the source case. Together with the PPA-analogy, the AAP-analogy and the AAA-analogy, the PPP-analogy is of the form that is called positive analogy\index{analogy!positive}. This means that the target case and the source case share the presence or the absence of some occasioning characteristic\index{characteristic!occasioning} and that we transfer the presence or absence of the entailed characteristic\index{characteristic!entailed} in the source case to the target case. The analogy can be described in the following way, where the entailed characteristic\index{characteristic!entailed} is in parentheses:
		%
                	\begin{table}[H]
                	\centering	
                	$
                    \begin{array}{ccc}
                    	\textbf{Source} &                 & \textbf{Target} \\
                    	A               &  & A               \\
                    	\downarrow      & \leftrightarrow & \downarrow      \\
                    	B               &  & (B).
                    \end{array}
               	 	$
               	 	\end{table}
               	 	%
               	 For the moves, we will use \textbf{Mov}, \textbf{Cha} and \textbf{Def} as abbreviations for \textbf{Move}, \textbf{Challenge} and \textbf{Defence} respectively. The formation of the PPP-analogy is introduced in \autoref{PPPAnalogyFormation}. The formation requirement for this form of analogy is shared with the PPA-analogy. 
               	 	
               		\begin{Scheme}[H]\footnotesize
               		\centering
               		\begin{tabular}{l l} 
               			                   & \textbf{PPP-Analogy Formation} \\ \toprule
               			\textbf{Mov}      & \textbf{X} PPP-Analogy$[A,B] : prop$ \\ \midrule
               			\textbf{Cha} & \textbf{Y} ? $F_{PPP-Analogy[A,B]}$ \\ \midrule
               			\textbf{Def}   & \textbf{X} $PA_1(z) : set [z : (x : A)B(x) \lor (y : A) \neg B(y)]$ \\ \bottomrule
               		\end{tabular}
               		\caption{PPP-Analogy Formation Rule\index{rule!formation}}
               		\label{PPPAnalogyFormation}
					\end{Scheme}               	 
               	 	
               	 \noindent The dialogical rule for performing PPP-analogies is given in \autoref{PPPAnalogyExplanation}.
               	 
               		\begin{Scheme}[H]\footnotesize
               		\centering
               		\begin{tabular}{l l}
               			                   & \textbf{PPP-Analogy}                                                                                                                                                                                                                                                                               \\ \toprule
               			\textbf{Mov}      & \textbf{X} ! PPP-Analogy$[A,B]$                                                                                                                                                                                                                                                                                     \\ \midrule
               			\textbf{Cha} & \textbf{Y} ! $(\forall y: \{ x: A| Sou(x)\} ) B(y) \supset (\forall z : \{ x: A| Tar(x)\} ) B(z)$                                                                                                                                                                                                                                                                                      \\ \midrule
               			\textbf{Def}   & \textbf{X} ! $B(z)$ \\ \bottomrule
               		\end{tabular}
               		\caption{PPP-Analogy Explanation Rule\index{rule!explanation}}
               		\label{PPPAnalogyExplanation}
					\end{Scheme}

			\newpage
			\paragraph{PPA-analogies}
			
				The present-present-absent analogy is a positive analogy\index{analogy!positive}, where the occasioning characteristic\index{characteristic!occasioning} is present in both the target case and the source case and where the entailed characteristic\index{characteristic!entailed} is absent in the source case. This analogy form\index{analogy!form} can be described in the following way:
		%
                	\begin{table}[H]
                	\centering	
                	$
                    \begin{array}{ccc}
                    	\textbf{Source} &                 & \textbf{Target} \\
                    	A               & & A               \\
                    	\downarrow      & \leftrightarrow & \downarrow      \\
                    	\neg B          & & (\neg B).
                    \end{array}
               	 	$
               	 	\end{table}
               	 	%
               	 The formation of the PPP-analogy is introduced in \autoref{PPAAnalogyFormation}. The formation requirement for this form of analogy is shared with the PPP-analogy. 
               	 	
               		\begin{Scheme}[H]\footnotesize
               		\centering
               		\begin{tabular}{l l}\footnotesize
               			                   & \textbf{PPA-Analogy Formation} \\ \toprule
               			\textbf{Mov}      & \textbf{X} PPA-Analogy$[A,B] : prop$ \\ \midrule
               			\textbf{Cha} & \textbf{Y} ? $F_{PPA-Analogy[A,B]}$ \\ \midrule
               			\textbf{Def}   & \textbf{X} $PA_1(z) : set [z : (x : A)B(x) \lor (y : A) \neg B(y)]$ \\ \bottomrule
               		\end{tabular}
               		\caption{PPA-Analogy Formation Rule\index{rule!formation}}
               		\label{PPAAnalogyFormation}
					\end{Scheme}               	 
               	 	               	 	
               	 \noindent The dialogical rule for performing analogies of this form is given in \autoref{PPAAnalogyExplanation}.
					
					\begin{Scheme}[H]\footnotesize
               		\centering
               		\begin{tabular}{l l}
               			                   & \textbf{PPA-Analogy}                                                                                                                                                                                                                                                                               \\ \toprule
               			\textbf{Mov}      & \textbf{X} ! PPA-Analogy$[A,B]$                                                                                                                                                                                                                                                                                     \\ \midrule
               			\textbf{Cha} & \textbf{Y} $(\forall y: \{ x: A| Sou(x)\} ) \neg B(y) \supset (\forall z : \{ x: A| Tar(x)\} ) \neg B(z)$                                                                                                                                                                                                                                                                                          \\ \midrule
               			\textbf{Def}   & \textbf{X} ! $\neg B(z)$ \\ \bottomrule
               		\end{tabular}
               		\caption{PPA-Analogy Explanation Rule\index{rule!explanation}}
               		\label{PPAAnalogyExplanation}
					\end{Scheme}
			
			\newpage
			\paragraph{PAP-analogies}
			
				The present-absent-present analogy is of the form of a negative analogy\index{analogy!negative}, where the occasioning characteristic\index{characteristic!occasioning} is present in  the target case, absent in the source case and where the entailed characteristic\index{characteristic!entailed} is present in the source case. This analogy form can be described in the following way:
		%
                	\begin{table}[H]
                	\centering	
                	$
                    \begin{array}{ccc}
                    	\textbf{Source} &                 & \textbf{Target} \\
                    	\neg A          &  & A               \\
                    	\downarrow      & \leftrightarrow & \downarrow      \\
                    	B               &  & (\neg B).
                    \end{array}
                    $
               	 	\end{table}
               	 	%
               	 The formation of the PAP-analogy is introduced in \autoref{PAPAnalogyFormation}. The formation requirement for this form of analogy is shared with the PAA-analogy. 
               	 	
               		\begin{Scheme}[H]\footnotesize
               		\centering
               		\begin{tabular}{l l}
               			                   & \textbf{PAP-Analogy Formation} \\ \toprule
               			\textbf{Mov}      & \textbf{X} PAP-Analogy$[A,B] : prop$ \\ \midrule
               			\textbf{Cha} & \textbf{Y} ? $F_{PAP-Analogy[A,B]}$ \\ \midrule
               			\textbf{Def}   & \textbf{X} $PA_3(z) : set [z : (x : A) \neg B(x) \lor (y : A) \neg \neg B(y)]$ \\ \bottomrule
               		\end{tabular}
               		\caption{PAP-Analogy Formation Rule\index{rule!formation}}
               		\label{PAPAnalogyFormation}
					\end{Scheme}               	 
               	 	               	 	
               	 \noindent The dialogical rule for performing analogies of this form is given in \autoref{PAPAnalogyExplanation}.
               	 
					
					\begin{Scheme}[H]\footnotesize
               		\centering
               		\begin{tabular}{l l}
               			                   & \textbf{PAP-Analogy}                                                                                                                                                                                                                                                                               \\ \toprule
               			\textbf{Mov}      & \textbf{X} ! PAP-Analogy$[A,B]$                                                                                                                                                                                                                                                                                     \\ \midrule
               			\textbf{Cha} & \textbf{Y} ! $(\forall y: \{ x: \neg A| Sou(x)\} ) B(y) \supset (\forall z : \{ x: A| Tar(x)\} ) \neg B(z)$                                                                                                                                                                                                                                                                                           \\ \midrule
               			\textbf{Def}   & \textbf{X} ! $\neg B(z)$ \\ \bottomrule
               		\end{tabular}
               		\caption{PAP-Analogy Explanation Rule\index{rule!explanation}}
               		\label{PAPAnalogyExplanation}
					\end{Scheme}
					
			\newpage
			\paragraph{PAA-analogies}
			
				The present-absent-absent analogy is of the form of a negative analogy\index{analogy!negative}, where the occasioning characteristic\index{characteristic!occasioning} is present in  the target case, absent in the source case and where the entailed characteristic\index{characteristic!entailed} is absent in the source case. This analogy form\index{analogy!form} can be described in the following way:
		%
                	\begin{table}[H]
                	\centering	
                	$
                    \begin{array}{ccc}
                    	\textbf{Source} &                 & \textbf{Target} \\
                    	\neg A          &  & A               \\
                    	\downarrow      &  \leftrightarrow & \downarrow      \\
                    	\neg B          &  & (\neg \neg B).
                    \end{array}
               	 	$
               	 	\end{table}
               	 	%
               	 	The formation of the PAA-analogy is introduced in \autoref{PAAAnalogyFormation}. The formation requirement for this form of analogy is shared with the PAP-analogy. 
               	 	
               		\begin{Scheme}[H]\footnotesize
               		\centering
               		\begin{tabular}{l l}
               			                   & \textbf{PAA-Analogy Formation} \\ \toprule
               			\textbf{Mov}      & \textbf{X} PAA-Analogy$[A,B] : prop$ \\ \midrule
               			\textbf{Cha} & \textbf{Y} ? $F_{PAA-Analogy[A,B]}$ \\ \midrule
               			\textbf{Def}   & \textbf{X} $PA_3(z) : set [z : (x : A) \neg B(x) \lor (y : A) \neg \neg B(y)]$ \\ \bottomrule
               		\end{tabular}
               		\caption{PAA-Analogy Formation Rule\index{rule!formation}}
               		\label{PAAAnalogyFormation}
					\end{Scheme}  
               	 	
               	 \noindent The dialogical rule for performing analogies of this form is given in \autoref{PAAAnalogyExplanation}.
					
					\begin{Scheme}[H]\footnotesize
               		\centering
               		\begin{tabular}{l l}
               			                   & \textbf{PAA-Analogy}                                                                                                                                                                                                                                                                         \\ \toprule
               			\textbf{Mov}      & \textbf{X} ! PAA-Analogy$[A,B]$                                                                                                                                                                                                                                                                                     \\ \midrule
               			\textbf{Cha} & \textbf{Y} ! $(\forall y: \{ x: \neg A| Sou(x)\} ) \neg B(y) \supset (\forall z : \{ x: A| Tar(x)\} ) \neg \neg B(z)$                                                                                                                                                                                                                                                                                           \\ \midrule
               			\textbf{Def}   & \textbf{X} ! $\neg \neg B(z)$ \\ \bottomrule
               		\end{tabular}
               		\caption{PAA-Analogy Explanation Rule\index{rule!explanation}}
               		\label{PAAAnalogyExplanation}
					\end{Scheme}

			\newpage
			\paragraph{APP-analogies}
			
				The absent-present-present analogy is of the form of a negative analogy\index{analogy!negative}, where the occasioning characteristic\index{characteristic!occasioning} is absent in  the target case, present in the source case and where the entailed characteristic\index{characteristic!entailed} is present in the source case. This analogy form\index{analogy!form} can be described in the following way:
		%
                	\begin{table}[H]
                	\centering	
                	$
                    \begin{array}{ccc}
                    	\textbf{Source} &                 & \textbf{Target} \\
                    	A               &  & \neg A          \\
                    	\downarrow      & \leftrightarrow & \downarrow      \\
                    	B               &  & (\neg B).
                    \end{array}
               	 	$
               	 	\end{table}
%
               	 	The formation of the APP-analogy is introduced in \autoref{APPAnalogyFormation}. The formation requirement for this form of analogy is shared with the APA-analogy. 
               	 	
               		\begin{Scheme}[H]\footnotesize
               		\centering
               		\begin{tabular}{l l}
               			                   & \textbf{APP-Analogy Formation} \\ \toprule
               			\textbf{Mov}      & \textbf{X} APP-Analogy$[A,B] : prop$ \\ \midrule
               			\textbf{Cha} & \textbf{Y} ? $F_{APP-Analogy[A,B]}$ \\ \midrule
               			\textbf{Def}   & \textbf{X} $PA_2(z) : set [z : (x : \neg A) \neg B(x) \lor (y : \neg A) \neg \neg B(y)]$ \\ \bottomrule
               		\end{tabular}
               		\caption{APP-Analogy Formation Rule\index{rule!formation}}
               		\label{APPAnalogyFormation}
					\end{Scheme} 
               	 	
               	 \noindent The dialogical rule for performing analogies of this form is given \autoref{APPAnalogyExplanation}.
					
					\begin{Scheme}[H]\footnotesize
               		\centering
               		\begin{tabular}{l l}
               			                   & \textbf{APP-Analogy}                                                                                                                                                                                                                                                                              \\ \toprule
               			\textbf{Mov}      & \textbf{X} ! APP-Analogy$[A,B]$                                                                                                                                                                                                                                                                                     \\ \midrule
               			\textbf{Cha} & \textbf{Y} ! $(\forall y: \{ x: A| Sou(x)\} ) B(y) \supset (\forall z : \{ x: \neg A| Tar(x)\} ) \neg B(z)$                                                                                                                                                                                                                                                                                         \\ \midrule 
               			\textbf{Def}   & \textbf{X} ! $\neg B(z)$ \\ \bottomrule
               		\end{tabular}
               		\caption{APP-Analogy Explanation Rule\index{rule!explanation}}
               		\label{APPAnalogyExplanation}
					\end{Scheme}
				
			\newpage		
			\paragraph{APA-analogies}
			
				The absent-present-absent analogy is of the form of a negative analogy\index{analogy!negative}, where the occasioning characteristic\index{characteristic!occasioning} is absent in  the target case, present in the source case and where the entailed characteristic\index{characteristic!entailed} is absent in the source case. This analogy form\index{analogy!form} can be described in the following way:
		%
                	\begin{table}[H]
                	\centering	
                	$
                    \begin{array}{ccc}
                    	\textbf{Source} &                 & \textbf{Target} \\
                    	A               &  & \neg A          \\
                    	\downarrow      & \leftrightarrow & \downarrow      \\
                    	\neg B          &  & (\neg \neg B).
                    \end{array}
               	 	$
               	 	\end{table}
               	 	%
               	 The formation of the APA-analogy is introduced in \autoref{APAAnalogyFormation}. The formation requirement for this form of analogy is shared with the APP-analogy. 
               	 	
               		\begin{Scheme}[H]\footnotesize
               		\centering
               		\begin{tabular}{l l}
               			                   & \textbf{APA-Analogy Formation} \\ \toprule
               			\textbf{Mov}      & \textbf{X} APA-Analogy$[A,B] : prop$ \\ \midrule
               			\textbf{Cha} & \textbf{Y} ? $F_{APA-Analogy[A,B]}$ \\ \midrule
               			\textbf{Def}   & \textbf{X} $PA_2(z) : set [z : (x : \neg A) \neg B(x) \lor (y : \neg A) \neg \neg B(y)]$ \\ \bottomrule
               		\end{tabular}
               		\caption{APA-Analogy Formation Rule\index{rule!formation}}
               		\label{APAAnalogyFormation}
					\end{Scheme} 	
               	 	
               	\noindent  The dialogical rule for performing analogies of this form is given in \autoref{APAAnalogyExplanation}.
					
					\begin{Scheme}[H]\footnotesize
               		\centering
               		\begin{tabular}{l l}
               			                   & \textbf{APA-Analogy}                                                                                                                                                                                                                                                                               \\ \toprule
               			\textbf{Mov}      & \textbf{X} ! APA-Analogy$[A,B]$                                                                                                                                                                                                                                                                                     \\ \midrule
               			\textbf{Cha} & \textbf{Y} ! $(\forall y: \{ x: A| Sou(x)\} ) \neg B(y) \supset (\forall z : \{ x: \neg A| Tar(x)\} ) \neg \neg B(z)$                                                                                                                                                                                                                                                                                           \\ \midrule
               			\textbf{Def}   & \textbf{X} ! $\neg \neg B(z)$ \\ \bottomrule
               		\end{tabular}
               		\caption{APA-Analogy Explanation Rule\index{rule!explanation}}
               		\label{APAAnalogyExplanation}
					\end{Scheme}
			
			\newpage		
			\paragraph{AAP-analogies}
			
				The absent-absent-present analogy is a positive analogy\index{analogy!positive}, where the occasioning characteristic\index{characteristic!occasioning} is absent in both the target case and the source case and where the entailed characteristic\index{characteristic!entailed} is present in the source case. This analogy form\index{analogy!form} can be described in the following way:
				%
                	\begin{table}[H]
                	\centering	
                	$
                    \begin{array}{ccc}
                    	\textbf{Source} &                 & \textbf{Target} \\
                    	\neg A          &  & \neg A          \\
                    	\downarrow      & \leftrightarrow & \downarrow      \\
                    	B               &  & (B).
                    \end{array}
               	 	$
               	 	\end{table}
               	 	%
               	 The formation of the AAP-analogy is introduced in \autoref{AAPAnalogyFormation}. The formation requirement for this form of analogy is shared with the AAA-analogy. 
               	 	
               		\begin{Scheme}[H]\footnotesize
               		\centering
               		\begin{tabular}{l l}
               			                   & \textbf{AAP-Analogy Formation} \\ \toprule
               			\textbf{Mov}      & \textbf{X} AAP-Analogy$[A,B] : prop$ \\ \midrule
               			\textbf{Cha} & \textbf{Y} ? $F_{AAP-Analogy[A,B]}$ \\ \midrule
               			\textbf{Def}   & \textbf{X} $PA_4(z) : set [z : (x : \neg A) B(x) \lor (y : \neg A) \neg B(y)]$ \\ \bottomrule
               		\end{tabular}
               		\caption{AAP-Analogy Formation Rule\index{rule!formation}}
               		\label{AAPAnalogyFormation}
					\end{Scheme} 	
               	 	
               	 \noindent The dialogical rule for performing analogies of this form is given in \autoref{AAPAnalogyExplanation}.
					
					\begin{Scheme}[H]\footnotesize
               		\centering
               		\begin{tabular}{l l}
               			                   & \textbf{AAP-Analogy}                                                                                                                                                                                                                                                                               \\ \toprule
               			\textbf{Mov}      & \textbf{X} ! AAP-Analogy$[A,B]$                                                                                                                                                                                                                                                                                     \\ \midrule
               			\textbf{Cha} & \textbf{Y} ! $(\forall y: \{ x: \neg A| Sou(x)\} ) B(y) \supset (\forall z : \{ x: \neg A| Tar(x)\} ) B(z)$                                                                                                                                                                                                                                                                                          \\ \midrule
               			\textbf{Def}   & \textbf{X} ! $B(z)$ \\ \bottomrule
               		\end{tabular}
               		\caption{AAP-Analogy Explanation Rule\index{rule!explanation}}
               		\label{AAPAnalogyExplanation}
					\end{Scheme}
					
			\newpage
			\paragraph{AAA-analogies}
			
				The absent-absent-absent analogy is a positive analogy\index{analogy!positive}, where the occasioning characteristic\index{characteristic!occasioning} is absent in both the target case and the source case and where the entailed characteristic\index{characteristic!entailed} is present in the source case. This analogy form\index{analogy!form} can be described in the following way:
				%
                	\begin{table}[H]
                	\centering	
                	$
                    \begin{array}{ccc}
                    	\textbf{Source} &                 & \textbf{Target} \\
                    	\neg A          &  & \neg A          \\
                    	\downarrow      & \leftrightarrow & \downarrow      \\
                    	\neg B          &  & (\neg B).
                    \end{array}
               	 	$
               	 	\end{table}
               	 	%
               	 The formation of the AAA-analogy is introduced in \autoref{AAAAnalogyFormation}. The formation requirement for this form of analogy is shared with the AAP-analogy. 
               	 	
               		\begin{Scheme}[H]\footnotesize
               		\centering
               		\begin{tabular}{l l}
               			                   & \textbf{AAA-Analogy Formation} \\ \toprule
               			\textbf{Mov}      & \textbf{X} AAA-Analogy$[A,B] : prop$ \\ \midrule
               			\textbf{Cha} & \textbf{Y} ? $F_{AAA-Analogy[A,B]}$ \\ \midrule
               			\textbf{Def}   & \textbf{X} $PA_4(z) : set [z : (x : \neg A) B(x) \lor (y : \neg A) \neg B(y)]$ \\ \bottomrule
               		\end{tabular}
               		\caption{AAA-Analogy Formation Rule\index{rule!formation}}
               		\label{AAAAnalogyFormation}
					\end{Scheme}	
               	 	
               	 \noindent The dialogical rule for performing analogies of this form is given in \autoref{AAAAnalogyExplanation}.
					
					\begin{Scheme}[H]\footnotesize
               		\centering
               		\begin{tabular}{l l}
               			                   & \textbf{AAA-Analogy}                                                                                                                                                                                                                                                                               \\ \toprule
               			\textbf{Mov}      & \textbf{X} ! AAA-Analogy$[A,B]$                                                                                                                                                                                                                                                                                     \\ \midrule
               			\textbf{Cha} & \textbf{Y} ! $(\forall y: \{ x: \neg A| Sou(x)\} ) \neg B(y) \supset (\forall z : \{ x: \neg A| Tar(x)\} ) \neg B(z)$                                                                                                                                                                                                                                                                                         \\ \midrule
               			\textbf{Def}   & \textbf{X} ! $\neg B(z)$ \\ \bottomrule
               		\end{tabular}
               		\caption{AAA-Analogy Explanation Rule\index{rule!explanation}}
               		\label{AAAAnalogyExplanation}
					\end{Scheme}
				
	\newpage
	\section{Explaining and extending the analysis}
	
		\subsection{Explanation of the dialogical approach}
		
		As mentioned in the beginning of \autoref{Sec:DialogicalImplementation}, this approach intends to reduce the procedure of analogical argumentation to standard dialogues of immanent reasoning\index{immanent reasoning}. The goal of this section is then to explain how the different rules can work together in order to represent the procedure of analogical reasoning. In short, we might describe the development of the explanation play by the following procedure:
		%
		\begin{enumerate}
			\item A player proposes that an analogy holds between $A$ and $B$. This player is now called \textbf{X}.
			\item The other player, \textbf{Y}, now has to challenge the claim of \textbf{X} by an Analogy Challenge Rule\index{rule!challenge} 1 (\autoref{AnalogyChallengeRule1}). This is because this rule forces \textbf{X} to choose the form of analogy that will be proposed, before \textbf{Y} chooses any analogy form\index{analogy!form} intended to counter the first one. This is reflected in the Analogy Challenge Rule\index{rule!challenge} Restriction. 
			\item \textbf{X} now has to defend the challenge in the previous step. This is done by stating a chosen form of analogy. In this step, the source cases in support of \textbf{X}'s chosen analogy form\index{analogy!form} are introduced in the initial concessions. Given these source cases, \textbf{X} claims that the chosen analogy form\index{analogy!form} holds. 
			\item \textbf{Y} then challenges the previous step by stating the type-theoretical formulation of that particular analogy form\index{analogy!form}. \textbf{Y} has to use the appropriate Analogy Explanation Rule\index{rule!explanation} corresponding to the chosen analogy form\index{analogy!form} in the previous steps.  
			\item The play will then develop in a normal way, based on the earlier defence chosen by \textbf{X}. 
			\item If the analogy form\index{analogy!form} holds, \textbf{Y} can force \textbf{X} to state the entailed characteristic\index{characteristic!entailed} in the target case as a result of defending the challenge in the previous step. If the chosen analogy form\index{analogy!form} does not hold, this cannot be done.
			\item In the next step, \textbf{Y} chooses to do the other alternative from step 2. This means to challenge the thesis by the Analogy Challenge Rule\index{rule!challenge} 2. This obliges \textbf{Y} to forward a particular analogy form\index{analogy!form}. It is here up to \textbf{Y} to choose the analogy form\index{analogy!form} that \textbf{X} should be committed to. This step introduces a subplay with the chosen analogy form\index{analogy!form} as a thesis. We might say that the roles in a subplay are turned, so that the original challenger becomes the defender and opposite. This step corresponds to bringing forward a counterexample\index{counterexample} to the proposed analogy. In this step, the source cases in support of \textbf{Y}'s chosen analogy form\index{analogy!form} are introduced in the initial concessions. Given these additional source cases, \textbf{Y} claims that the chosen analogy form\index{analogy!form} holds. For \textbf{Y} to succeed with this, \textbf{Y} has to choose an analogy form\index{analogy!form} that will be defendable and that will give an entailed characteristic\index{characteristic!entailed} that is incompatible with the entailed characteristic\index{characteristic!entailed} in the analogy form\index{analogy!form} proposed by \textbf{X}.\footnote{Notice that this step requires \textbf{Y} to have a repetition rank of $2$ or higher. If \textbf{Y}'s repetition rank is $1$, the analogy will be \emph{unrestricted}.} For \textbf{Y} to win, \textbf{X} should not be able to defend this challenge as it is intended to provide an inconsistency in \textbf{X}'s concessions.
			\item \textbf{X} then challenges the previous step by stating the type-theoretical formulation of that particular analogy form\index{analogy!form}. \textbf{X} has to use the appropriate Analogy Explanation Rule\index{rule!explanation} corresponding to the chosen analogy form\index{analogy!form} in the previous steps.
			\item The play will then develop in a normal way, based on the earlier challenge chosen by \textbf{Y}.
			\item Depending on \textbf{Y}'s choice in the Analogy Challenge Rule\index{rule!challenge} 2 and its success in the play, the result of the first analogy form\index{analogy!form} might be incompatible with the result of the second analogy form\index{analogy!form}. \textbf{X} will be committed to both results. In this way, \textbf{Y} might force \textbf{X} to posit an inconsistency and therefore win the play. This rejects the proposed analogy and might provide motivation for modifying the occasioning characteristic\index{characteristic!occasioning}. This corresponds to showing that the proposed analogy does not satisfy the condition of efficiency\index{requirement!efficiency} or the principle of proportionality\index{proportionality}. 
		\end{enumerate}	
		
		We will illustrate the functioning of these rules by a brief and simplified example in \autoref{ExamplePlayPPPPPA}, where \textbf{P} proposes a PPP-analogy between $A$ and $B$ and where \textbf{O} finds a counterexample\index{counterexample} by a PPA-analogy. In this play, the formation play is ignored for simplicity reasons. The play is also simplified in the sense that it does not account for the interaction between synthesis rules and analysis rules by the substitution of introduced variables. The play objects are marked according to whether they provide justification related to the source cases, $p^s$, to the target cases, $p^t$, or to the counterexample\index{counterexample}, $p^c$. We see here that \textbf{P} is forced to give up as he is not able to respond. 
			
			
				\begin{Play}[h] 
				\scriptsize
				\centering
				\begin{tabularx}{\textwidth}{| l | Z | l | l | Z | l |}
\hline
\multicolumn{3}{|c|}{\textbf{O}}       & \multicolumn{3}{c|}{\textbf{P}}     \\ \hline
I   & a : A \lor \neg A           &    &    &                           &    \\ \hline
II  & t : Tar(L^\lor (a))         &    &    &                           &    \\ \hline
III & s_1 : Sou(L^\lor (a))       &    &    &                           &    \\ \hline
IV  & b : B(s_1)       &    &    &                           &    \\ \hline
V   & s_2 : Sou(L^\lor (a))       &    &    &                           &    \\ \hline
    &                             &    &    & $! Analogy$[A,B]          & 0  \\ \hline
1   & $n:=2$                      &    &    & $n:=3$                    & 2  \\ \hline
3   & $? AnForm$[A,B]             & 0  &    & $! PPP-Analogy$[A,B]      & 4  \\ \hline
5 &
  $! $ (\forall y: \{ x: A| Sou(x)\} ) B(y) \newline \supset (\forall z : \{ x: A| Tar(x)\} ) \newline B(z) &
  4 &
   &
  $! $ B(p^t_2) &
  12 \\ \hline
9 &
  p^t_1 : (\forall z : \{ x: A| Tar(x)\} ) \newline B(z) &
   &
  5 &
  p^s_1 : (\forall y: \{ x: A| Sou(x)\} ) \newline B(y) &
  6 \\ \hline
7   & s_1 : \{ x: A| Sou(x)\}   & 6  &    & b : B(s_1)              & 8  \\ \hline
11  & p^t_3 : B(p^t_2)            &    & 9  & p^t_2 : \{ x: A| Tar(x)\} & 10 \\ \hline
13  & $! PPA-Analogy$[A,B]        & 0  &    &                           &    \\ \hline
\multicolumn{6}{|c|}{[Opening of subplay]}                                   \\ \hline
 &
   &
   &
  13 &
  $! $ (\forall y: \{ x: A| Sou(x)\} ) \newline \neg B(y) \supset \newline (\forall z : \{  x: A| Tar(x)\} ) \neg B(z) &
  14 \\ \hline
15 &
  p^c_1 : (\forall y: \{ x : A | Sou(x) \} ) \newline \neg B(y) &
  14 &
   &
  p^c_2 : (\forall z : \{ x: A| Tar(x)\} ) \newline \neg B(z) &
  16 \\ \hline
17  & p^t_4 : \{ x: A| Tar(x)\}   & 16 &    &                           &    \\ \hline
19  & a : A                       &    & 17 & ? L                       & 18 \\ \hline
21  & t : Tar((L^\lor (a))        &    & 17 & ? R                       & 20 \\ \hline
23  & b' : \neg B(p^c_3)          &    & 15 & p^c_3 : \{ x: A| Sou(x)\} & 22 \\ \hline
				\end{tabularx}
				\caption{Example play with PPP-Analogy and PPA-Analogy}
				\label{ExamplePlayPPPPPA}
				\end{Play}	
				%
			\begin{itemize}
				\item I to VI are the initial concessions that form the assumptions for that particular play. I formulates the disjunction of the occasioning characteristic\index{characteristic!occasioning} and its negation. II states that $A$ in I (the occasioning characteristic\index{characteristic!occasioning}) holds in the target case. This is formulated as a left instruction on the disjunction $A \lor \neg A$. III to V are introduced as claims related to the suggested analogy form\index{analogy!form}s. III states that $s_1$ is a source case and that $A$ holds in this source case. IV states that $B$ holds in the source case $s_1$. $V$ states that $s_2$ is a source case and that $A$ holds in this source case. Notice that in the situations of the target and source cases, $t$, $s_1$ and $s_2$, they are instructions on $a$, $t(L^\lor (a))$, $s_1(L^\lor (a))$ and $s_2(L^\lor (a))$.
				\item Move 0 states the proposed analogy, that there is an analogical relation between $A$ and $B$. 
				\item In moves 1 and 2, the players choose their repetition ranks. Notice that the Opponent chooses a repetition rank that is at least $2$. This is to ensure the ability to challenge the proposed thesis twice. It is this move that ensures that we are speaking about a \textit{restricted} analogy.
				\item In move 3, \textbf{O} challenges the proposed analogy in move 0 by the Analogy Challenge Rule\index{rule!challenge} 1. The Opponent here asks the Proponent to choose one analogy form\index{analogy!form} that he thinks can be defended. \textbf{O} cannot use the Analogy Challenge Rule\index{rule!challenge} 2 here as this would go against the Analogy Challenge Rule\index{rule!challenge} Restriction, which states that the second rule can only be played after the first rule has been played.
				\item In move 4, \textbf{P} chooses the analogy form\index{analogy!form} to be a PPP-Analogy, where $A$ is present in both the source case and the target case and where $B$ is present in the source case. In terms of meaning explanation\index{meaning!explanation}s, it is here that the initial concessions III and IV are introduced. 
				\item In move 5, \textbf{O} challenges \textbf{P}'s chosen analogy form\index{analogy!form} in move 4 by its corresponding explanation rule\index{rule!explanation}, here the PPP-Analogy Explanation Rule\index{rule!explanation}. 
				\item Moves 6 to 11 show a normal development according to the rules of immanent reasoning\index{immanent reasoning}. In move 6, \textbf{P} challenges the statement in move 5 by stating that the presence of $B$ in the source cases is dependent on the presence of $A$. \textbf{O} challenges this statement in move 7 by stating the source cases where $A$ is present.\footnote{We have chosen to let the Opponent counterattack in move 7 instead of defending the challenge from move 6. In this way, we are provided with the entailed characteristic\index{characteristic!entailed} in the source case before the introduction of the target case. Strategically, the order of the moves does not change anything for the development of the play, but it provides meaning explanation\index{meaning!explanation}s that seem more closely related to actual practice.} In move 8, \textbf{P} defends this challenge by stating the presence of $B$ in the source case from $IV$. In move 9, \textbf{O} defends the challenge in move 6 by stating that the presence of $B$ in the target case depends on the presence of $A$. move 10 establishes the presence of $A$ in the target case by \textbf{P} challenging the statement in move 9. We let \textbf{O} choose to simply defend this challenge for simplicity reasons. \textbf{O} can choose to challenge move 10 in a similar way as \textbf{P} challenges move 17 in move 18 and 20. This will however not add anything further to the play as \textbf{P} will simply respond based on the initial concessions in move I and II. 
				 \textbf{O} could play out also these moves by carefully choosing the target case formulation, though here we leave that implicit. In move 11, \textbf{O} defends the challenge in move 10 by stating the presence of $B$ in the target case. 
				\item In move 12, \textbf{P} then defends the challenge in move 5 by stating that $B$ is present in the target case. \textbf{P} can do this because of \textbf{O}'s statement in move 11. This finishes the first analogy form\index{analogy!form}. 
				\item In move 13, \textbf{O} challenges the thesis in move 0 again. This can be done because of \textbf{O}'s choice of repetition rank in move 1. The challenge is by the Analogy Challenge Rule\index{rule!challenge} 2, which now can be used since \textbf{O} already challenged by the Analogy Challenge Rule\index{rule!challenge} 1. This enables \textbf{O} to bring up a counterexample\index{counterexample} to the proposed analogy. It is now up to \textbf{O} to choose an analogy form\index{analogy!form}. In this example, \textbf{O} chooses the form PPA-analogy. This move also opens up a subplay where the proposed counterexample\index{counterexample} will be brought in. It is here that the initial concession V is introduced.\footnote{Notice that since $s_2$ is brought in as a counterexample\index{counterexample} by \textbf{O} we do not include the entailed characteristic\index{characteristic!entailed} in $s_2$ in the initial concessions, but directly in \textbf{O}'s move 23.} 
				\item In move 14, \textbf{P} challenges the proposed analogy form\index{analogy!form} in move 13 by the corresponding explanation rule\index{rule!explanation}, the PPA-Analogy Explanation Rule\index{rule!explanation}. 
				\item Moves 15 to 23 follow normal development according the rules of immanent reasoning\index{immanent reasoning}. In move 15, \textbf{O} challenges the statement in move 14 by claiming that the absence of $B$ is dependent on the presence of $A$ in the source cases. \textbf{P} defends this challenge in move 16 by stating that the absence of $B$ is dependent on the presence of $A$ also in the target case. \textbf{O} challenges this statement in move 17 by stating the presence of $A$ in the target case. In moves 18 to 21, \textbf{P} demands \textbf{O} to show the presence of $A$ in the target case. In move 22, \textbf{P} challenges the statement in move 15 by claiming the presence of $A$ in the source cases. \textbf{O} directly responds to this as it will assure the winning of \textbf{O}, by stating the absence of $B$ in the source case. 
				\item \textbf{P} is not able to respond to the challenge in move 17 and \textbf{O}'s counterexample\index{counterexample} is therefore effective and \textbf{O} wins the play.
			\end{itemize}
			
			In this example it is assumed that the formations have already been set. Following the notational traditions in the contemporary literature, this step is given in a play of its own. In the following section we provide a separate formation play that describes the development of the presuppositions for the explanation plays.
			
	\subsection{Formation plays and \index{analogy!permitted}permitted analogies}\label{FormationDialogAnalogy}
		
		One of the particularities of the present approach is its ability to express the permission of the analogy (and other formations). This means that we are not only able to have plays about the content of the analogy, but also about the initial conditions\index{initial conditions} that justifies performing analogical reasoning in the first place. Conceptually, such formation play is supposed to occur before every play to establish that the play is meaningful. Though in the contemporary literature, the formation plays are often left implicit or as separate plays. 
		
		As an example, we will use a play where the Proponent proposes an analogy between $A$ and $B$, based on a PPP-analogy and where the Opponent suggests a PAP-analogy as a counterexample\index{counterexample}. We choose this because it illustrates the introduction of different forms of \index{analogy!permitted}permitted analogies. In the example, given in \autoref{ExampleFormationPlay}, the description of development of the content will be left implicit. 
		
				\begin{Play}[h] 
				\scriptsize
				\centering
				\begin{tabularx}{\textwidth}{| l | Z | l | l | Z | l |}
\hline
\multicolumn{3}{|c|}{\textbf{O}}                                        & \multicolumn{3}{c|}{\textbf{P}}                  \\ \hline
I                        & A : prop                                &    &    &                                        &    \\ \hline
II                       & Tar : set                               &    &    &                                        &    \\ \hline
III                      & Sou : set                               &    &    &                                        &    \\ \hline
IV                       & B(x,y) : prop [ \newline x : A \lor \neg A, y : Tar \lor Sou] &    &    &                                        &    \\ \hline
V  & PA_1(z) : set [ z : (x : A)B(x)  \newline \lor (y : A) \neg B(y)]            &    &    &                                                                    &    \\ \hline
                         &                                         &    &    & $! Analogy$[A,B]                       & 0  \\ \hline
1                        & $n:=4$                                  &    &    & $n:=5$                                 & 2  \\ \hline
3                        & ?_{prop}                                & 0  &    & $Analogy$[A,B] : prop                  & 4  \\ \hline
5                        & $? $F_{Analogy[A,B]1}                   & 4  &    & A : prop                               & 6  \\ \hline
7                        & $? $F_{Analogy[A,B]2}                   & 4  &    & Tar : set                              & 8  \\ \hline
9                        & $? $F_{Analogy[A,B]3}                   & 4  &    & Sou : set                              & 10 \\ \hline
11                       & $? $F_{Analogy[A,B]4}                   & 4  &    & B(x,y) : prop[\newline x : A \lor \neg A, y : Tar \lor Sou] & 12 \\ \hline
\multicolumn{1}{|c|}{13} & $? AnForm$[A,B]                         & 0  &    & $! PPP-Analogy$[A,B]                   & 14 \\ \hline
15                       & ?_{prop}                                & 14 &    & $PPP-Analogy$[A,B] : prop              & 16 \\ \hline
17 & $? $F_{PPP-Analogy[A,B]}                                                      & 16 &    & PA_1(z) : set [ z : (x : A)B(x) \newline \lor (y : A) \neg B(y)] & 18 \\ \hline
\multicolumn{6}{|c|}{[Play develops normally]}                                                                             \\ \hline
19                       & $! PAP-Analogy$[A,B]                    & 0  &    &                                        &    \\ \hline
21                       & $PAP-Analogy$[A,B] : prop               &    & 19 & ?_{prop}                               & 20 \\ \hline
23 & PA_3(z) : set [ \newline z : (x : A) \neg B(x)  \newline \lor (y : A) \neg \neg B(y)] &    & 21 & $? $F_{PAP-Analogy[A,B]}                                           & 22 \\ \hline
\multicolumn{6}{|c|}{[Play develops normally]}                                                                              \\ \hline 
				\end{tabularx}
				\caption{Example play with formation}
				\label{ExampleFormationPlay}
				\end{Play}			
		
					\begin{itemize}
				\item I to V are the initial concessions that form the assumptions for that particular play. I states that $A$ is a proposition. II specifies the set of target cases. III specifies the set of source cases. IV states that $B$ is a proposition under the assumption that $A$ or $\neg A$ is present in either a source case or a target case. V states that analogies where $A$ is present in the target case are permitted. In terms of meaning explanation\index{meaning!explanation}s, V is introduced in move 14. 
				\item Move 0 states the proposed analogy, that there is an analogical relation between $A$ and $B$. 
				\item In move 1 and 2, the players choose their repetition ranks.\footnote{Notice that for simplicity of the example, the Opponent chooses a repetition rank that is $4$. This is to ensure the ability to challenge the formation of the proposed thesis by all four formation rule\index{rule!formation}s. We could also have developed the different formation rule\index{rule!formation}s in different plays and then the Opponent would only need a repetition rank of $2$.}
				\item In move 3, \textbf{O} demands \textbf{P} to show that the proposed analogy is a proposition. \textbf{P} defends this in move 4.
				\item In move 5 to 12, \textbf{O} challenges \textbf{P}'s claim in move 4 by each of the four different challenges in the Analogy form\index{analogy!form}ation Rule\index{rule!formation}. \textbf{P} defends by the corresponding claims, given in I, II, III and IV. 
				\item In move 13, \textbf{O} challenges the proposed analogy in move 0 by the Analogy Challenge Rule\index{rule!challenge} 1. The Opponent here asks the Proponent to choose one analogy form\index{analogy!form} that he thinks can be defended.
				\item In move 14, \textbf{P} chooses the analogy form\index{analogy!form} to be a PPP-Analogy, where $A$ is present in both the source case and the target case and where $B$ is present in the source case. This move also introduces the permission of the analogy in V.
				\item In move 15, \textbf{O} demands \textbf{P} to show that the proposed analogy form\index{analogy!form} is a proposition. \textbf{P} defends this in move 16.
				\item In move 17, \textbf{O} challenges \textbf{P}'s chosen analogy form\index{analogy!form} in move 16 by demanding \textbf{P} to show that the use of a PPP-analogy is permitted in the context of $B$ being analogical with $A$. \textbf{P} defends by stating $PA_1$, already stated by \textbf{O} in V. The precise formation of $PA_1$ is left implicit, though can be formulated by $A$ being a proposition and $B$ being a proposition under the assumption of $A$.
				\item In move 19, \textbf{O} challenges the thesis in move 0 again. The challenge is by the Analogy Challenge Rule\index{rule!challenge} 2, which now can be used since \textbf{O} already challenged by the Analogy Challenge Rule\index{rule!challenge} 1. This enables \textbf{O} to bring up a counterexample\index{counterexample} to the proposed analogy. It is now up to \textbf{O} to choose an analogy form\index{analogy!form}. In this example, \textbf{O} chooses the form PAP-analogy. 
				\item In move 20, \textbf{P} demands \textbf{O} to show that the proposed PAP-analogy is a proposition. \textbf{O} defends this in move 21. 
				\item In move 22, \textbf{P} challenges \textbf{O}'s statement in move 21 by asking for the permission of the use of PAP-analogy for $A$ and $B$. \textbf{O} defends this by stating $PA_3$. The precise formation of $PA_3$ is also left implicit.
			\end{itemize}
		
		
		
			
	\subsection{Introducing heteronomous imperatives\index{heteronomous imperatives}}	


		In the CTT analysis, we distinguished between analogies based on characteristics and analogies based on heteronomous imperatives\index{heteronomous imperatives}. The result was that we can defend stronger claims when dealing with deontic notions than we can when dealing with characteristics. This will naturally also be reflected in the dialogical implementation. Though contrary to the CTT analysis, we will not introduce a separate formalisation for these deontic analogies. It will suffice to add two definitional assumptions that enable the intended transformation. The rules introduced for analogies with properties will also hold for imperatives, though by themselves they will not be strong enough to capture all aspects of these inferences. Following previously introduced abbreviations, $LA$ stands for law-abiding and $LB$ stands for law-breaking.
		
		We can establish this particular inference by including the following instances of the transmission of definitional equality 1 from \autoref{TransmissionDefinitionalEquality1} as general assumptions:
			%
			\[ 
				\neg LB(x) = LA(x) : prop [x : A];
			\]
			\[
				LB(x) = \neg LA(x) : prop [x : A]. 
			\]
			%
		The first reads that $A$ not being law-breaking is equal to $A$ being law-abiding and the second that $A$ being law-breaking is equal to $A$ not being law-abiding. Generally, this seems to hold as it lies in the definition of both law-abidingness that it is not law-breaking and in law-breakingness that it is not law-abiding. They are incompatible concepts by definition. By including this definitional equality, we are able to explain the special inferences that analogies with imperatives seem to enable. 
		
		We can then treat the deontic qualification\index{deontic!qualification}s as any other proposition when doing the analogical analysis. If the result in the target case is either that the performance or non-performance of $A$ is not law-breaking, we can transform this result into it being law-abiding in the target case. Similarly, if the result in the target case is either that the performance or non-performance of $A$ is law-abiding, we can transform this result into it not being law-breaking. The first might typically be used to establish the law-abidingness of $A$ while the second might be used to establish the law-breakingness.
		
	\section{Dialogical example}\label{SteamboatExampleDialogical}
			
		\subsection{Steamboat example}
		
			We will illustrate this approach by the steamboat example, \textsf{\textit{Adams v. New Jersey Steamboat Co.}, 151 N.Y. 163 (1896)}\index{Adams v. New Jersey Steamboat Co., 151 N.Y. 163 (1896)}. The example is previously introduced and analysed in \autoref{SteamboatExampleCTT}. The example is a case about whether or not a certain steamboat owner is liable for a theft from a customer. 
			
			Following the previous treatment of this example, $F : prop$ stands for the proposition \textit{refusing strict liability for the theft of valuables when the client paid for a room for some specified reasons and that the company has tempting opportunity for fraud and plunder of the client}. In the dialogical approach, we will let $i$ represent the innkeeper case where the presence of $F$ was law-breaking and $r$ represent the railway case where the absence of $F$ was law-abiding. We can here recognise this argumentative process as a case of the Proponent suggesting a PPP-analogy between $F$ and law-breakingness, based on $i$, and where the Opponent attempts to utilise $r$ as an unsuccessful counterexample\index{counterexample} by a PAA-analogy.
		
			\subsubsection{Formation and permissibility}
		
			The first step is to establish the formation and general permissibility of the use of the analogical argument. This means to ensure that all terms are well-typed and that the use of the particular form of analogy is legally permissible\index{action!permissible} in this precise context. This is analysed in \autoref{SteamboatExampleFormation}.
			
			
				\begin{Play}[h] 
				\scriptsize
				\centering
				\begin{tabularx}{\textwidth}{| l | Z | l | l | Z | l |}
\hline
\multicolumn{3}{|c|}{\textbf{O}}                                & \multicolumn{3}{c|}{\textbf{P}}      \\ \hline
I   & F : prop                                             &    &    &                            &    \\ \hline
II  & Tar : set                                            &    &    &                            &    \\ \hline
III & Sou : set                                            &    &    &                            &    \\ \hline
IV  & LB(x,y) : prop [\newline x : F \lor \neg F, y : Tar \lor Sou] &    &    &                            &    \\ \hline
V &
  PA_1(z) : set [\newline z : (x : F)LB(x) \newline \lor (y : F) \neg LB(y)] &
   &
   &
   &
   \\ \hline
    &                                                      &    &    & $! Analogy$[F,LB]          & 0  \\ \hline
1   & $n:=4$                                               &    &    & $n:=5$                     & 2  \\ \hline
3   & ?_{prop}                                             & 0  &    & $Analogy$[F,LB] : prop     & 4  \\ \hline
5   & $? $F_{Analogy[F,LB]1}                               & 4  &    & F : prop                   & 6  \\ \hline
7   & $? $F_{Analogy[F,LB]2}                               & 4  &    & Tar : set                  & 8  \\ \hline
9   & $? $F_{Analogy[F,LB]3}                               & 4  &    & Sou : set                  & 10 \\ \hline
11 &
  $? $F_{Analogy[F,LB]4} &
  4 &
   &
  LB(x,y) : prop [\newline x : F \lor \neg F, y : Tar \lor Sou] &
  12 \\ \hline
\multicolumn{1}{|c|}{13} &
  $? AnForm$[F,LB] &
  0 &
   &
  $! PPP-Analogy$[F,LB] &
  14 \\ \hline
15  & ?_{prop}                                             & 14 &    & $PPP-Analogy$[F,LB] : prop & 16 \\ \hline
17 &
  $? $F_{PPP-Analogy[F,LB]} &
  16 &
   &
  PA_1(z) : set [\newline z : (x : F)LB(x) \newline \lor (y : F) \neg LB(y)] &
  18 \\ \hline
\multicolumn{6}{|c|}{[Play develops normally]}                                                         \\ \hline
19  & $! PAA-Analogy$[F,LB]                                & 0  &    &                            &    \\ \hline
21  & $PAA-Analogy$[F,LB] : prop                           &    & 19 & ?_{prop}                   & 20 \\ \hline
23 &
  PA_3(z) : set [\newline z : (x : F) \neg LB(x) \newline \lor (y : F) \neg \neg LB(x)] &
   &
  21 &
  $? $F_{PAA-Analogy[F,LB]} &
  22 \\ \hline
\multicolumn{6}{|c|}{[Play develops normally]}                                                         \\ \hline
				\end{tabularx}
				\caption{Steamboat example, formation play}
				\label{SteamboatExampleFormation}
				\end{Play}			
			
			We can see that this play closely resembles the example in \autoref{ExampleFormationPlay}. The only difference is that we in this example discuss a particular characteristic $F$ and its law-breakingness, rather than the characteristic $B$ that is dependent on the characteristic $A$. Similar explanations of the different moves will therefore also hold here. We will simply focus the attention on the most important aspect of this constructive approach, namely the permission of the analogy. The permission of the PPP-analogy is here introduced in the initial concessions. The permission of the PAA-analogy is instead introduced by the Opponent when proposing this form as a counterargument. 
			
			In the initial concessions, we only introduce the particular permission that is needed for this particular argument, as other forms of analogy could (at least conceptually) not be permitted. There might be a discussion whether this actually can be reflected in the legal system, though we seem to safely claim that the permission of other forms of analogy might not have been \textit{assessed}. We could have chosen to also include this second permission in the initial concessions, but by introducing it in the course of the play, our approach seems more closely connected to actual legal practice where it is the proposer of the argument that carries the burden to show that this argument is legally permitted. 
			
			
			\subsubsection{Justifying the analogy}
			
			In the formation play, no justification for the analogy is established. This is done in the particle play. This justificatory process has the form of the Proponent suggesting an analogy between $F$ and being law-breaking, on the basis of a PPP-analogy. The Opponent tries to provide a counterexample\index{counterexample} to this by a PAA-analogy, though fails since the PAA-analogy will not provide sufficient results to actually reject the PPP-analogy. The justificatory process is described in \autoref{SteamboatExample}.
			
				\begin{Play}[h]
				\scriptsize
               	\centering
				\begin{tabularx}{\textwidth}{| l | Z | l | l | Z | l |} 
\hline
\multicolumn{3}{|c|}{\textbf{O}}     & \multicolumn{3}{c|}{\textbf{P}}       \\ \hline
I   & f : F \lor \neg F         &    &    &                             &    \\ \hline
II  & t : Tar(L^\lor (f))       &    &    &                             &    \\ \hline
III & i : Sou(L^\lor (f))       &    &    &                             &    \\ \hline
IV  & d : LB(i)      &    &    &                             &    \\ \hline
V   & r : Sou(R^\lor (f))       &    &    &                             &    \\ \hline
    &                           &    &    & $! Analogy$[F,LB]           & 0  \\ \hline
1   & $n:=2$                    &    &    & $n:=3$                      & 2  \\ \hline
3   & $? AnForm$[F,LB]          & 0  &    & $! PPP-Analogy$[F,LB]       & 4  \\ \hline
5 &
  $! $(\forall y: \{ x: F| Sou(x)\} ) \newline LB(y) \supset \newline (\forall z : \{ x: F| Tar(x)\} ) LB(z) &
  4 &
   &
  $! $LB(p^t_2) &
  12 \\ \hline
9 &
  p^t_1 : (\forall z : \{ x: F| Tar(x)\} ) \newline LB(z) &
   &
  5 &
  p^s_1 : (\forall y: \{ x: F| Sou(x)\} ) \newline LB(y) &
  6 \\ \hline
7   & i : \{ x: F| Sou(x)\} & 6  &    & d : LB(i)               & 8  \\ \hline
11  & p^t_3 : LB(p^t_2)         &    & 9  & p^t_2 : \{ x: F| Tar(x)\}   & 10 \\ \hline
13  & $! PAA-Analogy$[F,LB]     & 0  &    &                             &    \\ \hline
\multicolumn{6}{|c|}{[Opening of subplay]}                                   \\ \hline
 &
   &
   &
  13 &
  $! $(\forall y: \{ x: \neg F| Sou(x)\} ) \newline \neg LB(y) \supset (\forall z : \newline \{  x: F| Tar(x)\} ) \neg \neg LB(z) &
  14 \\ \hline
15 &
  p^c_1 : (\forall y: \{ \newline x : \neg F | Sou(x) \} ) \neg LB(y) &
  14 &
   &
  p^c_2 : (\forall z : \{ \newline x: \neg F| Tar(x)\} ) \neg \neg LB(z) &
  16 \\ \hline
17  & p^t_2 : \{ x: F| Tar(x)\} & 16 &    & p^t_4 : \neg \neg LB(p^t_2) & 18 \\ \hline
19  & p^t_5 : \neg LB(p^t_2)    & 18 &    & you_{gave \ up} (21) : \bot & 22 \\ \hline
21  & p^t_6 : \bot              &    & 19 & p^t_3 : LB(p^t_2)           & 20 \\ \hline
				\end{tabularx}
               	\caption{Steamboat example}
               	\label{SteamboatExample}
				\end{Play}				
			
			\subsubsection{Describing the justificatory process}
				
				\begin{itemize}
				\item I formulates the disjunction of the occasioning proposition and its negation. II states that $F$, the left side in I, holds in the steamboat case $t$. III states that the innkeeper case $i$ is a source case and that $F$ is present in this source case. IV states that $F$ was law-breaking in $i$. V states that the railway case $r$ is a source case and that $F$ is absent in this source case. 
				\item Move 0 states the proposed analogy, that there is an analogical relation between $F$ and law-breakingness. 
				\item In moves 1 and 2, the players choose their repetition ranks. 
				\item In move 3, \textbf{O} challenges the proposed analogy in move 0 by the Analogy Challenge Rule\index{rule!challenge} 1. The Opponent here asks the Proponent to choose one analogy form\index{analogy!form} that he thinks can be defended.
				\item In move 4, \textbf{P} chooses the analogy form\index{analogy!form} to be a PPP-Analogy, where $F$ is present in both the source case and the target case and where $F$ was law-breaking in the source case. In terms of meaning explanation\index{meaning!explanation}s, it is here that the initial concessions III and IV are introduced. 
				\item In move 5, \textbf{O} challenges \textbf{P}'s chosen analogy form\index{analogy!form} in move 4 by the PPP-analogy Explanation Rule\index{rule!explanation}. 
				\item Moves 6 to 11 show a normal development according to the rules of immanent reasoning\index{immanent reasoning}. In move 6, \textbf{P} challenges the statement in move 5 by stating that the law-breakingness in the source cases is dependent on the presence of $F$. \textbf{O} challenges this statement in move 7 by stating the source case where $F$ is present, here the innkeeper case $i$. In move 8, \textbf{P} defends this challenge by stating the law-breakingness in the source case from IV. In move 9, \textbf{O} defends the challenge in move 6 by stating that the law-breakingness in the target case depends on the presence of $F$. Move 10 establishes the presence of $F$ in the target case by \textbf{P} challenging the statement in move 9. In move 11, \textbf{O} defends the challenge in move 10 by stating the law-breakingness in the target case. 
				\item In move 12, \textbf{P} then defends the challenge in move 5 by stating the law-breakingness in the target case. \textbf{P} can do this because of \textbf{O}'s statement in move 11. This finishes the first analogy form\index{analogy!form}. 
				\item In move 13, \textbf{O} challenges the thesis in move 0 again. This can be done because of \textbf{O}'s choice of repetition rank in move 1. The challenge is by the Analogy Challenge Rule\index{rule!challenge} 2, which now can be used since \textbf{O} already challenged by the Analogy Challenge Rule\index{rule!challenge} 1. This enables \textbf{O} to bring up a counterexample\index{counterexample} to the proposed analogy. \textbf{O} chooses to bring in the PAA-analogy form\index{analogy!form} as a proposed counterexample\index{counterexample}. This move opens up a subplay where the proposed counterexample\index{counterexample} will be brought in. It is here that the initial concession V is introduced. 
				\item In move 14, \textbf{P} challenges the proposed analogy form\index{analogy!form} in move 13, by the PAA-Analogy Explanation Rule\index{rule!explanation}. 
				\item Moves 15 to 22 follow normal development according the rules of immanent reasoning\index{immanent reasoning}. In move 15, \textbf{O} challenges the statement in move 16, claiming that an absence of law-breakingness is dependent on the absence of $F$ in the source cases. \textbf{P} defends this challenge by stating that the double negated law-breakingness is dependent on the absence of $F$ in the target case. \textbf{O} challenges this statement in move 17 by stating the presence $F$ in the target case. In move 18, \textbf{P} defends the challenge in move 17 by stating the double negated law-breakingness in the target case. In move 19, \textbf{O} challenges the statement in move 18 by stating that it is not law-breaking in the target case. In move 20, \textbf{P} states that the target case is law-breaking. \textbf{P} can do this because \textbf{O} already stated this in move 11. In move 21, \textbf{O} is forced to posit $\bot$. This gives \textbf{P} the possibility to state the special play object $you_{gave \ up} (21)$ and the play terminates.
				\item \textbf{P} has the last word in the play and \textbf{O}'s proposed counterexample\index{counterexample} is therefore not effective and \textbf{P} wins the play.
				\end{itemize}
				
				We can see that this corresponds well together with the intended result, where the steamboat case was decided to be similar to the innkeeper case and not to the railway case. 
			
		\subsection{Mortgage loan example}
			
			This section describes and analyses a particular example taken from Spanish Law, regarding Supreme Court decisions on the duty of paying the IAJD (Impuesto sobre Actos Jurídicos Documentados [Tax on Documented Legal Acts] related to the loan. The precise analysis of this particular case requires moves that go beyond the moves that are presented here and we will therefore provide only a semi-formal presentation of this example. The goal of this example is to show the role the different moves have in a concrete example taken from a contemporary civil law\index{Civil Law} system.\footnote{The analysis and the description in this section is a modified analysis taken from \citetitle{martinez2022elements} by \AtNextCite{\AtEachCitekey{\defcounter{maxnames}{999}}} \textcite{martinez2022elements}.} 
			
			\subsubsection{The cases}
			
			Actually, there are three main cases. However, all of them can be conceived as different plays on deciding about the interpretation of the Law concerning who must pay some particular tax specific to loans linked to a mortgage (either a mortgage loan or a credit warranted with a mortgage) included in the tax called \textit{Tax on Documented Legal Acts} (Impuesto sobre Actos Jurídicos Documentados – IAJD). More crucially, they can be seen as different plays concerning the meaning of the concepts of \textit{Mortgage Loan}, \textit{Right} (to acquire a Mortgage Loan) and \textit{Beneficiary\index{beneficiary} of a Mortgage Loan}.
			
			
			\paragraph{The first case}
			
				In the first case, Supreme Court Judgment 9012/2001\index{Sentencia del Tribunal Supremo!9012/2001}, the appellant party, the borrower \textit{Inmobiliaria Manuel Asín, S.A.} (IMA), submits a cassation appeal (an appeal to overturn the previous decision) against the decision that it is, themselves, the borrower; who is in charge of paying the IAJD tax involving the mortgage loan granted by the \textit{Caja de Ahorros y Monte Piedad de Zaragoza, Aragón y Rioja} (Ibercaja). The argument of the appellant is based on the idea that though a mortgage loan is a loan, one should distinguish the two components. In other words, the point of the appellant is that mortgage loans should be understood in the divided sense – in Islamic Jurisprudence\index{Islamic Law} such a move is called \textit{kasr}\index{kasr text@{\textit{kasr}}} or breaking apart. The point is that the IAJD tax is linked to the mortgage component, not to the loan as such. In other words, according to the appellant, the property of being a loan is not the occasioning factor for determining who is in charge of the taxes at stake. The Supreme Court dismissed the appeal, based on denying the divided reading of the notion of mortgage loans and stressing the fact that this unity also leads to the unity of beneficiary\index{beneficiary}, namely the borrower:
				%
				\begin{itquote}
					[...] it is true that the traditional interpretation of this Chamber [3rd Chamber of the Supreme Court of Spain] has always accepted the premise that the taxable event, mortgage loan, was and is unique, and therefore, the conclusion of its subjection to AJD is, nowadays, coherent, whatever the legislative tendencies that, may be in the near future, could consecrate mortgage loan exemption in this particular tax—.
				\end{itquote} 
				%
				\begin{itquote}
				In any case, the unity of the taxable event related to the loan produces the consequence that the only possible beneficiary\index{beneficiary} is the borrower, in accordance with the provision in art. $8^\circ$.d)—. 
				\end{itquote} 
				\hfill (p. 3, para. 2-3), 
				
				\hfill Supreme Court Judgment 9012/2001\index{Sentencia del Tribunal Supremo!9012/2001}\footnote{Sentencia del Tribunal Supremo 9012/2001\index{Sentencia del Tribunal Supremo!9012/2001}, dated 19. November 2001. English summary found in \textcite[pp. 245-247]{martinez2022elements}.} \medskip
				
			One way to put the issue of the interpretative contention concerning this case is to focus on the different ways the contenders build the meaning dependence between loan, mortgage and IAJD-duty.
			
			Indeed, whereas the argument of IMA, the appellant party, is based on the following meaning formations, which break apart the notions of Mortgage, Loan and the Mortgage-dependent tax duty IAJD:
			%
				\begin{itemize}
					\item $Mortgage : prop$, $Loan: prop$, \\ 
					$BIAJDDuty(x): prop (x: Mortgage)$.
					\item In words, the tax duty IAJD is dependent upon the notion of Mortgage. Accordingly, this duty is independent of the notion of Loan.
				\end{itemize}
						
			The argument of the Supreme Court in favour of Ibercaja is based on the following meaning constitution:
			%
				\begin{itemize}
					\item $Loan: prop$, $Mortgage(x): prop (x: Loan)$, \\ 
						$Beneficiary\index{beneficiary}(x,y): prop (x: Loan, y: Mortgage(x))$, \\
						$BIAJDDuty(x,y,z): prop (x: Loan, \\ y: Mortgage(x), z: Beneficiary\index{beneficiary}(x,y))$. 
					\item In words, \textit{Mortgage Loan} is a complex concept, namely it concerns those mortgages dependent upon a loan. Accordingly, the tax duty IAJD is dependent upon the complex concept Mortgage Loan, they are inseparable the notion of Loan. Moreover, the notion of Beneficiary\index{beneficiary} is made dependent upon the notion of the acquirer of the Mortgage Loan. Thus, strictly speaking, the tax duty IAJD is understood as dependent upon the \textit{Mortgage-Loan-Beneficiary\index{beneficiary}}.
					\item Notice that the beneficiary\index{beneficiary} is defined as the one that benefited of the mortgage loan. It defines the \textit{Borrower} as the beneficiary\index{beneficiary}.
				\end{itemize}

			\paragraph{Second case}

			The second case, Supreme Court Judgment 7141/2006\index{Sentencia del Tribunal Supremo!7141/2006}, also involving mortgage and loan, yields the same juridical decision as the precedent case. However, it is interesting that the reason brought forward by the Court stresses, as relevant for the decision, an aspect of the legal feature of the transaction different to the one occasioning the decision 9012/2001\index{Sentencia del Tribunal Supremo!9012/2001}. Indeed, the argument does not contest the unicity of the tax event, the credit opened by the \textit{Caixa d’Estalvis} in \textit{Pensions de Barcelona} (La Caixa) in favour of \textit{Establecimientos Industriales y Servicios, S.A.} (EISSA, S.A.) and linked with a \textit{mortgage warrant}, nevertheless, it stresses the point that the passive subject of the purchase of the \textit{right}, namely the credit, is the beneficiary\index{beneficiary}, i.e., the borrower. Thus, according to this argument of the Court, the uniqueness of the beneficiary\index{beneficiary} of this kind of transaction is the relevant feature occasioning the decision that it is the borrower’s duty rather than the lender’s duty to pay the taxes involving the mortgage. The point is that, according to the Supreme Court, the beneficiary\index{beneficiary} is the beneficiary\index{beneficiary} of the main business or of the purchase of the right. The main business is the loan, the mortgage being a subject of the loan; the beneficiary\index{beneficiary} of the loan is the borrower, namely EISSA; therefore, it is EISSA who has the duty to pay the due taxes:
				%
				\begin{itquote}
					[...] the beneficiary\index{beneficiary} is the purchaser of the good or of the right and, failing that, the persons who request notarial documents, or those in whose interest the documents are issued—.
				\end{itquote} 
				%
				\begin{itquote}
					[...] The purchaser of the good or of the right can only be the borrower, not because of an argument such as the unity of the taxable event related to the loan, [...], but because the right referred to in the precept is the loan reflected in the notarial document, even if it is guaranteed with a mortgage and its registration in the Property Registry is the constituent element of guarantee—. 
				\end{itquote} 
				\hfill (p. 3, para. 3), 
				
				\hfill Supreme Court Judgment 7141/2006\index{Sentencia del Tribunal Supremo!7141/2006}\footnote{Sentencia del Tribunal Supremo 7141/2006\index{Sentencia del Tribunal Supremo!7141/2006}, dated 31. October 2006. English summary found in \textcite[pp. 247-249]{martinez2022elements}.}\medskip
			
			From the meaning constitution point of view, the Supreme Court adds more complexity by squeezing the notion of \textit{right} (to acquire a loan), between the compound \textit{Mortgage-Loan-Beneficiary\index{beneficiary}} and the \textit{IAJD-duty}. What determines the ratio legis\index{ratio text@\textit{ratio}!\textit{legis}} is benefiting from the acquisition of the right implied in a loan, whatever the loan is:
				%
				\begin{itemize}
					\item $BIAJDDuty(x,y,z,w): prop (x: Loan, \\ y: Mortgage(x), z: Right(x,y), w: Beneficiary\index{beneficiary}(x,y,z))$. 
					\item Under this perspective; the notion of \textit{acquired right} is dependent upon the compound \textit{Mortgage-Loan-Beneficiary\index{beneficiary}}. In other words, the right is \textit{the right acquired by being the Beneficiary\index{beneficiary} of the Loan in any way attached to a Mortgage}, and the duty to the pay the IAJD is then made dependent upon this right.
					\item Hence, this alternative interpretation, that defines the right as the one acquired by the beneficiary\index{beneficiary} of the loan attached to a mortgage, also leads to identifying the \textit{Borrower} as the one who has to carry the burden of the IAJD.
				\end{itemize}
				
			\paragraph{Third case}	
				
				The last case of our study, Supreme Court Judgment 3422/2018\index{Sentencia del Tribunal Supremo!3422/2018}, also involving mortgage and loan, overturns the \textit{ratio legis}\index{ratio text@\textit{ratio}!\textit{legis}} behind the decisions deployed in the precedent cases concerning who carries the duty of paying the taxes induced by the mortgage loan. Indeed, the decision 3422/2018\index{Sentencia del Tribunal Supremo!3422/2018} establishes that it is the lender, not the borrower, who has to pay the due taxes. Moreover, it explicitly overturns juridical decisions as the ones established by Judgments 9012/2001\index{Sentencia del Tribunal Supremo!9012/2001} and 7141/2006\index{Sentencia del Tribunal Supremo!7141/2006}. The argument behind the overturning indicates that if, as argued in 7141/2006\index{Sentencia del Tribunal Supremo!7141/2006}, it is the case that the main business is the loan, i.e., the purchasing of a right, this right is not a \textit{real one}, in the sense that, for example, it does induce change of ownership. A real right is the one linked to the mortgage, but this is accessory to the right acquired by the beneficiary\index{beneficiary} and in fact the beneficiary\index{beneficiary} of that real right is the lender, not the borrower. Hence, the due taxes must be paid by the direct beneficiary\index{beneficiary} of the mortgage, namely the lender.
					%
				\begin{itquote}
					The Supreme Court held that loans are not registrable, [...], as they are obviously not a real right, nor does the right have the typical real significance mentioned in the second of these precepts (since they do not modify, now or in the future, several of the rights of ownership over real estate or inherent to real rights). The mortgage, on the other hand, is not only registrable, but it is also the mortgage is a real right—.
					
					The fact that the mortgage is a real right of registry constitution makes it clearly the main business for tax purposes in public deeds in which mortgage loans or loans with mortgage guarantee are documented—.
					
					If we still consider the loan as the main business it does not make much sense to submit to the tax a non-registrable legal business only because there is an accessory real right constituted as a guarantee of compliance with the main one.
				\end{itquote} 
				%
				The Supreme Court held also that:
				%
				\begin{itquote}
					[...] there is no doubt that the beneficiary\index{beneficiary} of the document in question is no other than the creditor, because they (and only they) are qualified to exercise the (privileged) actions that the code offers to the holders of the registered rights. They are the only party interested in the registration of the mortgage (the determining element subject to the tax analysed here), since the mortgage is ineffective if it is not registered in the Property Registry. 
				\end{itquote} 
				%
				Thus, the conclusions were:
				%
				\begin{itquote}
					\begin{enumerate}
						\item Based on the previous reasoning, we can now answer the question that we have considered preferential, out of the two questions raised by the First Section (Civil Chamber) of this Chamber (Supreme Court). The beneficiary\index{beneficiary} of a mortgage (by loan over itself or as guarantee of a loan) is the money-lender and not the borrower. Therefore, the tax on Documented Legal Acts –when the document subject to the tax is a public deed of a mortgage (by loan over itself or as guarantee of a loan)– should be paid by the lender and not by the borrower.
						\item In order to comply with the decree of admission, the above statement needs to be completed making it explicit that such a decision involves adoption of a guideline opposite to that supported by the jurisprudence of this Chamber (Third Camber – Contentious-Administrative Chamber– of the Supreme Court) until now, as presented in the judgments (STS 9012/2001\index{Sentencia del Tribunal Supremo!9012/2001} and STS 7141/2006\index{Sentencia del Tribunal Supremo!7141/2006}) among others, and therefore modifying the previous jurisprudential doctrine.
					\end{enumerate}
				\end{itquote} 
				\hfill (p. 11, para. 9), 
				
				\hfill Supreme Court Judgment 3422/2018\index{Sentencia del Tribunal Supremo!3422/2018}\footnote{Sentencia del Tribunal Supremo 3422/2018\index{Sentencia del Tribunal Supremo!3422/2018}, dated 16. October 2018. English summary found in \textcite[pp. 249-252]{martinez2022elements}.} \medskip	
				
				Supreme Court Judgment 3422/2018\index{Sentencia del Tribunal Supremo!3422/2018} concerns the request of the \textit{Empresa Municipal de la Vivienda de Rivas-Vaciamadrid, S.A.} (EMVRivas, S.A.) to be exempted of the taxes required by the \textit{Public Administration} linked to the mortgage that warranted a loan credited to EMVRivas by a bank entity. Nevertheless, as mentioned above, the decision involves a general judgment on who is the beneficiary\index{beneficiary} of the mortgage linked to a mortgage loan. The argument can again here be put as concerning meaning constitution.
				
				The main point of the Supreme Court’s argument is related to distinguishing \textit{real rights} from those acquired by taking a \textit{loan}, and more crucially, to set as beneficiary\index{beneficiary}, the beneficiary\index{beneficiary} of a real right. There are several ways to implement these distinctions, but for keeping our framework as simple as possible let us compose \textit{Loan} and \textit{Mortgage} by a conjunction. However, the notion of \textit{Real-Right} will be made dependent upon \textit{Mortgage}, furthermore, \textit{Beneficiary\index{beneficiary}} will be defined as those who acquire a \textit{Real-Right} by registering the \textit{Mortgage} (brought forward as a warrant by the borrower). Accordingly, the \textit{IAJD-duty} will be defined as the duty of the \textit{R-Beneficiary\index{beneficiary}}, i.e., the \textit{Beneficiary\index{beneficiary}} of the \textit{Real-Right}.
					%
				\begin{itemize}
					\item $Loan: prop$, \\
						$BIAJDDuty(x,y,z): prop (x: Mortgage, \\ y: RealRight(x), z: RBeneficiary\index{beneficiary}(x,y))$.
				\end{itemize}
			
			In order to stress the accessory feature of the Mortgage and the dependence of the notion of \textit{R-Beneficiary\index{beneficiary}} upon the concept of \textit{Real-Right}, and the dependence of \textit{IAJD-duty} upon the former, we can express all this as the conjunction of \textit{Loan} with the sigma-type (the existential) expressing those dependences:
			%
				\begin{itemize}
					\item $Loan \land \exists v: [(x: Mortgage, y: RealRight(x), \\ z: RBeneficiary\index{beneficiary}(x,y)) BIAJDDuty(v)]$.
				\end{itemize}
			
			\subsubsection{Presuppositions and remarks}
			
			
			All these cases are in fact cassation appeals and, as mentioned above, they all amount to say it bluntly to decide who of both, borrower or money-lender, is the beneficiary\index{beneficiary} of either a mortgage loan or a credit warranted with a mortgage, if we are prepared (or not) to distinguish the (real) right linked to the mortgage from the right acquired with the loan.
			
			Thus, we can see the three plays as sub-plays of a whole argument. However, for the sake of oversight in the present reconstruction, we will present each play by its own, where each relevant step is explicitly associated with a dialogical move. In order to keep stress, for the general structure of each argument, we adopt as starting point (the target case) the point of view of the Supreme Court. In the second and third play the source case refers to the precedent plays.
			
			The given example from Spanish law is very complex and requires moves that go beyond what is introduced in this book. We will therefore provide semiformal dialogical reconstructions of the meaning explanation\index{meaning!explanation}s behind these cases. The dialogical approach, though, facilitates an integration of such moves into more general analyses of legal reasoning and argumentation. As shown in the following section, we are able to integrate the analysis into the more complex and general plays that analyse the three concrete legal cases in a very simple and effective way.
			
			To facilitate reading we also assume the following definitions and abbreviations:
			
			\begin{tabularx}{0.95\textwidth}{l X}
				$l$ & $l : Loan$  \\
				$m$ & $m : Mortgage$  \\
				$b$ & $b : Beneficiary\index{beneficiary}$  \\
				$r$ & $r : Right$  \\
				$rr$ & $rr : RealRight$  \\
				$rb$ & $rb : RBeneficiary\index{beneficiary}$  \\
				$d$ & $d : BIAJDDuty$  \\
				$r^u : Right$ & This stands for $u$ having a legal right acquired by acquiring a mortgage loan. Thus, ‘$r^{EISSA}$’ stands for EISSA in its quality of enjoying such a right. \\
				$rr^u : RealRight$ & This stands for $u$ having a real right. Thus, ‘$rr^{UCE}$’ stands for the Unnamed Credit Entity (UCE) in its quality having a real right.  \\
				$b^u : Beneficiary\index{beneficiary}$ & This stands for $u$ being beneficiary\index{beneficiary}. Thus, ‘$b^{IMA}$’ stands for IMA in its quality as beneficiary\index{beneficiary}. \\
				$rb^u : RBeneficiary\index{beneficiary}$ & This stands for $u$ being the Real-Right beneficiary\index{beneficiary}. Thus, ‘$rb^{UCE}$’ stands for the unnamed credit entity in its quality as Real-Right beneficiary\index{beneficiary}.  \\
				$d^u : BIAJDDuty$ & This stands for $u$ being the bearer of the IAJD-duty. Thus, ‘$d^{IMA}$’ stands for IMA in its quality as bearer of the IAJD-duty.  \\
			\end{tabularx}

			For simplicity we will omit the superscript indicating the individual in the context together with the previously mentioned abbreviations. This means that 
			%
			\[ b^u: Beneficiary\index{beneficiary} (l^u: Loan, m^u: Mortgage) \] 
			%
			will be written 
			%
			\[ b^u: Beneficiary\index{beneficiary}(l,m). \]
			%
			Fore a more detailed description and background for the analysis of the different cases, see \textcite{martinez2022elements}.
			
		\subsubsection{Dialogical analysis}	
			
			\paragraph{First play (9012/2001\index{Sentencia del Tribunal Supremo!9012/2001})}
			
				The first play describes a situation where the proponent uses a PPP-analogy form\index{analogy!form} on the basis of some precedent cases decided in favour of the given meaning explanation\index{meaning!explanation}. The opponent's initial counterargument fails as it cannot be given foundation in the source cases. The analyses in this section is intended to show the role the introduced rules can play in a general dialogical framework and we will therefore leave the complete run (as explicitly described for the steamboat case) of the different forms implicit. Dialogically, this situation can be described by the following semi-formal play:

				{\scriptsize
               	\centering
				\begin{xltabular}{\textwidth}{| l | X | X | l |} 
\hline
\multicolumn{2}{|c|}{\textbf{O}}     & \multicolumn{2}{c|}{\textbf{P}}       \\ \hline
\endhead
I & Precedent-Cases decided in favour of the creditor concerning the payment of IAJD tax by a mortgage loan granted by $u$ & & \\ \hline
    &                          & IMA has to pay the IAJD-duty.           & 0  \\ \hline
1   & Why?                   & Because IMA is the beneficiary\index{beneficiary} of the mortgage loan. & 2  \\ \hline
3   & I do not agree. \textit{Mortgage Loan} should be divided in its constituents \textit{Mortgage} + \textit{Loan}. &  Develop, please. & 4  \\ \hline
\multicolumn{4}{|c|}{[Opening of subplay]}                                   \\ \hline
5   & If we divide the compounds of \textit{Mortgage Loan}, we realise that the Bearer of. the IAJD-duty should be the one having the \textit{Mortgaged} good as warranty. More precisely: \[\begin{multlined}BIAJDDuty(x): prop \\ (x: Mortgage), \end{multlined}\] given that Mortgage Loan can be divided into \[Mortgage: prop\] and \[Loan: prop.\] 
In other words, the notion of Bearer of the tax-duty is dependent upon \textit{Mortgage}. Hence the ratio legis for determining who has to pay is dependent of who has granted the mortgage. This also explains that though IMA is indeed the beneficiary\index{beneficiary} of the mortgage, it is not the bearer of the tax-duty. The divided sense allows to define the beneficiary\index{beneficiary} in the following way: \[ \begin{multlined} Beneficiary\index{beneficiary}(x): prop \\ (x: Mortgage).\end{multlined}\] & No. The notion of Mortgage Loan is a specific kind of loan and must be considered as a unity. In fact, it separates a class of loans. Thus, the meaning constitution of Mortgage Loan is \[Mortgage(x): prop (x: Loan). \]
Since it cannot be divided, the beneficiary\index{beneficiary} is the one to whom the mortgage loan has been granted: \[ \begin{multlined} Beneficiary\index{beneficiary}(x,y): prop \\ (x: Loan, y: Mortgage(x)).\end{multlined}\]
Thus, as witnessed by precedent cases, by PPP-analogy form\index{analogy!form} we get the result that the beneficiary\index{beneficiary} of a mortgage loan is the borrower $u$, right? \[b^u: Beneficiary\index{beneficiary}(l,m) ?\] & 6 \\ \hline
7 & I see. The PPP-analogy form\index{analogy!form} has indeed foundation in the precedent cases. \[b^u: Beneficiary\index{beneficiary}(l,m)\] & So, the borrower, $u$, bears the tax payment duty of the IAJD because of $u$’s role as a beneficiary\index{beneficiary} of the mortgage loan, right? \[d^u: BIAJDDuty(l,m,b)\] given that \[\begin{multlined}BIAJDDuty(x,y,z): prop \\ ((x: Loan, y: Mortgage(x), \\ z: Beneficiary\index{beneficiary}(x,y)).\end{multlined}\] & 8 \\ \hline
9 & Indeed, \[d^u: BIAJDDuty(l,m,b).\] & & \\ \hline
\multicolumn{4}{|c|}{[End of subplay]}                                   \\ \hline
11 & Yes, such a universal step seems to be grounded. & So, we can both agree that generally: \textit{if the tax duty IAJD has to be paid by whoever is the beneficiary\index{beneficiary} a mortgage loan (i.e., the borrower), then every such a borrower does}? & 10 \\ \hline
13 & Yes. \[ b^{IMA}: Beneficiary\index{beneficiary}(l,m)\] & And do you also agree that the beneficiary\index{beneficiary} in this case is IMA? \[b^{IMA}: Beneficiary\index{beneficiary}(l,m) ?\] & 12 \\ \hline
15 & Yes, if Bearer is defined in this way, IMA is the bearer of the IAJD-duty. \[d^{IMA}: BIAJDDuty(l,m,b)\] & So, you also agree that IMA is the bearer of the IAJD-duty? \[d^{IMA}: BIAJDDuty(l,m,b) ?\] & 14 \\ \hline
17 & I give up! & This is the reason that IMA is the bearer of the IAJD-duty. \[d^{IMA}: BIAJDDuty(l,m,b)\] & 16 \\ \hline
				\end{xltabular}}
				%\caption{Supreme Court Judgment 9012/2001\index{Sentencia del Tribunal Supremo!9012/2001}: IMA versus Ibercaja}
               	%\label{9012/2001}
					



			\paragraph{Second play (7141/2006\index{Sentencia del Tribunal Supremo!7141/2006})}
			
			The second play describes a situation where the proponent initiates an argument by a PPP-analogy form\index{analogy!form} and the opponent attacks this by providing an PPA-analogy form\index{analogy!form} with foundation in the precedents. This forces the proponent to explain the precise meaning explanation\index{meaning!explanation} for the analogy. By providing this precise meaning, the proponent shows that the counterargument by a PPA-analogy form\index{analogy!form} of the opponent actually does not hold since this was based on another understanding than what actually was the foundation of the analogy. Dialogically, this situation can be described by the following play:

				{\scriptsize
               	\centering
				\begin{xltabular}{\textwidth}{| l | X  | X | l |} 
\hline
\multicolumn{2}{|c|}{\textbf{O}}     & \multicolumn{2}{c|}{\textbf{P}}       \\ \hline
\endhead
I & $s_1$ : Supreme Court Judgment 9012/2001\index{Sentencia del Tribunal Supremo!9012/2001} & & \\ \hline
    &                          & EISSA has to pay the IAJD-duty.           & 0  \\ \hline
1 & Why? & In the precedent case $s_1$, the IAJD-duty has to be paid by the borrower $u$ in their quality of beneficiary\index{beneficiary}. This can be grounded by a PPP-analogy form\index{analogy!form}. Do you agree? & 2 \\ \hline
3 & However, even if in the precedent case it is the borrower who had to pay the IAJD, we can form a counterargument by the form of a PPA-analogy based on other precedent cases. & Develop, please. & 4 \\ \hline
\multicolumn{4}{|c|}{[Opening of subplay]}                                   \\ \hline
5 & There are sufficient precedent cases that indicate that being the beneficiary\index{beneficiary} of the loan warranted by a mortgage is not enough to determine that it is the borrower who is in charge of paying the registration tax. We can therefore provide foundation for a PPA-analogy form\index{analogy!form}. & Let us see. The borrower is the one who has acquired a right. Do you agree? \[r^u: Right(l,m) ? \] given that \[\begin{multlined}Right(x,y,): prop \\ (x: Loan, y: Mortgage(x)).\end{multlined}\] & 6 \\ \hline
7 & I do. \[r^u: Right(l,m)\] & Based on your endorsement, the borrower, u, is the bearer of the IAJD-duty because of $u$’s acquired the Right of being the beneficiary\index{beneficiary} of the loan, right? \[d^u: BIAJDDuty(l,m,b,r) ?\] given that \[\begin{multlined}BIAJDDuty(x,y,z,w): prop \\(x: Loan, y: Mortgage(x), \\ z: Right(x,y), \\ w: Beneficiary\index{beneficiary}(x,y,z)).\end{multlined}\] & 8 \\ \hline
9 & Yes, if benefiting of a right is added, we can form a PPP-analogy form\index{analogy!form} and claim its result. \[d^u: BIAJDDuty(l,m,r,b)\] & & \\ \hline 
\multicolumn{4}{|c|}{[End of subplay]}                                   \\ \hline
11 & Yes, such a universal generalisation can be introduced then as a rule. & So, we can both agree that generally: \textit{if the tax duty IAJD has to be paid by the borrower, whoever this borrower is, (this borrower being the one who acquired the right associated with being the beneficiary\index{beneficiary} of the loan (warranted by a mortgage) granted by the creditor), then every such borrower does}, can't we? & 10 \\ \hline
13 & Yes. \[b^{EISSA}: Beneficary(l,m,r)\] & And you agree that the part that has acquired the right associated with being beneficiary\index{beneficiary} to the loan warranted by a mortgage, in this case is EISSA? \[b^{EISSA}: Beneficary(l,m,r) ?\] & 12 \\ \hline
15 & Yes, EISSA is the bearer of the IAJD-duty. \[d^{EISSA}: BIAJDDuty(l,m,r,b) \] & So, you agree that it follows from your endorsements that the tax duty IAJD has to be paid by EISSA, as a result of EISSA acquiring the right associated with being the beneficiary\index{beneficiary} of the loan which has been warranted by a mortgage? \[d^{EISSA}: BIAJDDuty(l,m,r,b) ?\] & 14 \\ \hline
17 & Conceded! & This is the reason that EISSA is the bearer of the IAJD-duty. \[d^{EISSA}: BIAJDDuty(l,m,r,b)\] & 16 \\ \hline
\end{xltabular}}






			\paragraph{Third play (3422/2018\index{Sentencia del Tribunal Supremo!3422/2018})}
			
			The third play describes a situation where the proponent initiates an argument by a PPP-analogy form\index{analogy!form} and the opponent attacks this by providing an entirely new analogy, that will show to be inconsistent with the originally proposed analogy. By utilising a PPP-analogy form\index{analogy!form} on this new analogy, the opponent convinces the proponent of the incompatibility between the two meaning explanation\index{meaning!explanation}s and asks the proponent to provide a revised thesis based on this new analogy. Dialogically, this situation can be described by the following play:

				{\scriptsize
               	\centering
				\begin{xltabular}{\textwidth}{| l | X  | X | l |} 
\hline
\multicolumn{2}{|c|}{\textbf{O}}     & \multicolumn{2}{c|}{\textbf{P}}       \\ \hline
\endhead
I & $s_1$ : Supreme Court Judgment 9012/2001\index{Sentencia del Tribunal Supremo!9012/2001} & & \\ \hline
II & $s_2$ : Supreme Court Judgment 7141/2006\index{Sentencia del Tribunal Supremo!7141/2006} & & \\ \hline
    &                          & EMVRivas has to pay the IAJD-duty.           & 0  \\ \hline
1 & Why? & The point of the case 7141/2006\index{Sentencia del Tribunal Supremo!7141/2006} is that it amends the decision 9012/2001\index{Sentencia del Tribunal Supremo!9012/2001} by enriching the notion of beneficiary\index{beneficiary} with the notion of right, so that it is the borrower u that is the beneficiary\index{beneficiary} of the right associated with the mortgage loan. So, we should have \[b^u: Beneficiary\index{beneficiary}(l,m,r) ?\] Do you agree? & 2 \\ \hline
3 & I do. \[b^u: Beneficiary\index{beneficiary}(l,m,r) \] & By a PPP-analogy form\index{analogy!form}, this leads to the 7141/2006\index{Sentencia del Tribunal Supremo!7141/2006}-conclusion that it is the beneficiary\index{beneficiary} of the right associated with the mortgage loan that is the bearer of the IAJD-duty, right? \[d^u: BIAJDDuty(l,m,r,b) ?\] & 4 \\ \hline
5 & No, I do not agree. This is inconsistent with another meaning explanation\index{meaning!explanation} for this notion, and this indicates that advocated \textit{ratio legis}must be revised. I therefore propose an entirely new analogy and I will show that a based on a PPP-analogy form\index{analogy!form}, we will get a new result, incompatible with the your point in step 4. & Develop, please. & 6 \\ \hline
\multicolumn{4}{|c|}{[Opening of subplay]}                                   \\ \hline
7 & The creditor $v$ is the one who has the real right being associated with the mortgage itself. Do you agree? \[rr^v: RealRight(m) ?\] & Yes, I do. \[rr^v: RealRight(m)\] & 8 \\ \hline
9 & Do you also agree that the creditor $v$ is the beneficiary\index{beneficiary} as a result of having acquired the real right associated with the mortgage itself? \[rb^v: RBeneficiary\index{beneficiary} (m,rr) ?\] & Yes, they are the beneficiary\index{beneficiary} of the real right. \[rb^v: RBeneficiary\index{beneficiary} (m,rr)\] & 10 \\ \hline
11 & So, since the beneficiary\index{beneficiary} is the creditor, and the beneficiary\index{beneficiary} is the bearer of the IAJD-duty, then it follows that it is the creditor who is in charge of paying the due taxes, right? \[d^v: BIAJDDuty(m,rr,rb) ?\] & Yes, that does indeed seem to be following from this notion of beneficiary\index{beneficiary} of the real right. \[d^v: BIAJDDuty(m,rr,rb) \] & 12 \\ \hline
13 & And you also agree that the money-lender and the borrower cannot both be bearers of the IAJD-duty, so that your statement in step 4, corresponding to the conclusions contained in the cases 9012/2001\index{Sentencia del Tribunal Supremo!9012/2001} and 7141/2006\index{Sentencia del Tribunal Supremo!7141/2006} is incompatible with your statement in step 12? & Yes, my statement in step 4 corresponding to the conclusions contained in the cases 9012/2001\index{Sentencia del Tribunal Supremo!9012/2001} and 7141/2006\index{Sentencia del Tribunal Supremo!7141/2006} is incompatible with my statement in step 12. My statement in step 4 must be given up and the meaning explanation\index{meaning!explanation}s must be overturned. & 14 \\ \hline
15 & The occasioning factor determining the ratio legis is being the beneficiary\index{beneficiary} of the real right associated to warranting a mortgage. Recall that the beneficiary\index{beneficiary} of a real right is the one who benefits of a change of property. Do you agree now that the appropriate meaning explanation\index{meaning!explanation} of beneficiary\index{beneficiary} is one who acquired a real right associated with the mortgage? \[\begin{multlined}RBeneficiary\index{beneficiary}(x,y): prop \\(x: Mortgage, \\ y: RealRight(x)) ?\end{multlined}\] & I agree. \[\begin{multlined}RBeneficiary\index{beneficiary}(x,y): prop \\ (x: Mortgage, \\ y: RealRight(x))\end{multlined}\] & 16 \\ \hline
17 & Do we furthermore agree that the meaning explanation\index{meaning!explanation} for the bearer of the IAJD-duty is the beneficiary\index{beneficiary} that acquired the real right associated with the mortgage? \[\begin{multlined}BIAJDDuty(x,y,z): prop \\(x: Mortgage, \\ y: RealRight(x), \\ z: RBeneficiary\index{beneficiary} (x,y)) ?\end{multlined}\] & Indeed. \[\begin{multlined}BIAJDDuty(x,y,z): prop \\ (x: Mortgage, \\ y: RealRight(x), \\ z: RBeneficiary\index{beneficiary} (x,y))\end{multlined}\] & 18 \\ \hline 
19 & I ask you therefore to revise your original statement, based on an overturning of the meaning explanation\index{meaning!explanation}s in the precedent cases 9012/2001\index{Sentencia del Tribunal Supremo!9012/2001} and 7141/2006\index{Sentencia del Tribunal Supremo!7141/2006} as argued for from step 7 to 19. & & \\ \hline 
\multicolumn{4}{|c|}{[End of subplay]}                                   \\ \hline
& & Based on your new analogy I will defend the following revised thesis: UCE has to pay the IAJD-duty. & 20 \\ \hline
21 & I agree. Though, please develop your justification for your revised thesis. & Recall, since the beneficiary\index{beneficiary} of the real right associated with the mortgage is the creditor, and this beneficiary\index{beneficiary} is the bearer of the IAJD-duty, then it follows that is the creditor who is in charge of paying the due taxes, right? \[d^v: BIAJDDuty(m,rr,rb) ?\] & 22 \\ \hline
23 & Yes, it does. \[d^v: BIAJDDuty(m,rr,rb)\] & Do you agree on the following general formulation: \textit{If the tax duty IAJD has to be paid by the money-lender, whoever this creditor is (this creditor being the one who acquired the real right associated with being the (real) beneficiary\index{beneficiary} of the registration of the mortgage linked to the loan granted by this creditor), then every such creditor does}? & 24 \\ \hline
25 & Yes, the generalisation is now grounded. & Do you agree then that it follows that the meaning explanation\index{meaning!explanation}s that provide the conclusions of the precedent cases have to be overturned? And further, in the particular case of the UCE, having the real right of the mortgage, do you agree that it is also the bearer of the IAJD-duty? \[d^{UCE}: BIAJDDuty(m,rr,rb) ?\] & 26 \\ \hline
27 & Yes. I concede both the overturning and the conclusion. \[d^{UCE}: BIAJDDuty(m,rr,rb)\] & This is the reason that you demanded in step 21 for my revised thesis that UCE is the bearer of the IAJD-duty. \[d^{UCE}: BIAJDDuty(m,rr,rb)\] & 28 \\ \hline
\end{xltabular}}

	
		\subsection{Results}
			
			\subsubsection{Dialogues, counterexample\index{counterexample}s and interactions}
			
				In the dialogical approach we might distinguish between material and formal dialogue\index{dialogue!formal}s. We are here in the context of \textit{material} dialogues. This means that the plays will be reasoning with content, rather than being purely formal or logical. This is reflected in the way the players interact with each other, in the sense that their ability to win a play is dependent on some material facts about the source and target cases. 
				
				\paragraph{Interacting rules}
				
				This dialogical analysis introduces different rules that together provide a way to deal with the notion of analogical reasoning. There is here a distinction between what is called \textit{challenge rule\index{rule!challenge}s\index{rule!challenge}} and \textit{explanation rule\index{rule!explanation}s}. In addition, there is also a structural rule\index{rule!structural} regarding the use of the challenge Rule\index{rule!challenge}s\index{rule!challenge}, namely the \textit{Analogy Challenge Rule\index{rule!challenge} Restriction}. We argue that by combining these rules, we can provide an analysis of analogical reasoning for the legal context. As previously described, the idea is to let the player proposing the analogy first explain the foundation for the proposed analogy, namely to suggest one analogy form\index{analogy!form} that holds for the source cases. At this step we are still only speaking about an \textit{unrestricted} analogy, as the player can choose any analogy form\index{analogy!form} that can be confirmed by at least one source case. The next step is to let the other player suggest a counterexample\index{counterexample} to the proposed analogy form\index{analogy!form}. The second player must then assure two things on the choice of counterexample\index{counterexample}, first that the proposed analogy form\index{analogy!form} actually can provide a claim that is incompatible with the first players claim, and second that this proposed analogy form\index{analogy!form} can be confirmed by at least one source case. In the steamboat example, the Opponent looses because of the first requirement, namely that the proposed analogy form\index{analogy!form} did not provide an actual counterexample\index{counterexample} even though it had a confirmation in the source cases. This means that even when the railway case provided justification for the PAA-analogy, the PAA-analogy form\index{analogy!form} is not sufficient to provide a counterexample\index{counterexample} to a PPP-analogy. 
				
				The idea behind this analysis of analogical reasoning is that the different explanation rule\index{rule!explanation}s can provide counterexample\index{counterexample}s to each other, based on how they affect the entailed characteristic\index{characteristic!entailed} in the target case. An analogy form\index{analogy!form} that provides justification for $\neg B$ in the target case can be used as a counterexample\index{counterexample} to an analogy form\index{analogy!form} that provides justification for $B$ in the target case, and opposite. Also, an analogy form\index{analogy!form} that provides justification for $\neg \neg B$ in the target case can be used as a counterexample\index{counterexample} to an analogy form\index{analogy!form} that provides justification for $\neg B$ in the target case, and opposite. This enables us to categorise the different analogy form\index{analogy!form}s, based on how they provide counterexample\index{counterexample}s to each other, shown in \autoref{AnalogyFormCounter}.
				
				
					\begin{table}[h]
               		\centering
               		\begin{tabular}{c c c c c}
               			$B$  & $\sim$ & $\neg B$ & $\sim$ & $\neg \neg B$                                                                                                                                                                                                                                                                               \\ \toprule
               			PPP  && PPA && PAA \\
               			AAP  && PAP && APA \\
               				  && APP && \\
               				&&	AAA &&
						\\ \bottomrule
               		\end{tabular}
               		\caption{Analogy form\index{analogy!form}s, counterexample\index{counterexample}s}
               		\label{AnalogyFormCounter}
					\end{table}				
				
				'$\sim$' should simply be understood as incompatibility. We see from this table that the forms PPP and AAP can be used as counterexample\index{counterexample}s to the forms PPA, PAP, APP and AAA and opposite. PAA and APA can be used as counterexample\index{counterexample}s to the last ones mentioned, though they do not produce justification for $B$, only for $\neg \neg B$, which are not intuitionistically equivalent. This means that these particular analogy form\index{analogy!form}s cannot provide justification for $B$, but still can provide justification for a counterexample\index{counterexample} to $\neg B$. The reason for this is that they are \textit{negative analogies}, namely that they are based on a difference in the source case and the target case. In terms of the Proportionality\index{proportionality}-principle, they provide justification for not treating the cases similarly, which is not the same as to provide justification for how the cases should be treated. This corresponds to a distinction between what we might call direct and indirect justification. We claim here that indirect justification (showing that the alternatives are unjustified) is not sufficient for establishing justification for the initial claim and that this is the reason for not enabling these negative analogies to provide justification for $B$ instead of only $\neg \neg B$.  

				\paragraph{Player independence and creativity}

				All rules are described player independently. This means that none of them are specific for either the Proponent or the Opponent. The advantage of this is first that it provides natural meaning explanation\index{meaning!explanation}s where both agents are described as equals and second that it can be used as a foundation to describe more complex argumentative processes. In the steamboat example, the Opponent provided a counterexample\index{counterexample} by the same analogy as was proposed by the Proponent, namely between $F$ and $LB$. However, because of the player independent rules, the Opponent could also have responded to this by providing an entirely new analogy. For the Opponent to succeed in doing this, he must of course choose an analogy that will actually be incompatible with the initial analogy form\index{analogy!form} by the Proponent. Such move would open up a subplay in a very similar manner as with the Analogy Challenge Rule\index{rule!challenge} 2. In the steamboat example, an alternative analysis would be to consider that the Opponent proposes a new analogy between $F$ and $LA$, and justifies it by a PAP-analogy. This will yield the same result as in the described play because of the definitional incompatibility between $LA$ and $LB$. 
				
				We could have included the proposal of a new analogy as a ninth way of challenging in the Analogy Challenge Rule\index{rule!challenge} 2, though this threatens the finiteness of the plays, as every play could be potentially infinite. From the perspective of meaning, it is not sure that this is unwanted as it provides a natural way to understand legal discourse as a never-ending discussion of specification, interpretation and re-interpretation. However, from the point of view of dialogues and strategies, we do want to ensure the possibility to settle on a, even if temporary decision. This is the reason for leaving the question on introducing new analogies into a play rather open. Proposing a new analogy in this way seems to demand an aspect of \textit{creativity} that resists thorough logical description. It corresponds closely to what \textcite{Brewer1996} calls \textit{the abductive step} and seems then also related to the argumentative move of \textit{inference to the best explanation}. 
				
				We do not intend to provide an analysis of such creativity, though simply note that if such move were to be introduced in a play, it might directly be implemented into the analysis as a potential counterexample\index{counterexample} (or eventually a counter-counterexample\index{counterexample} and so on), in the same way as any other example. We claim then to be able capture by the introduced rules, the assessment of the move after its introduction, but because of its creative aspect, not the introduction of the move itself. 
				
				\paragraph{Combining notions}
				
				A common challenge for many logical analyses of juridicial phenomena is the combination of deontic and modal\index{modality} notions, which in many frameworks gives rise to a number of paradoxes.\footnote{See for example \textcite{navarro2014deontic} for a detailed account of many standard paradoxes.} In immanent reasoning\index{immanent reasoning} this is avoided by precisely allowing both the modal\index{modality} and the deontic notions to be based on a type-theoretical foundation. 
				
				The deontic imperatives\index{deontic!imperative} (also the modal\index{modality} notions of possibility and necessity) are defined by a type-theoretical formulation and are not introduced as distinct operators. This allows for a framework where are free of most common paradoxes of deontic logic\index{deontic!logic}.\footnote{See \textcite{Rahman2019} for comments on how many paradoxes simply do not occur when depending on a type-theoretical definition of deontic imperatives\index{deontic!imperative}.} The most important reason for this is that the framework interprets the performance of actions directly in the object language, rather than considering them in a model or higher order predicate. What allows for this is the particular notion of dependent types\index{dependent types}, found only in CTT and closely related extensions. 
				
				With this analysis of analogical reasoning, we have also included deontic imperatives\index{deontic!imperative} directly into our description of analogical reasoning. This is a rather unconventional strategy as many theories would consider these questions separately. However as mentioned, these questions become interconnected in CTT as we consider actions in the object language. For analogies, this first enables us to see the particular aspect of analogical reasoning. Second, it shows how analogies might be used for reasoning with actions, not only propositions. We can thereby explain how a conclusion of an analogical argument might be the performance of some action, not simply a proposition describing a rule. This apprehends the point of Jørgensen's dilemma, by showing how analogical reasoning can occur not only with truth and falsity, but also with actions, rules and principles. 

			\subsubsection{Initial permission of analogy}
			
				One of the core features of this analysis is its ability to represent initial conditions\index{initial conditions} concerning the permission of the use of an analogical argument. This seems to be feature newly introduced here, not found in other contemporary logical analyses or representations of analogical reasoning. Even though other analyses do not include such permission, it is frequently highlighted as an essential aspect in the utilisation of analogical argumentation, especially related to Civil Law\index{Civil Law} and European legal systems. A particular advantage in the given analysis in immanent reasoning\index{immanent reasoning} is that the the initial permissions can be assessed in the same play as the content of the analogy. This kind of interplay between higher- and lower-order notions is a particularity of CTT that we do not find in other frameworks. 
				
				\paragraph{Legal restrictions\index{restrictions!legal}}
				
				In many civil-law inspired systems, there might be situations or areas where the use of analogical arguments is limited because of for example constitutional restrictions\index{restrictions!constitutional} or for going against some fundamental values. Most contemporary legal systems incorporate some variant of the principle of legality, or \textit{nulla poena sine lege}. This principle is often interpreted as prohibiting the use of analogies in penal law for imposing sanction\index{sanction}s. An analogy is in this sense understood as an extension of the present legal framework and such extension can thereby not be used for grounding punishment, as the action was not illegal at the time of the crime. This corresponds to what by \textcite{Langenbucher2017} is called a \textit{rule-based\index{analogy!rule-based} analogy}. The use of \textit{principle-based\index{analogy!principle-based} analogies} on the other hand might be permitted also in penal law as it provides what is considered to be a precision or interpretation of an already existing rule, rather than an extension of the present legal framework. The distinction between the two variants and their permissibility in different legal contexts might vary, both between different legal systems and between particular cases or situations. 
				
				There might also be restrictions on the basis of some fundamental values. We might consider the use of analogical argumentation as impermissible\index{action!permissible} if it breaks or goes against some values that are considered to be essential for the state, the legal system or individuals as such. An example is the Norwegian Constitution (Grunnlova) that states:
					%
					\begin{itquote}
						Verdigrunnlaget skal framleis vere den kristne og humanistiske arven vår. Denne grunnlova skal tryggje demokratiet, rettsstaten og menneskerettane.
					\end{itquote}
					%
					\begin{displayquote}
						Our values will remain our Christian and humanist heritage. This Constitution shall ensure democracy, a state based on the rule of law and human rights.
					\end{displayquote}
					\hfill Grunnlova (The Constitution), 1814 \S 2 \medskip
			
				If the use of an analogical argument goes against human rights or undermines the democracy or the rule of law, it should not be permitted. In EU law, the use of an analogical argument that threatens the goals of the union might for example also be impermissible\index{action!permissible}. 
				
				One might consider the condition of \textit{being in a context of doubt} or \textit{lacuna}\index{lacuna text@\textit{lacuna}} as an initial condition on the use of analogy. However, in the dialogical approach we can understand this condition in two ways. First, it can be an initial condition. Second, it can be a way to challenge the analogy in the play by providing some explicit rule on the matter. The difference between the two is whether we allow for the analogical argument to be introduced before challenging it by a rule, or whether we reject the introduction of the analogy in the first place because of a lacking context of doubt. These two ways of understanding the context of doubt are not incompatible and we might very well consider both to be present in the framework.
				
				The mentioned restrictions for the use of analogical arguments should be considered as examples, and surely not as an exhaustive list of requirements. The effect these restrictions have on analogical arguments seems to depend heavily on the particular legal framework we operate within and legal practice within that framework. The goal here is not to enter into the discussion on the content of these requirements for permitting analogical reasoning, as this would demand a proper logical treatment in itself. It is rather to enable analysing these restrictions as prohibiting the use of analogical arguments. The analogy is then not rejected by the quality of the analogy, but rather because of the context that this argument is used within. 
				
				\paragraph{Structure of permission}
				
				The analysis provides four different variants of the permission of analogies, where the form resembles:
				%
					\[
			 			PA_1(z) : set [z : (x : A)B(x) \lor (y : A) \neg B(y)]. 
			 		\]
			%
				This statement is a condition for the formulation of a particular analogical form. Each formulation assesses analogies of two forms, which is why we are provided with four formulations of a permission of an analogy, rather than eight. In this example, we permit analogy form\index{analogy!form}s that provide justification for
					\[
						(x : A)B(x) 
					\]
					and
					\[
						(x : A)\neg B(x).
					\]
				
				Generally, their structure permits an analogy based on some $A$ to infer some $B$ or its negation. They formulate what we can call \textit{direct} counterexample\index{counterexample}s. If one accepts an analogical argument between some $A$ and some $B$, one must also accept the possibility of providing a direct counterexample\index{counterexample}, $\neg B$, to this argument. If each permission of analogy only consisted of one part, we could risk ending up with a \textit{unrestricted} analogy simply because of lacking permission. Such analogy should be given very little consideration in terms of justification. This is the reason for making this condition twofold, so that the permission of the analogy always ensures the ability of providing a counterexample\index{counterexample}. By connecting them with a disjunction, we represent the possibility of either side to fulfil. The permission of the analogy, $PA(z)$, is formulated as a set on the basis of the given disjunction. When permitting the analogy, we therefore do not commit to any given result in the target case. 
				
				\paragraph{Permission and dialogues}
				
				In the CTT analysis, the \index{analogy!permitted}permitted analogies are all provided as conditions for the formulation of the analogy itself. This means that one has to accept the permission of \textit{all} analogy form\index{analogy!form}s when providing an analogy of \textit{one} form. In the dialogical approach, this gets more refined. Here we connect the permission of the analogy not to the analogy in itself, but rather to the particular analogy form\index{analogy!form}. This seems to be closer to actual legal practice as one would not have to explicitly permit analogies of all forms when using only a few. It avoids so to speak to assess the permissibility of "irrelevant" analogy form\index{analogy!form}s. There is no logical reason for this not to be done also in the CTT analysis, but the dialogical approach offers a way to this in a natural and comprehensible way without entering into overly complex formulations. 
				
				The goal here is to show how the given analysis of analogies can provide a way to formulate the initial conditions\index{initial conditions} imposed on the use of analogical reasoning. Such condition is commonly found in the contemporary literature, though usually left implicit or unexplained in the theoretical analyses. By the higher-order notation of dependent types\index{dependent types}, the framework of CTT enables us to formulate such restriction explicitly and thereby provide further descriptions of the meaning explanation\index{meaning!explanation}s regarding the use of analogies in legal reasoning.	

			\subsubsection{Proportionality\index{proportionality} and dialogues}
			
				\paragraph{Compounds}
				
				In \autoref{ExplainingRelations}, we argued for considering analogical reasoning as a question of \textit{similar relevancy\index{relevancy}}, in line with Aristotle\index{Aristotle}'s conception of proportionality\index{proportionality}. In this dialogical analysis, we consider the transfer from the source case to the target case by means of a hypothetical, represented in the formalisations as an intuitionistic implication. This hypothetical is what represents the notion of similar relevancy\index{relevancy} in this analysis. In both the source case and the target case, we have a formulations of relevance from $A$ to $B$ and the hypothetical captures how these relevancies are similar and how one can transfer such dependency in the target case, based on a similar dependency in the source case. Note that this is not a material conditional so it captures the transference of the content of the source case to the content of the target case, not simply their truth value. Contrary to other analyses\footnote{An exception is the Islamic model by \textcite{Rahman2017}, which is also based on immanent reasoning\index{immanent reasoning}.} we see here that such formalisation really captures the important interaction found between the similarity and the relevancy\index{relevancy} found in analogies by not simply considering them separately as two distinct imposed requirements. By considering analogies in CTT, we are able to reason with content, not only with truth and therefore capture such inference as directly related to proportionality\index{proportionality}.

				This analysis captures analogical reasoning with single properties and compound conjunctive and hypothetical properties. This means that the occasioning characteristic\index{characteristic!occasioning} that the analogy is based upon can have the form of either a single predicate or as a complex predicate consisting of a conjunction or a hypothetical. We have seen how this play out in the Steamboat example where we had a conjunctive property that was analysed as a single predicate. However, this approach has certain limits in the sense that it cannot analyse interpretations of the content of \textit{disjunctive compounds}. A legal concept can be said to consist of a disjunctive compound when something falls under that concept if it is A or B, but not necessarily both. This analysis can account for using such properties as a foundation for further reasoning, but not for interpreting the precise content of this compound. Say that there is a question whether something that has the property C falls under a certain concept. The question is then whether this concept is a disjunction of three properties, $A \lor B \lor C$, or whether it is a disjunction of a conjunction, $A \lor (B \land C)$. This requires an analysis of the interpretation of the content of this particular concept. In order to capture such interpretation, one will need to depend on some extension of the presented framework that can give an account of the content of such disjunctive compounds, not only use them to show some further property. To also include an analysis of disjunctive compounds would then seem like a natural, future extension to the present project. 
				
				\paragraph{Multiple\index{analogy!multiple} analogies}
				
				In \autoref{ComparingTheoriesAnalogy}, the way different theories of analogy handled multiple\index{analogy!multiple} analogies was heavily emphasised. The introduced analysis is very much in line with the approaches by \textcite{Rahman2017} and \textcite{Brewer1996}. We have therefore not introduced a higher-order operator that deals with ranking as in \textcite{Bartha2010} and \textcite{Prakken1996}, but we rather consider multiple\index{analogy!multiple} analogies by means of \textit{new moves} in a play. A competing analogy in this sense can either be a regular counterexample\index{counterexample} with a new analogy form\index{analogy!form} or as a newly introduced analogy. As described previously, the introduction of a new analogy seems to depend on a \textit{creative} aspect that is difficult to capture logically. Though, as soon as any new analogy is introduced, we can describe its precise interaction with any previously introduced analogy in the same way as the approach handles regular counterexample\index{counterexample}s. Since the analysis recognises the introduction of both a new analogy and an analogy form\index{analogy!form} as a \textit{proposition}, the framework also enables reasoning with analogies as with any other proposition. This includes considering an analogy as a characteristic in another analogy. We might in this way have an analogy of analogies (or an analogy of an analogy of analogies and so on). This is done without leaving the object language. It is a significant feature when representing legal reasoning as it enables the analysis of interpreting interpretations in a direct and natural way, without assuming any hierarchy of the interpretations. This important feature seems absent in the approaches of \textcite{Bartha2010} and \textcite{Prakken1996} as their object language only permits interpretations and analogies of a single level. 
			
				\paragraph{Policy\index{policy} and psychology}
			
				As described in \autoref{WeinrebPosner}, there seems to be a tension between what might be called a psychological view of analogies and a policy\index{policy}-based view. In the psychological view, the characteristics that ground the analogy are determined by some psychological or epistemological state. In the policy\index{policy}-based view, these characteristics are rather determined by a rule or policy\index{policy}. This analysis does not intend to take any standpoint in this discussion as the meaning explanation\index{meaning!explanation}s given here seem able to capture both views. However, it does seem to engage in the conflict as the dialogical approach can be used to resolve parts of this tension. 
				
				The motivation for developing a contemporary dialogical approach to logic is to bring back the ancient tradition of dialectics by uniting logic, argumentation and rhetoric. The main idea in a simple dialogical framework is that one agent, the Proponent, tries to convince another agent, the Opponent, of a certain claim. In this analysis of analogies, the Proponent tries to convince the Opponent of the quality of an analogical argument. It is not based on rules or similarities as such, but on giving and accepting \textit{reasons}. A reason here can be understood both as providing some consistent rule or policy\index{policy} that can explain the analogy, or as some relevant aspect that grounds the notion of relevant similarities. In this sense, the dialogical approach opens up for both views and do not take a standpoint in what these reasons should consist of. 
				
				However, because of its game-theoretical foundation, the ultimate goal for the dialogical approach relates to the notion of \textit{accept}. The precise content of these reasons might vary, as long as they are reasons that can be rationally accepted by both players. Something counts as a reason insofar it can be mutually recognised as such. Furthermore, we might say that something counts as a \textit{good} reason insofar it can be \textit{rationally} recognised as such. Reasoning can then be understood in terms of rational rhetoric. 
				
				The reluctancy towards rules and policies found in \textcite{weinreb2005legal} is based on the lack of explicitly formulated rules in practical argumentation, while \posscite{posner2006reasoning} scepticism to \citeauthor{weinreb2005legal}'s psychological view comes from the lack of requiring general understanding. \citeauthor{weinreb2005legal} and \citeauthor{posner2006reasoning} do not disagree on the logical requirements we impose on good analogies, as both presuppose rational guidelines for analogical reasoning. By representing analogical reasoning as a rational game of giving and accepting reasons, we seem to resolve parts of this tension. In a game, agents (players) bring forward and accept reasons, which explains the psychological aspect that these reasons have. In the same time, for these reasons to be good, they must contain some general aspect that explains the how these reasons should be rationally accepted. As mentioned, we do not claim to have resolved this debate, only to have highlighted how answering seems less pressing when the goal is to provide the meaning explanation\index{meaning!explanation}s behind analogical reasoning. The important aspect is then not the precise content of the given reasons, but how they can be used practically for resolving disagreements in a rational way.			
			
				
			
			
			
			
			
			
			
			
			
			
			
			
			
			
			
			
			